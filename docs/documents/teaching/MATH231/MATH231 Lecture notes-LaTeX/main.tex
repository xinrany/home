\documentclass[11pt,letterpaper,reqno]{amsart}
\usepackage[margin=1in]{geometry}



%------------------------------
% title page
%------------------------------
\title{MATH231 Calculus II}
\date{\today}
\author{Xinran Yu}
\address{University of Illinois at Urbana-Champaign}
\email{xinran4@illinois.edu}



%------------------------------
% page setting
%------------------------------
\setlength{\marginparwidth}{20mm}
% boldsymbol
\usepackage{stmaryrd}
\SetSymbolFont{stmry}{bold}{U}{stmry}{m}{n}
\pagestyle{plain}
\usepackage[foot]{amsaddr}
\usepackage[T1]{fontenc}
\pagenumbering{arabic} 


%------------------------------
% toc indent
%------------------------------
\makeatletter
\setcounter{tocdepth}{3}

% Add bold to \section titles in ToC and remove . after numbers
\renewcommand{\tocsection}[3]{%
    \indentlabel{\@ifnotempty{#2}{\bfseries\ignorespaces#1 #2\quad}}\bfseries#3}
% Remove . after numbers in \subsection
\renewcommand{\tocsubsection}[3]{%
    \indentlabel{\@ifnotempty{#2}{\ignorespaces#1 #2\quad}}#3}
%\let\tocsubsubsection\tocsubsection% Update for \subsubsection
%...
\renewcommand{\tocsubsubsection}[3]{%
    \indentlabel{\@ifnotempty{#2}{\ignorespaces#1 #2\quad}}#3}
\newcommand\@dotsep{4.5}
\def\@tocline#1#2#3#4#5#6#7{\relax
    \ifnum #1>\c@tocdepth % then omit
    \else
        \par    \addpenalty\@secpenalty\addvspace{#2}%
        \begingroup \hyphenpenalty\@M
        \@ifempty{#4}{%
        \@tempdima\csname r@tocindent\number#1\endcsname\relax
        }{%
            \@tempdima#4\relax
        }%
        \parindent\z@ \leftskip#3\relax \advance\leftskip\@tempdima\relax
        \rightskip\@pnumwidth plus1em \parfillskip-\@pnumwidth
        #5\leavevmode\hskip-\@tempdima{#6}\nobreak
        \leaders\hbox{$\m@th\mkern \@dotsep mu\hbox{.}\mkern \@dotsep mu$}\hfill
        \nobreak
        \hbox to\@pnumwidth{\@tocpagenum{\ifnum#1=1\bfseries\fi#7}}\par% <-- \bfseries for \section page
        \nobreak
        \endgroup
    \fi}
\AtBeginDocument{%
\expandafter\renewcommand\csname r@tocindent0\endcsname{0pt}
}
\def\l@subsection{\@tocline{2}{0pt}{2.5pc}{5pc}{}}
\def\l@subsubsection{\@tocline{2}{0pt}{4.5pc}{5pc}{}}

\makeatother



%------------------------------
% section style
%------------------------------
\usepackage{etoolbox}
\makeatletter
\renewcommand{\thesection}{\Roman{section}} % Use Roman numerals without a dot
\patchcmd{\section}{\@secnumfont{\csname thesection\endcsname}\quad}%
    {\@secnumfont{\csname thesection\endcsname}\hspace{0.5em}}{}{}
    
\renewcommand\section{\@startsection{section}{1}{\z@}%
    {-3.5ex \@plus -1ex \@minus -.2ex}%
    {2.3ex \@plus.2ex}%
    {\centering\Large\scshape}}


\patchcmd{\subsection}{\@secnumfont{\csname thesubsection\endcsname}\quad}%
    {\@secnumfont{\csname thesubsection\endcsname}\hspace{0.5em}}{}{}
    
\renewcommand\subsection{\@startsection{subsection}{1}{\z@}%
    {-3.5ex \@plus -1ex \@minus -.2ex}%
    {2.3ex \@plus.2ex}%
    {\centering\large\scshape}}

\patchcmd{\subsubsection}{\@secnumfont{\csname thesubsubsection\endcsname}\quad}%
    {\@secnumfont{\csname thesubsubsection\endcsname}\hspace{0.5em}}{}{}

\renewcommand\subsubsection{\@startsection{subsubsection}{2}{\z@}%
    {1.5ex \@plus .2ex}%
    {-1em}%
    {\bfseries}}  % Bold and inline with no extra space above
\makeatother
\let\oldsection\section
\renewcommand\section{\clearpage\oldsection}

\renewcommand{\thesubsection}{\arabic{subsection}}

\renewcommand{\thesubsubsection}{\arabic{subsubsection}}




%------------------------------
% enumeration setting
%------------------------------ 
\usepackage[inline]{enumitem}
\renewcommand{\theenumi}{\roman{enumi}}
\setlist[enumerate]{leftmargin = 5ex}
\setlist[itemize]{label = {\tiny$\bullet$}, leftmargin = 4ex}
\usepackage{tikz}
\protected\def\circled#1{%
\tikz[baseline]{\node[anchor = base,inner sep = 0pt,draw,circle]{#1};}%
}
\usepackage{pgfplots}



%------------------------------
% general packages
%------------------------------
\usepackage{amsmath}
\usepackage{amsfonts, mathrsfs} % mathbb, mathscr
\usepackage{mathtools}
\usepackage{tensor}
\usepackage{slashed} % Dirac
\usepackage{stmaryrd} % owedge




%------------------------------
% theorem environment
%------------------------------
\newcommand{\bfemph}[1]{\textbf{\textit{#1}}}


\usepackage{amsthm}
\theoremstyle{plain}
\newtheorem{thm}{Theorem}[subsection]
\newtheorem{lem}[thm]{Lemma}
\newtheorem{coro}[thm]{Corollary}
\newtheorem{prop}[thm]{Proposition}
\newtheorem{rmk}[thm]{Remark}
\newtheorem{qn}[thm]{Question}

\theoremstyle{definition}
\newtheorem{defn}[thm]{Definition}

\newtheorem{examplex}[thm]{Example}
\newenvironment{ex}
  {\pushQED{\qed}\renewcommand{\qedsymbol}{$\bullet$}\examplex}
  {\popQED\endexamplex}



%------------------------------
% color
%------------------------------
\usepackage{xcolor}
\definecolor{enji}{RGB}{159, 53, 58}   % dark red
\definecolor{akabeni}{RGB}{203, 64, 66}   % light red
\definecolor{kohaku}{RGB}{202, 122, 44}  % orange
\definecolor{darkgreen}{RGB}{66, 111, 83}
\definecolor{shinbashi}{RGB}{53, 143, 183}   % blue
\definecolor{purple}{RGB}{72, 75, 135}


% hyper ref
\usepackage[colorlinks = true,
            linkcolor = shinbashi,
            urlcolor  = shinbashi,
            citecolor = shinbashi,
            anchorcolor = shinbashi]{hyperref}

% box
\usepackage{tcolorbox}
\tcbset{colframe=shinbashi!70, colback=shinbashi!5, fonttitle=\bfseries}


% colored text
\newcommand{\red}[1]{\textcolor{enji}{#1}}
\newcommand{\orange}[1]{\textcolor{kohaku}{#1}}

\newcommand{\green}[1]{\textcolor{darkgreen}{#1}}
\newcommand{\blue}[1]{\textcolor{shinbashi!85!black}{#1}}
\newcommand{\purple}[1]{\textcolor{purple}{#1}}
\newcommand{\gray}[1]{\textcolor{gray}{#1}}

%------------------------------
% track changes
%------------------------------
\usepackage[commandnameprefix=always,  defaultcolor=kohaku]{changes}

\definechangesauthor[name={Xinran Yu}, color=darkgreen]{XY}
\usepackage{tcolorbox}
\tcbset{colback=darkgreen!10,
    colframe=darkgreen, 
    fonttitle=\bfseries, 
    width = \textwidth
}

%------------------------------
% tikz picture set up
%------------------------------
\usepackage{tikz-cd}
\usepackage{tikz-3dplot}
\usepackage{tikz,bm}
\usetikzlibrary{angles,arrows,calc,quotes}
\usepackage[edges]{forest}

\usepackage{physics}
\usepackage{siunitx}
\usepackage[outline]{contour} % glow around text
% for pic
\contourlength{1.3pt}
%\allowdisplaybrevaks


\tikzset{>=latex} % for LaTeX arrow head

\colorlet{veccol}{darkgreen}
\colorlet{vcol}{darkgreen}
\colorlet{xcol}{shinbashi}
\colorlet{projcol}{xcol!60}
\colorlet{unitcol}{xcol!60!black!85}
\colorlet{unitcol2}{vcol!60!black!85}
\colorlet{myblue}{blue!70!black}
\colorlet{myred}{red!70!black}
\tikzstyle{vector}=[->,very thick,xcol]
\tikzstyle{mydashed}=[dash pattern=on 2pt off 2pt]
\def\tick#1#2{\draw[thick] (#1) ++ (#2:0.1) --++ (#2-180:0.2)} %0.03*\xmax


%------------------------------
% other packages
%------------------------------
\usepackage{adjustbox}
\usepackage{graphicx}
\graphicspath{ {images/} }
\usepackage{caption}
\usepackage{subcaption}
\usepackage{float}



%------------------------------
% new symbols
%------------------------------
% limit, integral, derivatives
\renewcommand{\dd}{\;\mathrm{d}}
\newcommand{\pd}{\partial}
\newcommand{\dlim}{\displaystyle\lim} 
\newcommand{\dint}{\displaystyle\int} 
\newcommand{\dsum}{\displaystyle\sum} 

% norm, inner product
\newcommand\inner[2]{\langle #1, #2 \rangle}

%%%%%%%%%% operatorname %%%%%%%%%%
\newcommand{\id}{\operatorname{id}}
\newcommand{\im}{\operatorname{im}}

%%%%%%%%%% bb, cal, scr %%%%%%%%%%
% bb
\newcommand{\C}{\mathbb{C}}
\newcommand{\N}{\mathbb{N}}
\newcommand{\R}{\mathbb{R}}
\newcommand{\Z}{\mathbb{Z}}
% cal
\newcommand{\cA}{\mathcal{A}}
\newcommand{\cB}{\mathcal{B}}
\newcommand{\cC}{\mathcal{C}}
\newcommand{\cD}{\mathcal{D}}
\newcommand{\cE}{\mathcal{E}}
\newcommand{\cF}{\mathcal{F}}
\newcommand{\cG}{\mathcal{G}}
\newcommand{\cH}{\mathcal{H}}
\newcommand{\cI}{\mathcal{I}}
\newcommand{\cJ}{\mathcal{J}}
\newcommand{\cK}{\mathcal{K}}
\newcommand{\cL}{\mathcal{L}}
\newcommand{\cM}{\mathcal{M}}
\newcommand{\cN}{\mathcal{N}}
\newcommand{\cO}{\mathcal{O}}
\newcommand{\cP}{\mathcal{P}}
\newcommand{\cQ}{\mathcal{Q}}
\newcommand{\cR}{\mathcal{R}}
\newcommand{\cS}{\mathcal{S}}
\newcommand{\cT}{\mathcal{T}}
\newcommand{\cU}{\mathcal{U}}
\newcommand{\cV}{\mathcal{V}}
\newcommand{\cW}{\mathcal{W}}
\newcommand{\cX}{\mathcal{X}}
\newcommand{\cY}{\mathcal{Y}}
\newcommand{\cZ}{\mathcal{Z}}

% scr
\newcommand{\sA}{\mathscr{A}}
\newcommand{\sC}{\mathscr{C}}
\newcommand{\sG}{\mathscr{G}}
\newcommand{\sH}{\mathscr{H}}
\newcommand{\sO}{\mathscr{O}}
\newcommand{\sP}{\mathscr{P}}
\newcommand{\sS}{\mathscr{S}}
\newcommand{\sV}{\mathscr{V}}



%------------------------------
% document begins
%------------------------------
\begin{document}

\maketitle
\tableofcontents

\textbf{NB:}
\begin{itemize}
    \item \S[number] in the margin refers to the section covered in the book 
    \begin{center}
        James Stewart, \textit{Calculus: Early Transcendentals}, 8th edition, 2016.
    \end{center}
    \item "Typo22" means there is a typo in the handwritten notes from Spring 2022.
\end{itemize}


\clearpage
\setlength{\parindent}{0pt}
\setlength{\parskip}{1.5ex}


%------------------------------
% body
%------------------------------
\section*{An overview of Calculus II}
\vspace{1cm}
\begin{figure}[H]
    \centering
    \resizebox{\textwidth}{!}{\tikzset{
    my node/.style={
    draw=darkgreen,
    inner color=darkgreen!10,
    outer color=darkgreen!10,
    thick,
    minimum width=1cm,
    rounded corners=3,
    text height=2ex,
    text depth=0.5ex,
  }
}

\begin{forest}                      
    forked edges, for tree={%
        my node, l sep+=5pt, grow'=east,
    }
[Calculus II
    [Ch 7-10 
        [Integrals
        [Techniques
            [Anti Derivative/FTC]
            [Integration by Parts]
            [Substitution [Trig Substitution]]       
            [Trig Integrals]
            [Rational Function Decomposition]]
        [Types
            [Proper Integrals]
            [Improper Integrals]]
        [Applications
            [Arc Length]
            [Surface Area]
            [Physics]]
        ]
        [2D Curves
            [Coordinates
                [Rectangular]
                [Polar]
            ]
            [Description
                [By Graph of a Function]
                [By Parametrization]
            ]
        ]
    ]
    [Ch 11 Series
        [Sequences]
        [Series]
        [Tests and Convergence
            [Integral Test and Estimates of Sums]
            [Comparison Tests]
            [Alternating Series]
            [Absolute Convergence and the Ratio and Root Tests]
            [Strategy for Testing Series]
        ]
        [Power Series
            [Representation of Functions by Power Series]
            [Taylor and Maclaurin Series]
            [Applications of Taylor Polynomial]
        ]
    ]
]
\end{forest}
} % Include your TikZ file
    \label{fig:mindmap}
\end{figure}

\section{Calculus I Review}
\begin{center}
\begin{tcolorbox}
    \begin{itemize}
        \item Examples can be found in the assignment HW0. 
    
        \item Make sure you understand how to take \textit{derivatives} and compute \textit{limits} and the basic \textit{integrals}, as these concepts are essential for Calculus II.
    \end{itemize}
\end{tcolorbox}
\end{center}


\subsection{Limits, Derivatives, and Integrals}

\subsubsection{Limits Laws}
\begin{enumerate}
    \item $\dlim_{x \to c} (f(x) + g(x)) = \dlim_{x \to c} f(x) + \dlim_{x \to c} g(x)$
    \item $\dlim_{x \to c} (f(x) \cdot g(x)) = \dlim_{x \to c} f(x) \cdot \dlim_{x \to c} g(x)$
    \item $\dlim_{x \to c} \frac{f(x)}{g(x)} = \frac{\dlim_{x \to c} f(x)}{\dlim_{x \to c} g(x)}$ if $\dlim_{x \to c} g(x) \neq 0$
\end{enumerate}


\subsubsection{L'H\^{o}pital's Rule}
When $\dlim_{x \to c} f(x) = 0$ and $\dlim_{x \to c} g(x) = 0$, we may apply L'H\^{o}pital's Rule
\[\lim_{x \to c} \frac{f(x)}{g(x)} = \lim_{x \to c} \frac{f'(x)}{g'(x)}.\]

\begin{ex}[``Counterexample'' to L'H\^{o}pital's Rule]
    \[\lim_{x \to 0} \frac{x + \sin x}{x} = DNE.\]
L'H\^{o}pital's Rule does not apply because $\dlim_{x \to 0} x + \sin x = 1 \neq 0$.
\end{ex}


\subsubsection{Derivatives}
\begin{defn}
    \[f'(x) = \lim_{h \to 0} \frac{f(x+h) - f(x)}{h}.\]
\end{defn}

\textbf{Derivative Rules}
\begin{itemize}
    \item Product Rule: $\dfrac{d}{dx} [f(x) g(x))] = f'(x) \cdot g(x) + f(x) \cdot g'(x)$,
    \item Chain Rule: $\dfrac{d}{dx} [f(g(x))] = f'(g(x)) \cdot g'(x)$.
\end{itemize}

\textbf{Derivatives of elementary functions}
\[
\begin{aligned}
    &\frac{d}{dx} e^x = e^x, \quad \frac{d}{dx} \sin x = \cos x, \quad \frac{d}{dx} \cos x = -\sin x, \\
    &\frac{d}{dx} \ln x = \frac{1}{x}, \quad \frac{d}{dx} \tan x = \sec^2 x.
\end{aligned}
\]

\newpage
\subsection{Integration}
\subsubsection{Definition}
Reversing the process of differentiation:
\begin{defn}
    \[\int_a^b f(x) \dd x = \lim_{\Delta x \to 0} \sum_{i=1}^n f(a + \Delta x \cdot i) \cdot \Delta x.\]
\end{defn}

\subsubsection{Integration Laws}
\begin{itemize}
    \item $\dint (f(x) \pm g(x)) \dd x = \dint f(x) \dd x \pm \dint g(x) \dd x$,
    \item $\dint c \cdot f(x) \dd x = c \cdot \dint f(x) \dd x$.
\end{itemize}

\subsubsection{Fundamental Theorem of Calculus}

\begin{thm}[Fundamental Theorem of Calculus (FTC)]
\begin{enumerate}
    \item If $f$ is continuous on $[a,b]$, then the function $F$ defined by 
    \[F(x) = \int_a^x f(t) \, dt \]
    is continuous on $[a,b]$ and differentiable on $(a,b)$, and $F'(x)=f(x)$ for all $x\in(a,b)$.
    
    \item If $f$ is continuous on $[a,b]$ and $F$ is an antiderivative of $f$ on $[a,b]$, then
    \[\int_a^b f(x) \, dx = F(b) - F(a).\]
\end{enumerate}
\end{thm}

\subsubsection{Substitution Rule}
Let $u = g(x)$, then:
\[
\int f(g(x)) g'(x) \dd x = \int f(u) \, du.
\]
\section{Chapter 7}
\subsection{Integration by Parts} \chcomment{\S7.1} \chcomment{Week 1}
\begin{center}
\begin{tcolorbox}
    \begin{itemize}
        \item \textbf{New method for integrals}: Integration by parts provides a technique to evaluate integrals of \textit{products} of functions.
        \item \textbf{Inverse of the product rule}: If $ \dfrac{d}{\dd x}(uv) = u \dfrac{dv}{\dd x} + v \dfrac{\dd u}{\dd x} $, then
        \begin{equation}
            \int u \dd v = uv - \int v \dd u \tag{IBP}
        \end{equation}
    \end{itemize}
\end{tcolorbox}
\end{center}

\subsubsection{Motivation}
Using the \textit{Fundamental Theorem of Calculus}, we know the antiderivative of basic functions such as $x$ and $e^x$:
\begin{align*}
\int x \dd x &= \dfrac{x^2}{2} + C, \\
\int e^x \dd x &= e^x + C.
\end{align*}
However, for functions like $\ln x$ or $\tan x$, a new tool is required: \textbf{Integration by Parts}. 

\subsubsection{Formula Derivation}
\textbf{Key idea: product rule for differentiation}

\begin{proof}
    Let $u,v$ be functions of $x$. Recall that the product rule for differentiation formula says
    \[(uv)' = u'v + uv'. \]
    Integrating both sides with respect to $x$ gives
    \[\int uv' \dd x = uv - \int u'v \dd x.\]
    Rearranging gives the formula for integration by parts:
    \[ \int u \dd v = uv - \int v \dd u. \]
\end{proof}


\subsubsection{Example} \chcomment{There were typo in the statement of this question before.}
\begin{ex}[Evaluate $\dint \ln x \dd x$]
    Take $u = \ln x$ and $dv = \dd x$. Then:
    \begin{align*}
        \int \ln x \dd x &= x \ln x - \int x \cdot \dfrac{1}{x} \dd x \\
        &= x \ln x - \int 1 \dd x \\
        &= x \ln x - x + C.
    \end{align*}
\end{ex}

\begin{ex}[Evaluate $\dint x \ln x \dd x$] \chcomment{Here's how to solve $\dint x \ln x \dd x$}
    Take $u = \ln x$ and $dv = x \dd x$. Then $\dd u = \dfrac{1}{x} \dd x $ and $v = \dfrac{1}{2} x^2$:
    \begin{align*}
        \int \ln x \dd x &= \dfrac{1}{2}x^2 \ln x - \int \dfrac{1}{2}x^2 \cdot \dfrac{1}{x} \dd x \\
        &= \dfrac{1}{2}x^2 \ln x - \int \dfrac{1}{2} x \dd x \\
        &= \dfrac{1}{2}x^2 \ln x - \dfrac{1}{4}x^2 + C.
    \end{align*}
\end{ex}


\subsubsection{Steps to Apply IBP}
\begin{enumerate}
    \item Identify $u$ and $\dd v$(the LIATE rule can be used, see below).
    \item Compute $\dd u$ and $v$.
    \item Substitute the above expressions into the IBP formula and evaluate.
\end{enumerate}

\textbf{Choosing $u$ and $v$ (LIATE Rule)}:
When applying integration by parts, the choice of $u$ and$\dd v$ can be guided by the LIATE rule. Take $u$ to be the function that appears earlier in the list.
\begin{itemize}
    \item \textbf{L}ogarithmic functions $\ln x, \log_a x$
    \item \textbf{I}nverse trigonometric functions $\arcsin x, \arctan x$ etc.
    \item \textbf{A}lgebraic functions $x^a$
    \item \textbf{T}rigonometric functions $\sin x, \cos x$ etc.
    \item \textbf{E}xponential functions $e^x, a^x$
\end{itemize}

However, in general, there is no easy way to immediately determine which function to choose as $u$. In practice, you won't need to remember this rule, as the computation becomes second nature.



\subsubsection{More Examples}
\begin{ex}[Evaluate  $\dint \arctan x \dd x$]
    Let $ u = \arctan x $ and $ dv = \dd x $. Then:
\[
\dd u = \dfrac{1}{1+x^2} \dd x, \quad v = x
\]

Substitute into the integration by parts formula:
\begin{align*}
    \int \arctan x \dd x &= x \arctan x - \blue{\int x \cdot \dfrac{1}{1+x^2} \dd x} \\
    &= x \arctan x - \blue{\dfrac{1}{2} \ln(1+x^2)} + C.
\end{align*}

\blue{The last step is done by substituting $w = 1+x^2$:
\[
\int x \cdot \dfrac{1}{1+x^2} \dd x = \dfrac{1}{2} \int \dfrac{\dd w}{w} = \ln w+ C.
\]}
\end{ex}

\textbf{As in the example above, there are situations where both \textit{integration by parts} and \textit{substitution} are needed.}

\begin{comment}
    \begin{ex}[Evaluate  $\dint x e^x \dd x$]
    Take $u = x$ and $dv = e^x \dd x$. Then:
    \begin{align*}
        \int x e^x \dd x &= x e^x - \int e^x \dd x \\
        &= x e^x - e^x + C \\
        &= e^x (x - 1) + C.
    \end{align*}
\end{ex}
\end{comment}
Here's another example.

\begin{ex}[Evaluate $\dint \dfrac{x^3}{\sqrt{1+x^2}} \dd x$]
Take $ u = x^2 $ and $ dv = \dfrac{x}{\sqrt{1+x^2}} \dd x $, so:
\[
\dd u = 2x \dd x, \quad v = \sqrt{1+x^2}.
\]

Substitute into the integration by parts formula:
\begin{align*}
    \int \dfrac{x^3}{\sqrt{1+x^2}} \dd x &= \int x^2 \cdot \dfrac{x}{\sqrt{1+x^2}} \dd x = x^2 \sqrt{1+x^2} - \blue{\int 2x \sqrt{1+x^2} \dd x}.
\end{align*}

\blue{To compute $\dint 2x \sqrt{1+x^2} \dd x$: Using substitution, let $ w = 1+x^2 $, then $ \dd w = 2x \dd x $. We have
\begin{align*}
    \int 2x \sqrt{1+x^2} \dd x &= \int \sqrt{w} \dd w = \dfrac{2}{3} w^{3/2} + C = \dfrac{2}{3} (1+x^2)^{3/2} + C.
\end{align*}}
So the original integral is:
\begin{align*}
    \int \dfrac{x^3}{\sqrt{1+x^2}} \dd x &= x^2 \sqrt{1+x^2} - \dfrac{2}{3} (1+x^2)^{3/2} + C.
\end{align*}
\end{ex}
You may notices that there is no need to apply the IBP at all. Here's another way to solve the same problem.
\begin{ex}[The same problem with substitution]
    Using substitution rule, we let $u = 1 + x^2 $ so that $\dd u = 2x \dd x$. Then
\begin{align*}
    \int \dfrac{x^3}{\sqrt{1+x^2}} \dd x &= \int \dfrac{x^2 \cdot x}{\sqrt{u}} \dd x = \dfrac{1}{2} \int \dfrac{u-1}{\sqrt{u}} \dd u \\
    &= \dfrac{1}{2} \int u^{1/2} \dd u - \dfrac{1}{2} \int u^{-1/2} \dd u \\
    &= \dfrac{1}{3} u^{3/2} - u^{1/2} + C \\
    &= \dfrac{1}{3} (1 + x^2)^{3/2} - \sqrt{1 + x^2} + C.
\end{align*}
\end{ex}

\newpage
\subsection{Trigonometric Integrals} \chcomment{\S7.2}
\begin{center}
\begin{tcolorbox}
    \begin{itemize}
        \item \textbf{Particular type of integral}: \[\int \sin^n x \cos^m x \dd x.\]
        \item \textbf{Tools to use}: 
        \begin{itemize}
            \item Trig formulae and identities (to reduce the powers of $\sin x$ or $\cos x$)
            \item Substitution rule
            \item Integration by parts
        \end{itemize}
        \item \textbf{To memorize}:
        \begin{align*}
            &\sin^2 x + \cos^2 x = 1, \\
            &\sin(2x) = 2\sin x \cos x,  \qquad \cos(2x) = \cos^2 x - \sin^2 x
        \end{align*}
        The other formulae can be derived from the above.
    \end{itemize}
\end{tcolorbox}
\end{center}

In this section, we are interested in solving integrals of the form: 
\[\int \sin^n x \cos^m x \dd x\]
where $n, m$ are integers. (You will see in the homework that $n$ and $m$ could be noninteger).

We first recall the trigonometric formulae and identities. 
\begin{center}
    \renewcommand{\arraystretch}{2.5}
    \begin{tabular}{  p{0.35\textwidth} p{0.35\textwidth}  }
        $\red{\sin^2 x + \cos^2 x = 1}$ & $\tan^2 x + 1 = \sec^2 x$ \\
        $ \red{\sin(2x) = 2\sin x \cos x} $ & $ \red{\cos(2x) = \cos^2 x - \sin^2 x} $ \\
        $\displaystyle \sin x = \pm\sqrt{\dfrac{1 - \cos (2x)}{2}} $ & $\displaystyle \cos x = \pm\sqrt{\dfrac{1 + \cos (2x)}{2}} $
    \end{tabular}
\end{center}
It suffices to remember the equations in red; the rest can be derived from them. (Try it).


\subsubsection{First Example of Trigonometric Integral}
Before discussing the general approach, let's first look at an example to motivate the method and the overall strategy.
\begin{ex}[Evaluate $\dint \sin^3 x \cos^2 x \dd x$] \leavevmode
    Apply the substitution rule. Let $u = \cos x$, then $\dd u = -\sin x \dd x$. We get
    \begin{align*}
        \int \sin^3 x \cos^2 x \dd x &= \int \sin^2 x \cos^2 x \cdot \sin x \dd x  \tag{substitution $u = \cos x$}\\
        &= \int (1 - u^2) u^2 (-\dd u) \\
        &= -\int (u^4 - u^2) \dd u \\
        &= -\left(\dfrac{u^5}{5} - \dfrac{u^3}{3}\right) + C \tag{Constant $C$ indefinite integral} \\
        &= -\left(\dfrac{\cos^5 x}{5} - \dfrac{\cos^3 x}{3}\right) + C.
    \end{align*}
\end{ex}

\subsubsection{Steps to Evaluate Trig Integrals}
\begin{enumerate}
    \item Identify the type of the integrand 
    \begin{itemize}
        \item If $n$ or $m$ is odd, substitution rule is needed. \gray{E.g. $n$ is odd, rewrite
        \[\int \sin^n x \cos^m x \dd x = \int \sin^{n-1} x \cos^m x ~~\red{\sin x\dd x}.\]
    Then take $u = \cos x$ so that $\dd u = -\red{\sin x\dd x}$.}
        \item If both of the powers are even, use trig formulae to reduce the powers of $\sin x$ and $\cos x$'s.
    
    \end{itemize}
    
    
    \item \red{Be careful with the sign!}
\end{enumerate}


\subsubsection{More Examples}
In the next example, you will see that the half-angle formulae are particularly useful when dealing with even powers of sine and cosine.
\begin{ex}
    [Evaluate $\dint \cos^4 \dd x$] \leavevmode
    Use $\cos^2 x = \dfrac{1 + \cos (2x)}{2}$ to lower the order of $\cos x$'s, we get: 
    \begin{align*}
        \int \cos^4 x \dd x &= \int \Big(\dfrac{1 + \cos (2x)}{2}\Big)^2 \dd x \\
        & =  \int \dfrac{1 + 2\cos(2x) + (\cos^2 (2x))^2}{4} \dd x \tag{pull the constant $1/4$ out}\\
        &= \dfrac{1}{4} \int 1 + 2\cos(2x) +  \dfrac{1+\cos(4x)}{2} \dd x \\
        &= \dfrac{3}{8} x + \dfrac{\sin(2x)}{4} + \dfrac{1}{32} \sin(4x) + C.
    \end{align*}
\end{ex}
\chcomment{Week 2}

Similarly we can compute  
\[\int \tan^n x \sec^m x \dd x.\]
Recall that $(\tan x)' = \sec^2 x$ and $(\sec x)' = \sec x \tan x$.

\begin{ex}[Evaluate $\dint \tan x \sec^4 x \dd x$] \leavevmode
    \begin{align*}
        \int \tan x \sec^4 x \dd x &= \int \tan x \sec^2 x \sec^2 x \dd x = \int \tan x (1+\tan^2 x) \cdot \sec^2 x \dd x \tag{substitution $u = \tan x$}\\
        &= \int u(1 + u^2)\dd u\\
        &= \dfrac{u^2}{2} + \dfrac{u^4}{4} + C \tag{Constant $C$ due to indefinite integral} \\
        &= \red{\dfrac{\tan^2 x}{2} + \dfrac{\tan^4 x}{4}} + C.
    \end{align*}
\end{ex}



\begin{ex}[Another way to compute $\dint \tan x \sec^4 x \dd x$] \leavevmode Take $u = \sec x$ then $\dd u = 2 \sec^2 x \tan x \dd x$. We have
    \begin{align*}
        \int \tan x \sec^4 x \dd x &= \int \tan x \sec^2 x \sec^2 x \dd x = \int \dfrac{1}{2} \dd u =  \dfrac{u^2}{4} + C' = \dfrac{\sec^4 x}{4} + C'.
    \end{align*}
\end{ex}

\blue{Note that the two method gives the SAME answer. Here's why
\[\dfrac{\sec^4 x}{4} + C'= \dfrac{(1+\tan x)^2}{4} + C' = \red{\dfrac{\tan^2 x}{2} + \dfrac{\tan^4 x}{4}} + \dfrac{1}{4} + C'.\]
The constant $C$ and $C'$ satisfies the relation: $C = \dfrac{1}{4} + C'$. }


\subsubsection{Beyond Calculus II}
Why do we study $\dint \sin^n x \cos^m x \dd x$?

Integrals of the this form are studied for their broad applications in mathematics, physics, and engineering. These integrals appear in Fourier analysis, wave mechanics, and signal processing, where sine and cosine functions serve as fundamental building blocks. 



\subsection{Trigonometric substitution} \chcomment{\S7.3}
\begin{center}
\begin{tcolorbox}
    \begin{itemize}
        \item \textbf{Particular type of integral}: integral involving square root of quadric polynomials.
        \item \textbf{Tools to use}: Trig substitutions (the idea comes from the trig identities)
    \end{itemize}
    \begin{center}
    \renewcommand{\arraystretch}{2.5}
    \begin{tabular}{|c|c|c|c|c|} 
        \hline
         & $ x $ & Range of $ \theta $ & $ \dd x $ & $ \sqrt{\cdots} $ becomes \\ 
        \hline
        $ \sqrt{a^2-x^2} $ & $ a \sin(\theta) $ & $ -\dfrac{\pi}{2} \leq \theta \leq \dfrac{\pi}{2} $ & $ a \cos(\theta) \dd \theta $ & $ a \cos(\theta) $ \\
        \hline
        $ \sqrt{x^2+a^2} $ & $ a \tan(\theta) $ & $ -\dfrac{\pi}{2} < \theta < \dfrac{\pi}{2} $ & $ a \sec^2(\theta) \dd \theta $ & $ a \sec(\theta) $ \\
        \hline
        $ \sqrt{x^2-a^2} $ & $ a \sec(\theta) $ & $ 0 \leq \theta \leq \dfrac{\pi}{2} \text{ or } \dfrac{\pi}{2}< \theta \leq \pi $ & $ a \sec(\theta) \tan(\theta) \dd \theta $ & $ a \tan(\theta) $ \\
        \hline
    \end{tabular}
    \end{center}
\end{tcolorbox}
\end{center}


\subsubsection{Trig Substitution Rule}
In this section, we consider integrals containing square roots of the form 
\[\sqrt{a^2 - x^2} \qquad \sqrt{x^2 + a^2} \qquad \sqrt{x^2 - a^2}.\]
We use trigonometric substitutions:
\begin{center}
    \renewcommand{\arraystretch}{2.5}
    \begin{tabular}{|c|c|c|c|c|} 
        \hline
         & $ x $ & Range of $ \theta $ & $ \dd x $ & $ \sqrt{\cdots} $ becomes \\ 
        \hline
        $ \sqrt{a^2-x^2} $ & $ a \sin(\theta) $ & $ -\dfrac{\pi}{2} \leq \theta \leq \dfrac{\pi}{2} $ & $ a \cos(\theta) \dd \theta $ & $ a \cos(\theta) $ \\
        \hline
        $ \sqrt{x^2+a^2} $ & $ a \tan(\theta) $ & $ -\dfrac{\pi}{2} < \theta < \dfrac{\pi}{2} $ & $ a \sec^2(\theta) \dd \theta $ & $ a \sec(\theta) $ \\
        \hline
        $ \sqrt{x^2-a^2} $ & $ a \sec(\theta) $ & $ 0 \leq \theta \leq \dfrac{\pi}{2} \text{ or } \dfrac{\pi}{2}< \theta \leq \pi $ & $ a \sec(\theta) \tan(\theta) \dd \theta $ & $ a \tan(\theta) $ \\
        \hline
    \end{tabular}
\end{center}
Note that we use trigonometric identities to simplify the square root expressions.

For example:
\begin{align*}
    x = a \sin \theta     \implies \sqrt{a^2 - x^2} &= \sqrt{a^2 - a^2 \sin^2 \theta}\\
    &= \sqrt{a^2 \cos^2 \theta} = |a \cos \theta|.
\end{align*}
\textbf{Warning: We have to specify the range of $\theta$ so that we can get rid of $|~~|$.}


\subsubsection{Steps to Apply Trig Substitutions}
\begin{enumerate}
    \item \textbf{Identify the integrand type}: there are three types 
    \[\sqrt{a^2 - x^2} \qquad \sqrt{x^2 + a^2} \qquad \sqrt{x^2 - a^2}.\]
    \item \textbf{Choose an appropriate substitution}: Use the table or trigonometric identities to eliminate the square root by substituting $x$ with a trigonometric function.  
    \item \red{Always \textbf{specify the range of $\boldsymbol{\theta}$} to ensure $x$ is the positive root $\boldsymbol{+}\sqrt{\cdots}$.}
    \item \textbf{Back-substitution}: Express the trig functions $\sin \theta, \tan \theta \cdots$ in terms of $x$ (trig identities and Calculus I knowledge are needed, and rewrite $\theta$ using inverse trig functions.
\end{enumerate}


\subsubsection{Examples}
\begin{ex}[Evaluate $\dint \sqrt{9 - x^2} \dd x $] Take $x=3\sin \theta$, with $\blue{-\dfrac{\pi}{2} \leq \theta \leq \dfrac{\pi}{2}}$, then 
    \begin{align*}
    \int \sqrt{9 - x^2} \dd x &= \int \sqrt{9 - (3\sin \theta)^2} \cdot 3 \cos \theta \dd \theta = \int \blue{|3 \cos \theta|} \cdot 3 \cos \theta \dd \theta \tag{\blue{Need $-\dfrac{\pi}{2}\leq \theta \leq \dfrac{\pi}{2}$ so that $\cos \theta \geq 0$}}\\
    &= \int 9 \cos^2 \theta \dd \theta = \int 9 \dfrac{1+\cos \theta}{2} \dd \theta =  \dfrac{9}{2} \theta + \dfrac{9}{4} \sin (2\theta) + C\\
    &= \dfrac{9}{2} \theta + \dfrac{9}{2} \sin \theta \cos \theta + C = \dfrac{9}{2} \arcsin \dfrac{x}{3} + \dfrac{9}{2} \dfrac{x}{3}\sqrt{1-\dfrac{x^2}{3^2}} + C \tag{Back-substitution}\\
    &= \dfrac{9}{2} \arcsin \dfrac{x}{3} + \dfrac{x\sqrt{9-x^2}}{2} + C.
\end{align*}
\blue{For the back-substitution step: note that $\sin \theta = \dfrac{x}{3}, \cos \theta = \sqrt{1-\sin^2 \theta} \text{ (by trig identity) } = \sqrt{1-\dfrac{x^3}{9}} = \dfrac{\sqrt{9-x^2}}{3}$, and $\theta = \arcsin\dfrac{x}{3}$.}
\end{ex}

\begin{ex}[Evaluate $\dint \dfrac{1}{x^2 \sqrt{x^2 + 4}} \dd x $] Take $x=2\tan \theta$, with $\blue{-\dfrac{\pi}{2} < \theta < \dfrac{\pi}{2}}$, then 
    \begin{align*}
    \int \dfrac{1}{x^2 \sqrt{x^2 + 4}} \dd x &\quad = \int \dfrac{1}{(2 \tan \theta)^2 \sqrt{4 \tan^2 \theta + 4}} (2 \sec^2 \theta) \dd \theta \\
    &= \int \dfrac{2 \sec^2 \theta}{4 \tan^2 \theta \cdot \blue{|2 \sec \theta|}} \dd \theta = \int \dfrac{\sec \theta}{4 \tan^2 \theta} \dd \theta \tag{\blue{Need $-\dfrac{\pi}{2}<\theta <\dfrac{\pi}{2}$ so that $\sec \theta > 0$}}\\
    &= \int \dfrac{\cos \theta}{4 \sin^2 \theta} \dd \theta \tag{Using substitution: $u = \sin \theta, \dd u = \cos \theta \dd \theta$} \\
    &= \int \dfrac{1}{4 u^2} \dd u = -\dfrac{1}{4u} + C
    \tag{Back-substitution}\\
    &= -\dfrac{1}{4 \sin \theta} + C = -\dfrac{\sqrt{4+x^2}}{x} + C.
\end{align*}
\blue{For the back-substitution step: Note that $\tan \theta = \dfrac{x}{2}$, so $\theta = \arctan \dfrac{x}{2}$. To rewrite $\sin \theta$, observe that $\tan \theta = \dfrac{Y}{X}$ and $\sin \theta = \dfrac{Y}{R}$. Solving for $R$, we get $R = \sqrt{X^2 + Y^2} = \sqrt{4 + x^2}$, which implies $\sin \theta = \dfrac{x}{\sqrt{4 + x^2}}$.}
\end{ex}

\begin{ex}[Evaluate $\dint \dfrac{x}{\sqrt{3 - 2x - x^2}} \dd x$]
Complete the square: $3 - 2x - x^2 = -(x^2+2x+1-1)+3 = -(x+1)^2+4$. Take $u = x+1$
    \begin{align*}
    \int \dfrac{x}{\sqrt{3 - 2x - x^2}} \dd x &= \int \dfrac{u-1}{\sqrt{4-u^2}} \dd u = \int \dfrac{2\sin \theta - 1}{\sqrt{4 - 4 \sin^2 \theta}}  2\cos \theta \dd \theta \\
    &= \int \dfrac{2\sin \theta - 1}{\blue{|2\cos \theta|}}  2\cos \theta \dd \theta \tag{\blue{Need $-\dfrac{\pi}{2} < \theta < \dfrac{\pi}{2}$. Note that $\cos \theta$ is strictly positive.}} \\
    &= \int 2\sin \theta -1 \dd\theta = -2\cos \theta -\theta + C. \\
    &= -\sqrt{4-u^2} - \arcsin\dfrac{u}{2} + C \tag{Back-substitution}\\
    &= -\sqrt{3-2x-x^2} - \arcsin\dfrac{x+1}{2} + C.
\end{align*}
\end{ex}

\subsubsection{Motivation} Why do we study these types of integrals? 

Because they frequently arise in problems related to \textit{arc length} and \textit{surface area} calculations. These integrals help us model and solve real-world geometric problems, such as determining the length of a curve or the area of a surface of revolution. Their importance will become evident as we explore further in Chapter 8.


\subsection{Integration of Rational Functions} \chcomment{\S7.4}
\begin{center}
\begin{tcolorbox}
    \begin{itemize}
        \item \textbf{Particular type of integral}: integral involving rational functions.
        \item \textbf{Tools to use}: partial fraction decomposition
        \begin{itemize}
        \item A \textit{proper} rational function $R(x) = \dfrac{P(x)}{Q(x)}$ satisfies $\deg P(x) < \deg Q(x)$.
        \item An \textit{improper} rational function satisfies $\deg P(x) \geq \deg Q(x) $.
        \item Improper rational functions are converted into proper ones via polynomial long division:
        \[
        \frac{P(x)}{Q(x)} = F(x) + \frac{\tilde{P}(x)}{Q(x)}.
        \]
    \end{itemize}
    
    \begin{center}
    \renewcommand{\arraystretch}{2.5}
    \begin{tabular}{|c|c|} 
        \hline
        Factor in denominator & Terms in the decomposition of a \textit{proper rational function}\\
        \hline
        $ax+b$ & $\dfrac{A}{ax+b}$\\
        \hline
        $(ax+b)^k$ & $\dfrac{A_1}{ax+b}+\dfrac{A_2}{(ax+b)^2}+\cdots+\dfrac{A_k}{(ax+b)^k}$\\
        \hline
        $ax^2+bx+c$ & $\dfrac{Ax+B}{ax^2+bx+c}$\\
        \hline
        $(ax^2+bx+c)^k$ & $\dfrac{A_1x+B_1}{ax^2+bx+c}+\dfrac{A_2x+B_2}{(ax^2+bx+c)^2}+\cdots+\dfrac{A_kx+B_k}{(ax^2+bx+c)^k}$\\
        \hline
    \end{tabular}
    \end{center}
    \end{itemize}
\end{tcolorbox}
\end{center}



\subsubsection{Rational Functions}
\begin{defn}
    A \bfemph{rational function} is a function of the form $\displaystyle R(x) = \dfrac{P(x)}{Q(x)}$, where $P(x)$ and $Q(x)$ are polynomials. 
    
    If $\deg P < \deg Q$, it is \bfemph{proper}; otherwise, it is \bfemph{improper}.
\end{defn}

\begin{ex}
    \[R_1(x) = \dfrac{1}{x+1}, \qquad R_2(x) = \dfrac{2x+1}{(x+1)^2}, \qquad R_3(x) = \dfrac{x^3-3}{(x-7)(x+5)}. \]
    The first two are proper; whereas the last one is improper.
\end{ex}


In this section we solve the integral of the type $\dint R(x) \dd x$.
The strategy is to rewrite $R(x)$ as a sum of simpler rational functions (using long division and partial fraction decomposition). Then use the substitution rule to solve the integral.

\newpage
\subsubsection{Partial Fraction Decomposition}
We start with proper rational functions. The following table lists the terms that appear in the decomposition of a proper rational function. \chcomment{Note that "proper" is necessary, otherwise, there will be a polynomial term appear in the decomposition. See Example \ref{ex: an improper rational funciton}.}
\begin{center}
    \renewcommand{\arraystretch}{2.5}
    \begin{tabular}{|c|c|} 
        \hline
        Factor in denominator & Terms in the decomposition of a \textit{proper rational function} \\
        \hline
        $ax+b$ & $\dfrac{A}{ax+b}$\\
        \hline
        $(ax+b)^k$ & $\dfrac{A_1}{ax+b}+\dfrac{A_2}{(ax+b)^2}+\cdots+\dfrac{A_k}{(ax+b)^k}$\\
        \hline
        $ax^2+bx+c$ & $\dfrac{Ax+B}{ax^2+bx+c}$\\
        \hline
        $(ax^2+bx+c)^k$ & $\dfrac{A_1x+B_1}{ax^2+bx+c}+\dfrac{A_2x+B_2}{(ax^2+bx+c)^2}+\cdots+\dfrac{A_kx+B_k}{(ax^2+bx+c)^k}$\\
        \hline
    \end{tabular}
    \end{center}
   Improper rational functions are converted into proper ones via polynomial long division:
        \[ R(x) = \dfrac{P(x)}{Q(x)} = F(x) + \frac{\tilde{P}(x)}{Q(x)},
        \]
        where $F(x)$ is a polynomial. Then we can apply partial fraction decomposition to $\frac{\tilde{P}(x)}{Q(x)}$.
\begin{ex}
    $R_2$ is proper, so we set \[R_2(x) = \dfrac{2x+1}{(x+1)^2} = \frac{A}{x+1} + \dfrac{B}{(x+1)^2}.\]
    Compare: $R_3$ is improper
    \[R_3(x) = \dfrac{x^3-3}{(x-7)(x+5)} = \dfrac{x(x-7)(x+5) + 2x^2 +35x - 3}{(x-7)(x+5)} = x + \dfrac{2x^2 +35x - 3}{(x-7)(x+5)}. \]
    Then decompose to the second term $\dfrac{2x^2 +35x - 3}{(x-7)(x+5)}$ using the table, we set 
    \[\dfrac{2x^2 +35x - 3}{(x-7)(x+5)} = \frac{A}{x-7} + \frac{B}{x+5}.\]
\end{ex}
\subsubsection{Examples of integrals of rational functions}

\begin{ex}[Evaluate $\dint \dfrac{x}{x+4} \dd x$] \label{ex: an improper rational funciton}
    The integrand is improper, so we first apply long division:
    \[\dfrac{x}{x+4} = \dfrac{x+4-4}{x+4} = 1-\dfrac{4}{x+4}.\]
    So 
    \begin{align*}
    \int \dfrac{x}{x+4} \dd x &= \int 1-\dfrac{4}{x+4} \dd x = (x+4) - 4 \ln|x+4| + C.
\end{align*}
\end{ex}


\begin{tcolorbox}
    If $Q$ is a product of distinct linear factors, 
    \[Q = (a_1x+b_1)(a_2x+b_2) \cdots (a_nx+b_n),\]
    we take 
    \[R = \dfrac{A_1}{a_1x+b_1} + \dfrac{A_2}{a_2x+b_2} + \cdots + \dfrac{A_n}{a_nx+b_n}.\]
\end{tcolorbox}
\begin{ex}[Evaluate $\dint \dfrac{1}{x^2-4} \dd x$]
    The integrand is proper and the denominator factors as $x^2 - 4 = (x-2)(x+2)$. Using partial fraction decomposition:
    \[\dfrac{1}{x^2-4} = \dfrac{A}{x-2} + \dfrac{B}{x+2}, \quad \text{where } A, B \text{ are constants.} \]
    The numerator gives 
    \[(A+B)x + 2(A-B) = 1 \implies A = -B = \dfrac{1}{4}.\]
    Plug this back into the integral, we have
    \begin{align*}
        \int \dfrac{1}{x^2-4} \dd x &= \int \left(\dfrac{1}{4(x-2)} - \dfrac{1}{4(x+2)}\right) \dd x \\
        &= \dfrac{1}{4} \ln|x-2| - \dfrac{1}{4} \ln|x+2| + C \\
        &= \dfrac{1}{4} \ln\left|\dfrac{x-2}{x+2}\right| + C.
    \end{align*}
\end{ex}

\begin{tcolorbox}
    If $Q$ contains distinct irreducible quadratic factors, take the corresponding quadratic form 
    \[R = (\text{fraction with linear terms}) + \cdots + \dfrac{Ax+B}{ax^2+bx+c}.\]
\end{tcolorbox}
\begin{ex}[Evaluate $\dint \dfrac{5x^2+2}{x(x^2+2x+2)} \dd x$]
    The integrand is proper. To decomposition the fraction, we set
    \[\dfrac{5x^2+2}{x(x^2+2x+2)} = \dfrac{A}{x} + \dfrac{Bx + C}{x^2 + 2x + 2}, \quad \text{where } A, B, C \text{ are constants.} \]
    The numerator gives 
    \[Ax^2 +2ax+2a+Bx^2+Cx = 5x^2 + 1 \implies \begin{cases}
        A+B = 5\\
        2A+C = 0\\
        2A = 2
    \end{cases} \implies \begin{cases}
        A=1\\
        B=4\\
        C=-2
    \end{cases}.\]
    Plug this back into the integral, we have
    \begin{align*}
        \int \dfrac{5x^2+2}{x(x^2+2x+2)} \dd x &= \int \dfrac{1}{x} \dd x + \int \dfrac{4x - 2}{x^2 + 2x + 2} \dd x \\
        &= \ln|x| + \int \dfrac{4x-2}{x^2+2x+2} \dd x = \ln|x| + \int \dfrac{4x-2}{(x+1)^2+1} \dd x \tag{Substitution: $u=x+1$}\\
        &= \ln|x| + \int \dfrac{4u-6}{u^2+1} \dd u = \ln|x| + \int \dfrac{4u}{u^2+1} \dd u - \int \dfrac{6}{u^2+1} \dd u + C\\
        &= \ln|x| + 2\int \dfrac{1}{w+1} \dd w - 6\arctan u + C \\
        &=\ln|x| + 2 \ln|w| - 6\arctan u + C \\
        &=\ln|x| + 2 \ln|u^2+1| - 6\arctan (x+1) + C \\
        &=\ln|x| + 2 \ln|x^2+2x+2| - 6\arctan (x+1) + C.
    \end{align*}
\end{ex} \chcomment{Typo22}


\begin{tcolorbox}
    If $Q$ contains a repeated linear factor, say $(ax+b)^r$, include terms of the form:
    \begin{align*}
        \dfrac{A_1}{ax+b} + \dfrac{A_2}{(ax+b)^2} + \cdots + \dfrac{A_r}{(ax+b)^r}.
    \end{align*}
    
    Similarly, for a repeated quadratic factor, say $(ax^2+bx+c)^r$, include terms of the form:
    \begin{align*}
        \dfrac{A_1x+B_1}{ax^2+bx+c} + \dfrac{A_2x+B_2}{(ax^2+bx+c)^2} + \cdots + \dfrac{A_rx+B_r}{(ax^2+bx+c)^r}.
    \end{align*}
\end{tcolorbox}


\begin{ex}[Evaluate $\dint \dfrac{4x}{x^3 - x^2 - x + 1} \dd x$]
   The integrand is proper and the denominator factors as :
    \begin{align*}
        x^3 - x^2 - x + 1 &= (x-1)^2(x+1).
    \end{align*}
    Using partial fraction decomposition, we set
    \[\dfrac{4x}{x^3 - x^2 - x + 1} = \dfrac{4x}{(x-1)^2(x+1)} = \dfrac{A}{x-1} + \dfrac{B}{(x-1)^2}+ \dfrac{C}{x+1}, \quad \text{where } A, B, C \text{ are constants.} \]
    The numerator gives 
    \[A(x+1)(x-1)+B(x-1)+C(x-1)^2 = (A+C)x^2+(B-2C)x+(-A+B+C) = 4x,\]
    so \[ A=1, B=2, C=-1.\]
    Thus, the integral becomes:
    \begin{align*}
        \int \dfrac{4x}{x^3 - x^2 - x + 1} \dd x &= \int \dfrac{1}{x-1} + \dfrac{2}{(x-1)^2} - \dfrac{1}{x+1} \dd x\\
        &= \ln\Big|\dfrac{x-1}{x+1}\Big| - \dfrac{2}{x-1} + C.
    \end{align*}
\end{ex}

\subsubsection{Steps to Evaluate Integrals of Rational Function} 
\begin{enumerate} 
    \item Apply \textbf{long division} to improper rational functions.
    \item \textbf{Factorize} the denominator of the proper rational functions.
    \item Set up the terms that appear (see the table above) in the \textbf{partial fraction decomposition} and solve for the constants.
    \item Apply the \textbf{integration} techniques learned in the preceding sections.
\end{enumerate}



\subsection{Approximate Integration} \chcomment{\S7.7}
\begin{center}
\begin{tcolorbox}
    \begin{itemize}
        \item \textbf{Approximating integral}

        \item \textbf{Tools to use}: Midpoint Rule, Trapezoidal Rule, and Simpson's Rule.
        \item No need to memorize the statements. Know how to use them.
    \end{itemize}
\end{tcolorbox}
\end{center}

\subsubsection{Motivation}
In general, it is difficult to compute the antiderivative of a function and apply the Fundamental Theorem of Calculus, even with techniques we have learned so far. Therefore, we seek an \textit{approximate} value of the integral.

Recall from Calculus I, the integral is defined as the limit of Riemann sums:
\begin{defn}
\begin{align*}
    \int_a^b f(x) \dd x &= \lim_{n \to \infty} \sum_{i=1}^n f(\xi_i) \Delta x.
\end{align*}
\end{defn}

Since we are interested in an approximate value of the integral, instead of taking $n \to \infty$, we sum over a finite number of intervals:
\begin{align*}
    \int_a^b f(x) \dd x \approx \sum_{i=1}^n f(\xi_i) \Delta x.
\end{align*}

For the finite sum above:
\begin{itemize}
    \item If $\xi_i = a + \Delta x \cdot (i-1)$, it is a \textit{left} endpoint approximation.
    \item If $\xi_i = a + \Delta x \cdot i$, it is a \textit{right} endpoint approximation.
    \item If $\xi_i = = a + \Delta x \cdot \dfrac{2i-1}{2}$ is the \textit{midpoint}, it is a midpoint approximation.
\end{itemize}

\begin{figure}[H]
    \centering
    \resizebox{0.5\textwidth}{!}{\begin{tikzpicture}
\def\a{1.7}
\def\b{5.7}
\def\c{3.7}
\def\L{0.5} % width of interval

\pgfmathsetmacro{\Va}{2*sin(\a r+1)+4} \pgfmathresult
\pgfmathsetmacro{\Vb}{2*sin(\b r+1)+4} \pgfmathresult
\pgfmathsetmacro{\Vc}{2*sin(\c r+1)+4} \pgfmathresult

\draw[->,thick] (-0.5,0) -- (7,0) coordinate (x axis) node[below] {$x$};
\draw[->,thick] (0,-0.5) -- (0,7) coordinate (y axis) node[left] {$y$};
\foreach \f in {1.7,2.2,...,6.2} {\pgfmathparse{2*sin(\f r+1)+4} \pgfmathresult
\draw[fill=shinbashi!20] (\f-\L/2,\pgfmathresult |- x axis) -- (\f-\L/2,\pgfmathresult) -- (\f+\L/2,\pgfmathresult) -- (\f+\L/2,\pgfmathresult |- x axis) -- cycle;}
\node at (\a-\L/2,-5pt) {\footnotesize{$a=x_0$}};
\node at (\b+\L/2+\L,-5pt) {\footnotesize{$b=x_n$}};
\draw[black, thick] (\c-\L/2,0) -- (\c-\L/2,\Vc) -- (\c+\L/2,\Vc) -- (\c+\L/2,0);
\draw[dashed] (\c,0) node[below] {\footnotesize{$\xi_i$}} -- (\c,\Vc) -- (0,\Vc) node[left] {$f(\xi_i)$};
\node at (\a+5*\L/2,-5pt) {\footnotesize{$x_{i-1}$}};
\node at (\a+7*\L/2,-5pt) {\footnotesize{$x_i$}};
\node at (\a+5*\L,-5pt) {\footnotesize{$x_{i+1}$}};
\draw[black,thick,smooth,samples=100,domain=1.45:6.2] plot(\x,{2*sin(\x r+1)+4});
\filldraw[black] (\c,\Vc) circle (.03cm);
\end{tikzpicture}} % Include your TikZ file
    \caption{Riemann sum}
    \label{fig:Riemann sum}
\end{figure}


\subsubsection{The Midpoint, Trapezoidal and Simpson's Rules}
We usually use the midpoint approximation. The formula is explicitly written as:
\begin{tcolorbox}[colframe=shinbashi!90, colback=shinbashi!5]
\begin{thm}[Midpoint rule]
    \begin{align*}
        \int_a^b f(x) \dd x \approx M_n  &= \left(f(\overline{x_1}) + f(\overline{x_2}) + \cdots + f(\overline{x_n})\right) \Delta x \\
        &= \sum_{i=1}^n f\Big(a + \Delta x \cdot \dfrac{2i-1}{2}\Big) \cdot \Delta x,
    \end{align*}
where $\overline{x_i}$ are the midpoints and $\Delta x$ is the width of each subinterval.
\end{thm}
\end{tcolorbox}


Another way to approximate the integral is the trapezoidal rule:
\begin{tcolorbox}
\begin{thm}[Trapezoidal rule]
    \begin{align*}
        \int_a^b f(x) \dd x \approx T_n &= \dfrac{\Delta x}{2} \left[f(a) + 2\sum_{i=1}^{n-1} f(x_i) + f(b)\right]\\
        &= \dfrac{\Delta x}{2} \left(f(a) + 2f(x_1) + 2f(x_2) + \cdots + 2f(x_{n-1}) + f(b)\right).
    \end{align*}
\end{thm}
\end{tcolorbox}

Note that 
\[T_n = \Bigg( \dfrac{f(x_0)+f(x_1)}{2} + \dfrac{f(x_1)+f(x_2)}{2} + \cdots + \dfrac{f(x_{n-1})+f(x_n)}{2} \Bigg) \Delta x.\]
Each term $\dfrac{f(x_{i-1})+f(x_i)}{2} \Delta x$ is the area of one trapezoid.

Similar to trapezoidal rule, another rule to approxiamte the integral is 
\begin{tcolorbox}[colframe=enji!80, colback=enji!5]
\begin{thm}[Simpson's Rule]
    \begin{align*}
        \int_a^b f(x) \dd x \approx S_n &= \dfrac{\Delta x}{3} \left[f(a) + 4f(x_1) + 2f(x_2) + \cdots + 4f(x_{n-1}) + f(b)\right].
    \end{align*}
\end{thm}
\end{tcolorbox}


\subsubsection{Error of Approximation}
\begin{align*}
    \blue{E_M = \dint_a^b f(x) \dd x - M_n}, \qquad
    \green{E_T = \dint_a^b f(x) \dd x - T_n} \qquad
    \red{E_S = \dint_a^b f(x) \dd x - S_n}.
\end{align*}
\textbf{Error bounds}: For $a \leq x \leq b$, \green{suppose $|f''(x)| \leq K$ for the trapezoidal rule} and \red{suppose $|f^{(4)}(x)| \leq K$ for Simpson's rule}, then:
\begin{align*}
    \blue{|E_M| \leq \dfrac{K (b-a)^3}{24n^2}}, \qquad
    \green{|E_T| \leq \dfrac{K (b-a)^3}{12n^2}} \qquad \red{|E_S| \leq \dfrac{K (b-a)^5}{180n^4}}.
\end{align*}


\subsubsection{Example}
\begin{ex}
    Let $f(x) = x^2$ on the interval $[1,4]$. Determine the number of subintervals $n$ required such that the error $E_M$ in the Midpoint Rule approximation satisfies 
    \[|E_M| < 0.1.\] 
    
    \textit{Solution}. The error bound for the Midpoint Rule is given by:
    \[|E_M| \leq \dfrac{K(b-a)^3}{24n^2} \]
    where $a=1, b=4$ and $K = \max_{c \in [1, 4]} |f''(c)| = 2$.
    
    Substitute into the error formula we have:
    \[|E_M| \leq \dfrac{2 \cdot (4-1)^3}{24n^2} = \dfrac{9}{4n^2} < 0.1 \implies n \geq \sqrt{\dfrac{9}{0.4}} \approx 4.74.\]
    Since $n$ must be an integer to ensure $|E_M| < 0.1$, the smallest number $n$ is 5.
\end{ex}

\newpage

\subsection{Improper Integrals} 
\chcomment{\S7.8}
\chcomment{Week 4}

\begin{tcolorbox}
    \begin{itemize}
        \item \textbf{Improper integrals}: deal with unbounded intervals or functions.
        \item \textbf{Tools to use}: taking limit of a proper integral. E.g.
        \begin{itemize}
            \item Type I: $\dint_a^\infty f(x) \dd x = \lim_{t \to \infty} \int_a^t f(x) \dd x$
            \item Type II: $\dint_a^c f(x) \dd x = \lim_{t \to c} \int_a^t f(x) \dd x$
        \end{itemize}
        \item \textbf{Comparison test}: \begin{enumerate*}[label = \circled{\arabic*}]
            \item $f,g$ continuous, \item $0 \leq f(x) \leq g(x)$
            \item for $x \geq a$.
        \end{enumerate*} 
        Then 
        \begin{align*}
        \int_a^\infty f(x) \dd x \text{ converges } \quad \boldsymbol{\implies} \quad \int_a^\infty g(x) \dd x \text{ converges } \\
        \int_a^\infty f(x) \dd x \text{ diverges } \quad \boldsymbol{\impliedby} \quad \int_a^\infty g(x) \dd x \text{ diverges } 
    \end{align*}
    \item \textbf{To memorize}: \begin{align*}
        \int_a^\infty \dfrac{1}{x^p} \dd x \qquad \begin{cases}
            \text{ converges} & \text p > 1\\
            \text{ diverges} & \text p \leq 1
        \end{cases}
    \end{align*}
    \end{itemize}
\end{tcolorbox}


In Chapter 5 (Calculus I), we studied definite integrals of the form $\dint_a^b f(x) \dd x$, where:

\begin{itemize}
    \item $f(x)$ is piecewise continuous, and
    \item $a,b$ are real numbers.
\end{itemize}

Such integrals are known as \textbf{proper integrals} (note that this has nothing to do with proper fractional functions $R(x) = \dfrac{P(x)}{Q(x)}$).


\subsubsection{Definition of Improper Integrals}

In this section, we extend our discussion to \textbf{improper integrals}, which arises in two main cases: when the limits of integration are infinite or when the function being integrated has discontinuities.
We define two types of improper integrals.

\begin{defn}[Type I improper integral]
    \begin{align*}
        \int_a^\infty f(x) \dd x &:= \lim_{t \to \infty} \int_a^t f(x) \dd x, \\
        \int_{-\infty}^b f(x) \dd x &:= \lim_{t \to -\infty} \int_t^b f(x) \dd x,\\
        \int_{-\infty}^\infty f(x) \dd x &:= \int_{-\infty}^a f(x) \dd x + \int_a^\infty f(x) \dd x = \lim_{t \to -\infty} \int_t^a f(x) \dd x + \lim_{t \to \infty} \int_a^t f(x) \dd x.
    \end{align*}
\end{defn}

\begin{defn}
    An improper integral is \bfemph{convergent} if the above limit exists; otherwise, it is \bfemph{divergent}.
\end{defn}

\begin{defn}[Type II improper integral]
If $f(x)$ has a discontinuity at some point $c \in [a,b]$, we define
    \begin{align*}
        \int_a^c f(x) \dd x &:= \lim_{t \to c} \int_a^t f(x) \dd x, \\
        \int_{c}^b f(x) \dd x &:= \lim_{t \to c} \int_t^b f(x) \dd x,\\
        \int_a^b f(x) \dd x &:= \int_a^c f(x) \dd x + \int_c^b f(x) \dd x = \lim_{t \to c} \int_a^t f(x) \dd x + \lim_{t \to c} \int_t^b f(x) \dd x.
    \end{align*}
\end{defn}

\begin{figure}[H]
    \centering
    \resizebox{0.4\textwidth}{!}{\begin{tikzpicture}
    % Axes
    \draw[->] (-1,0) -- (5,0) node[right] {\(x\)};
    \draw[->] (0,-1) -- (0,4) node[above] {\(y\)};
    
    % Vertical asymptote at x=3
    \draw[dashed, shinbashi, thick] (1.6,0.6) -- (1.6,-0.1) node[below] {$a$};
    \draw[dashed, shinbashi, thick] (4.7,0.5) -- (4.7,-0.1) node[below] {$b$};
    
    \draw[dashed, enji, thick] (3,4) -- (3,-0.1) node[below] {$c$};
    
    % Function graph
    \draw[domain=0.2:2.75,smooth,variable=\x,thick] 
        plot ({\x},{-1/(\x-3)});
    \draw[domain=3.25:5,smooth,variable=\x,thick] 
        plot ({\x},{1/(\x-3)});
    
    % Labels
    \node[below left] at (0,0) {\(0\)};
\end{tikzpicture}
} % Include your TikZ file
    \caption{Type II indefinite integral}
    \label{fig:type ii indefinite integral}
\end{figure}

\begin{tcolorbox}
    It is important to review the techniques for taking limits from Calculus I. For reference, see Chapter 2: Limits and Derivatives of the textbook.
\end{tcolorbox}



\subsubsection{Steps to Evaluate Improper Integrals}
\begin{enumerate}
    \item \textbf{Identify all points} where the integral is improper, including points at infinity and discontinuities.
    \item \textbf{Decompose the integral} into subintervals such that each integral is proper.
    \item \textbf{Express the improper integral} as a limit of proper integrals.
    \item \textbf{Evaluate} the proper integrals and take the limit.
\end{enumerate}


\subsubsection{Examples}
\begin{ex}[Evaluate $\dint_0^\infty e^{-x} \dd x$]
    \[\int_0^\infty e^{-x} \dd x = \lim_{t \to \infty} \int_0^t e^{-x} \dd x = \lim_{t \to \infty} \left[-e^{-x}\right]_0^t = \lim_{t \to \infty} \left(-e^{-t} + e^0\right) = 1.\]
\end{ex}

\begin{ex}[Evaluate $\dint_1^\infty \dfrac{1}{x^p} \dd x, p < 1$]
    \[\int_1^\infty \dfrac{1}{x^p} \dd x = \lim_{t \to \infty} \int_1^t \dfrac{1}{x^p} \dd x = \lim_{t \to \infty} \left[\dfrac{x^{-p+1}}{-p+1}\right]_1^t = \lim_{t \to \infty} \dfrac{1}{p-1}\left(\dfrac{1}{t^{p-1}} - 1\right) = \dfrac{1}{p-1}.\]
Note that this integral diverges if $p > 1$. 
\end{ex}

\begin{ex}[Evaluate $\dint_0^1 \dfrac{1}{x-1} \dd x$]
    \begin{align*}
    \int_0^1 \dfrac{1}{x-1} \dd x &= \lim_{t \to 1^-} \int_0^t \dfrac{1}{x-1} \dd x = \lim_{t \to 1^-} \big[\ln|x-1|\big]_0^t = \lim_{t \to 1^-} \ln|t-1| = -\infty.
\end{align*}
\end{ex}

\subsubsection{Comparison Test for Improper Integrals}
\chcomment{Week 5}

Comparison tests can establish the convergence or divergence of improper integrals.
Suppose \begin{enumerate*}[label = \circled{\arabic*}]
    \item $f(x)$ and $g(x)$ are \textbf{continuous} functions and \item $0 \leq f(x) \leq g(x)$ 
    \item for $x \geq a$.
\end{enumerate*}
Then
\begin{align*}
    \int_a^\infty g(x) \dd x \text{ converges } \quad \boldsymbol{\implies} \quad \int_a^\infty f(x) \dd x \text{ converges } \\
    \int_a^\infty g(x) \dd x \text{ diverges } \quad \boldsymbol{\impliedby} \quad \int_a^\infty f(x) \dd x \text{ diverges } 
\end{align*}

\begin{ex}[Example to remember] \label{p-test for improper integral}
    \begin{align*}
        \int_a^\infty \dfrac{1}{x^p} \dd x \qquad \begin{cases}
            \text{ converges} & \text p > 1\\
            \text{ diverges} & \text p \leq 1
        \end{cases}
    \end{align*}
\end{ex}



\subsubsection{Steps for Applying the Comparison Test}
\begin{enumerate}
    \item \textbf{Determine the dominant term} of the integrand as $x$ approaches infinity or a discontinuity, typically a power function.  
    \item \textbf{Identify the exponent $p$} in the power function and use Example \ref{p-test for improper integral} to make an initial guess.  
    \item \textbf{Find suitable $\circled{1}$ continuous functions} $f(x)$ and $g(x)$ for the comparison test.  
    \item \textbf{Justify the inequality} $\circled{2}~0 \leq f(x) \leq g(x)$ for $\circled{3}~x \geq a$. 
    \item \textbf{Apply the comparison test} to confirm the guess.  
\end{enumerate}



\subsubsection{Examples}
\begin{ex}[Show that $I = \dint_1^\infty \dfrac{1+e^{-x}}{x} \dd x$ diverges] 

    \textit{Step 1}. Note that the integrand is dominated by $ \dfrac{1}{x} $ as $ x \to \infty $. 
    
    \textit{Step 2}. This corresponds to the case $ p = 1 $, and we aim to justify divergence.
    
    \textit{Step 3}. We set $ f(x) = \dfrac{1 + e^{-x}}{x} $ and $ g(x) = \dfrac{1}{x} $. 
    \textit{Step 4}. Note that $ 0 \leq f(x) \leq g(x) $ for all $ x \geq 1 $. 
    \textit{Step 5}. Moreover, $p = 1$ so the integral 
    \begin{align*}
    \int_1^\infty g(x) \dd x &= \int_1^\infty \dfrac{1}{x} \dd x = \lim_{t \to \infty} \int_1^t \dfrac{1}{x} \dd x \\
    &= \lim_{t \to \infty} \left[\ln x\right]_1^t = \lim_{t \to \infty} (\ln t - \ln 1) = \infty \tag{The same computation as Example 6.5}.
    \end{align*}
    By the comparison test, $\displaystyle I = \int_1^\infty \dfrac{1 + e^{-x}}{x} \dd x$ also diverges.

\end{ex}


\begin{ex}[Apply the comparison test to $I = \dint_1^\infty \dfrac{1}{\sqrt{x^6 + 1}} \dd x$]

    \textit{Step 1}. Note that the integrand is dominated by $\dfrac{1}{\sqrt{x^6}} = \dfrac{1}{x^3}$ as $x \to \infty$. 
    
    \textit{Step 2}. This corresponds to the case $ p = 3 $, and we aim to justify convergence.
    
    \textit{Step 3}. We set $g(x) = \dfrac{1}{\sqrt{x^6 + 1}}$ and compare it to $f(x) = \dfrac{1}{\sqrt{x^6}}$.

    \textit{Step 4}. Note that for $x \geq 1$,
    \[
    0 \leq \sqrt{x^6} \leq \sqrt{x^6 + 1} \quad \implies \quad  0 \leq \dfrac{1}{\sqrt{x^6 + 1}} \leq \dfrac{1}{\sqrt{x^6}}.\]
    \textit{Step 5}. Moreover, $p = 3 > 1$ so the integral $\dint_1^\infty \dfrac{1}{\sqrt{x^6}} \dd x = \dint_1^\infty x^{-3} \dd x$ converges. 

    By the comparison test, the integral $I$ also converges. 
\end{ex}

\begin{ex}[Apply the comparison test to $I = \dint_2^\infty \dfrac{\cos^2 x}{x^2}\dd x$] \chcomment{Read the remaining examples we haven't discussed in class.}

    \textit{Step 1}. Note that the integrand is dominated by $\dfrac{1}{x^2}$ as $x \to \infty$. 
    
    \textit{Step 2}. This corresponds to the case $ p = 2 $, and we aim to justify convergence.
    
    \textit{Step 3}. We set $g(x) = \dfrac{\cos^2 x}{x^2}$ and compare it to $f(x) = \dfrac{1}{x^2}$.

    \textit{Step 4}. Note that for $0 \leq \cos^2 x \leq 1$ for all $x$, so 
    \[0 \leq \dfrac{\cos^2 x}{x^2} \leq \dfrac{1}{x^2}.\]
    \textit{Step 5}. Moreover, $p = 2 > 1$ so the integral $\dint_2^\infty \dfrac{1}{\sqrt{x^2}}$ converges. By the comparison test, the integral $I$ also converges. 
\end{ex}


\begin{ex}[Apply the comparison test to $I = \dint_3^\infty \dfrac{1}{x - e^{-x}}\dd x$]
    
    \textit{Step 1}. Note that the integrand is dominated by $\dfrac{1}{x}$ as $x \to \infty$. 
    
    \textit{Step 2}. This corresponds to the case $ p = 1 $, and we aim to justify divergence.
    
    \textit{Step 3}. We set $f(x) = \dfrac{1}{x - e^{-x}}$ and compare it to $g(x) = \dfrac{1}{x}$.

    \textit{Step 4}. Note that for $0 < e^{-x} < x$ with $x>3$, so 
    \[0 < x- e^{-x} \leq x < \infty \quad \implies \quad 0 < \dfrac{1}{x} < \dfrac{1}{x - e^{-x}}.\]
    \textit{Step 5}. Moreover, $p = 1$ so the integral $\dint_3^\infty \dfrac{1}{x}$ diverges. By the comparison test, the integral $I$ also diverges. 
\end{ex}
\section{Chapter 8}
In this chapter, we will explore the applications of the techniques we have learned so far. We will apply integration methods to problems involving arc length, surface area, and other geometric quantities. 


\subsection{Arc Length}  \chcomment{\S8.1}

\begin{center}
\begin{tcolorbox}
    \begin{itemize}
        \item \textbf{Infinitesimal line element}:
        \begin{align*}
            \dd s &= \sqrt{(\dd x)^2 +  (\dd y)^2} = \sqrt{1 + \left(f'(x)\right)^2} \dd x = \sqrt{1 + \left(g'(y)\right)^2} \dd y.
        \end{align*}
        \item \textbf{Arc length}:
            \[L = \int_A^B \dd s = \int_a^b \sqrt{1 + \left(f'(x)\right)^2} \dd x = \int_c^d \sqrt{1 + \left(g'(y)\right)^2} \dd y.\]
    \end{itemize}
\end{tcolorbox}
\end{center}
Let’s first consider how to compute arc length
\subsubsection{Derivation of the Arc Length Formula}
\begin{align*}
    L = \lim_{n\to \infty} \sum_{i=1}^n |P_{i-1}P_i|, \text{ where } |P_{i-1}P_i| &= \sqrt{(\Delta x_i)^2 +  (\Delta y_i)^2} = \sqrt{1+\left(\dfrac{\Delta y_i}{\Delta x_i}\right)^2} \Delta x_i.
\end{align*}

\begin{figure}[ht]
    \centering
    \resizebox{0.5\textwidth}{!}{\begin{tikzpicture}
    % Define the curve f(x)
    \draw[domain=0:6,samples=100] plot (\x,{sin(\x r)}) node[right] {$y = f(x)$};

    % Draw the points on the curve
    \fill (1, {sin(1 r)}) circle (2pt) node[above] {$P_0$};
    \fill (1.5, {sin(1.5 r)}) circle (2pt) node[above] {$P_1$};
    \fill (2, {sin(2 r)}) circle (2pt) node[above] {$P_2$};
    
    \fill (3.5, {sin(3.5 r)}) circle (2pt) node[below left] {$P_{n-1}$};
    \fill (4, {sin(4 r)}) circle (2pt) node[below] {$P_{n}$};

    % Draw the segments representing sublengths
    \draw[thick, enji] (1, {sin(1 r)}) -- (1.5, {sin(1.5 r)});
    \draw[thick, enji] (1.5, {sin(1.5 r)}) -- (2, {sin(2 r)});
    \draw[thick, enji] (3.5, {sin(3.5 r)}) -- (4, {sin(4 r)});

    % Draw the dashed lines for the height differences
    \draw[dashed] (1, 0) -- (1, {sin(1 r)});
    \draw[dashed] (1.5, 0) -- (1.5, {sin(1.5 r)});
    \draw[dashed] (2, 0) -- (2, {sin(2 r)});
    \draw[dashed] (2.5, 0) -- (2.5, {sin(2.5 r)});
    \draw[dashed] (3.5, 0) -- (3.5, {sin(3.5 r)});
    \draw[dashed] (4, 0) -- (4, {sin(4 r)});
    % Label the segments
    %\node at (2, -0.5) {\huge Sublengths approximation};

    % Draw the x-axis and y-axis
    \draw[->] (-0.5,0) -- (6.5,0) node[right] {$x$};
    \draw[->] (0,-1.5) -- (0,1.5) node[above] {$y$};

    % Label the arc length integral
    %\node at (2, -2) {Arc length approximation: $L \approx \sum_{i=1}^{n} \sqrt{(x_{i+1} - x_i)^2 + (f(x_{i+1}) - f(x_i))^2}$};

\end{tikzpicture}

} % Include your TikZ file
    \caption{Arc length}
    \label{fig:arc length}
\end{figure}
Suppose the curve is given by the graph of some differentiable function $y = f(x)$. Then, when taking the limit $\Delta x \to 0$, the expression $\dfrac{\Delta y}{\Delta x} \to f'$. This suggests 
\begin{align*}
    L = \lim_{n\to \infty} \sum_{i=1}^n \sqrt{1+\left(\dfrac{\Delta y}{\Delta x}\right)^2} \Delta x = \int_a^b \sqrt{1+\left(f'(x)\right)^2} \dd x .
\end{align*}

\begin{defn}
    We define the \bfemph{infinitesimal line element} 
    \[
    \dd s = \sqrt{(\dd x)^2 + (\dd y)^2} = \sqrt{1 + \left(f'(x)\right)^2} \, \dd x = \sqrt{1 + \left(g'(y)\right)^2} \, \dd y.
    \]
    Then the \bfemph{arc length} $L$ of a curve is given by 
    \[L = \int_A^B \dd s. \]
    In particular, if the curve is given by $y = f(x)$ for $x \in [a, b]$, where $f$ is continuous and differentiable, then 
    \begin{equation*}
        L = \int_a^b \sqrt{1 + \left(f'(x)\right)^2} \, dx.
    \end{equation*}
    Similarly, if the curve is given by $x = g(y)$ for $y \in [c, d]$, where $g$ is continuous and differentiable, then:
    \begin{equation*}
        L = \int_c^d \sqrt{1 + \left(g'(y)\right)^2} \, dy.
    \end{equation*}

\end{defn}

\subsubsection{Examples}
\begin{ex} Let $y = e^x$ for $x \in [0, 2]$:
    \begin{align*}
        L &= \int_0^2 \sqrt{1 + \left(e^x\right)^2} \dd x.
    \end{align*}
    Alternatively, using $x = \ln y$, we rewrite the integral:
    \begin{equation*}
        L = \int_1^{e^2} \sqrt{1 + \dfrac{1}{y^2}} \dd y.
    \end{equation*}
\end{ex}

\begin{ex} Let $y^2 + x^2 = 1$ (Unit Circle).

    We first compute the arc length of the upper half circle and then use symmetry to get the arc length of the full circle. The upper half of the circle ($y \geq 0$) is given by
    \[y = \sqrt{1-x^2}, \quad -1 \leq x \leq 1. \] 
    Therefore, 
    \begin{equation*}
    L = \int_{-1}^1 \sqrt{1 + \left(-\dfrac{x}{\sqrt{1 - x^2}}\right)^2} \dd x = \int_{-1}^1 \dfrac{1}{\sqrt{1 - x^2}} \dd x = \cdots = \pi.
    \end{equation*}
    The arc length of the full circle is given by $2L = 2\pi$.
\end{ex}



\subsubsection{Arc Length Function}
Given a curve $y = f(x)$, the arc length function $s(x)$ from $x = a$ to $x = b$ is:
\begin{equation*}
    s(x) = \int_a^x \sqrt{1 + \left(f'(t)\right)^2} \dd t.
\end{equation*}


\begin{ex} Let $f(x) = x^2 - \dfrac{\ln x}{8}$ for $x \in [1,\infty)$. Then
    \begin{align*}
    s(x) &= \int_1^x \sqrt{1 + \left(2t -\dfrac{1}{8t}\right)^2} \dd x = \int_1^x \sqrt{1 + 4t^2 - \dfrac{1}{2} + \dfrac{1}{64t^2} } \dd t\\
    &= \int_1^x 2t + \dfrac{1}{8t} \dd t = t^2 + \dfrac{\ln t}{8} \Big|_1^x = x^2 + \dfrac{\ln x}{8} - 1.
    \end{align*}
\end{ex}
\chcomment{Typo22}

\subsubsection{Steps to Compute Arc Length}
\begin{enumerate}
    \item Check if the function is \textbf{differentiable} and determine which variable to use.
    \item Write down the corresponding \textbf{infinitesimal line element} $\dd s$.
    \item Set up the arc length integral. Be careful with the limits of integration.
\end{enumerate}


\subsection{Area of a Surface of Revolution} \chcomment{8.2}
\subsubsection{Derivation of the Surface Area of Revolution formula}
\begin{center}
\begin{tcolorbox}
    \begin{itemize}
        \item \textbf{Infinitesimal area element}:
        \begin{align*}
            \dd A &= 2\pi R \dd s. 
        \end{align*}
        \item \textbf{Surface area}:
            \[A = \int \dd A.\]
    \end{itemize}
\end{tcolorbox}
\end{center}
A surface of revolution is formed by rotating a curve about a line (e.g. the $x$- or $y$-axis). 

To derive the area, recall the surface area of a cylinder is $2\pi R l$.

If we take infinitesimal line segments $\dd s$, the small piece is approximately a cylinder.
\[A = \lim_{n\to \infty} \sum_{i=1}^n 2 \pi f(x) \dd s= \int_a^b 2 \pi f(x) \dd s.\]
This expression needs to be rewritten in terms of $x$ to make it computable.

Recall from last section $s(x) = \dint_a^x \sqrt{1 + \left(f'(t)\right)^2} \dd t$. This tells us
\[\dd s = \sqrt{1 + \left(f'(t)\right)^2} \dd x.\]
We can rewrite the surface areas as follows.

\begin{tcolorbox}
    For a curve $y = f(x)$ rotated about the $x$-axis, the surface area $A = \int \dd A$ is:
    \begin{align*}
        A &= \int_{x_1}^{x_2}  2\pi f(x) \sqrt{1 + \left(f'(x)\right)^2} \dd x\\
        &= \int_{y_1}^{y_2} 2\pi y \sqrt{1+\Big(\frac{\dd x}{\dd y}\Big)^2} \dd y. \tag{$\dfrac{\dd x}{\dd y}$ is given by implicit differentiation}
    \end{align*}
    If the curve $x = g(y)$ is rotated about the $y$-axis, then:
    \begin{align*}
        A &= \int_{y_1}^{y_2} 2\pi g(y) \sqrt{1 + \left(g'(y)\right)^2} \dd y\\
        &= \int_{x_1}^{x_2} 2\pi x \sqrt{1+\Big(\frac{\dd y}{\dd x}\Big)^2} \dd x. \tag{$\dfrac{\dd y}{\dd x}$ is given by implicit differentiation}
    \end{align*}
\end{tcolorbox}


\subsubsection{Examples}
\begin{ex} Let $y = \sqrt{9 - x^2}$ for $x \in [-2, 2]$.
    Rotating about the $x$-axis, compute the surface area:
    \begin{align*}
        \dd s &= \sqrt{1 + \left(y'\right)^2} \dd x = \sqrt{1 + \left(-\dfrac{x}{\sqrt{9 - x^2}}\right)^2} \dd x  = \sqrt{1 + \dfrac{x^2}{9 - x^2}} \dd x \\
        &= \sqrt{\dfrac{9 - x^2 + x^2}{9 - x^2}} \dd x  = \dfrac{3}{\sqrt{9 - x^2}} \dd x.
    \end{align*}
    \begin{align*}
        A &= \int_{-2}^2 2\pi f(x) \dd s = \int_{-2}^2 2\pi \sqrt{9 - x^2} \cdot \dfrac{3}{\sqrt{9 - x^2}} \dd x \\
        &= \int_{-2}^2 6\pi \dd x = 6\pi (2 - (-2)) = 24\pi.
    \end{align*}
\end{ex}

Note that if the axis is shifted by $1$, (that is, rotated about the $y = -1$ axis), then $R = f(x) + 1$.

\begin{ex} Let $y = e^x$ for $x \in [0, 2]$. Rotating about the $y$-axis, set up the surface area of revolution.
    The radius is given by $R(y) = \ln(y)$. So 
    \[\dd s = \sqrt{1+\frac{1}{y^2}} \dd y,\] \chcomment{Typo25}
    and the surface area is given by
    \begin{align*}
        A = \int \dd A &= \int_1^{e^2} 2\pi \ln(y) \sqrt{1+\frac{1}{y^2}} \dd y.
    \end{align*}
\end{ex}
\subsubsection{Steps to Compute Surface Area}
\begin{enumerate}
    \item Determine whether $R$ is a function of $x$ or $y$; this will determine the variable used in the integration.
    \item Once you select the variable (either $x$ or $y$), write down the corresponding \textbf{infinitesimal line element} $\dd s$.
    \item Set up the \textbf{infinitesimal area element} $\dd A$ and the surface area integral. Be careful with the limits of integration.
\end{enumerate}

\subsection{Applications to physics and engineering} \chcomment{8.3}

\begin{tcolorbox}
    We likely won’t have time to cover this in class, but you’re welcome to read it if you're interested in the applications of the integral techniques we've learned. I'm happy to discuss any questions during discussion, office hours, or whenever we meet.
\end{tcolorbox}


\subsubsection{Hydrostatic Pressure and Force}
The force $F$ exerted by a fluid on a submerged plate is given by
\begin{equation*}
F = m g = \rho g A d 
\end{equation*}
where $\rho$ is the fluid density, $g$ is gravitational acceleration, $A$ is the surface area and $d$ is the depth/width.

\begin{ex} Compute the force on one end of a submerged cylinder with radius 3 and depth 10.
    Here we have $\rho = \rho(y)$ and $d = 7-y$ is a constant. Since the infinitesimal area $\Delta A$ is given by 
    \[\Delta A = 2 \sqrt{9-y_i^2} \Delta y,\]
    taking limits as $\Delta y \to 0$, we have $d$
    \[\dd A = 2 \sqrt{9-y^2} \dd y. \]
    Substitute into the force, we have
    \[F = \int_{-3}^3 \rho g (7-y)  \dd A = \int_{-3}^3 (7-y) \rho g \sqrt{9-y^2} \dd y.\]
\end{ex}


\subsubsection{Moments and Center of Mass}
For a lamina with density $\rho$, the total mass of the lamina is:
\[M = \int \rho(x, y) \dd A.\]
The moment about the $x$-axis is:
\[M_x = \rho \int_a^b f(x) \cdot \dfrac{f(x)}{2} \dd x.\]
The moment about the $y$-axis is:
\[M_y = \rho \int_a^b x f(x) \dd x.\]
The center of mass $(\overline{x}, \overline{y}) = (\dfrac{M_y}{M}, \dfrac{M_x}{M})$. (Notice the swap in $x$ and $y$).

\begin{ex}
    Find the center of mass of a semicircular plate, suppose $\rho$ is a constant:
    \begin{align*}
        \overline{y} &= \dfrac{1}{\rho A} \cdot \rho \int_{-r}^r \dfrac{1}{2} f(x)^2 \dd x = \dfrac{1}{\dfrac{1}{2}\pi r^2} \int_{-r}^r \dfrac{1}{2} (r^2 - x^2) \dd x \tag{Use symmetry}\\
        &= \dfrac{2}{\pi r^2} \int_{0}^r r^2 - x^2 \dd x = \dfrac{2}{\pi r^2} \left[ r^2 x - \dfrac{x^3}{3} \right]_0^r = \dfrac{4r}{3\pi}.
    \end{align*}
\end{ex}
\section{Chapter 10}
In this chapter, we introduce an alternative method for representing curves on the 2D plane. Within this framework, we will also explore the reformulated expressions for arc length and the surface area of a revolution. 

\subsection{Curves Defined by Parametric Equations} \chcomment{\S10.1} \chcomment{Week 6}
\begin{center}
\begin{tcolorbox}
    \begin{itemize}
        \item \textbf{New Concept}: Parametrization 
    \end{itemize}
\end{tcolorbox}
\end{center}

\subsubsection{Parametrization}
Consider a particle moving along a curve as follows:
\begin{figure}[ht]
    \centering
    \resizebox{0.5\textwidth}{!}{\begin{tikzpicture}[
  thick,>=stealth',
  declare function = {
    logx(\a,\b,\r) = \a*exp(-\b*\r)*cos(deg(\r));
    logy(\a,\b,\r) = \a*exp(-\b*\r)*sin(deg(\r));
  },
  point/.style={draw,thick,circle,inner sep=1pt,label={#1}},
  plot/.style={shinbashi, thick,smooth,samples=100}
  ]
  % Spiral parameters
  \def\a{5}
  \def\b{.2}
  % Axes
  \draw[->] (-4.5,0) -- (6,0) coordinate[label={below:$x$}] (A);
  \draw[->] (0,-3) -- (0,5) node[left] {$y$};
  % Spiral
  \draw[plot] plot[domain=0:25] ({logx(\a,\b,\x)},{logy(\a,\b,\x)});
  % Points M, O and M(t=0)
  \coordinate[label={below right:$O$}] (B) at (0,0);
  \node[point={above:$M$}] (C) at ({logx(\a,\b,.7)},{logy(\a,\b,.7)}) {};
  \node[below] at ({logx(\a,\b,0)},{logy(\a,\b,0)}) {$M(t=0)$};
  % Angle
  \draw pic[->,draw,"$\varphi$",angle radius=1.5cm] {angle};
  % Unit vectors
  \draw[->] (B) -- ($(B)!1.3!(C)$)       node[below right] {$\bm{e}_r$};
  \draw[->] (C) -- ($(C)!-1.5cm!90:(B)$) node[above right] {$\bm{e}_\theta$};
  \draw[->] (B) -- +(1,0)                node[below]       {$\bm{e}_x$};
  \draw[->] (B) -- +(0,1)                node[left]        {$\bm{e}_y$};
\end{tikzpicture}
} % Include your TikZ file
    \caption{A plot of the Golden Spiral.}
    \label{fig:golden-spiral}
\end{figure}
The curve cannot be expressed as $y = f(x)$ because it fails the vertical line test. However, by introducing a parameter $t$, representing the angle, the golden spiral can be represented as a parametric curve:
\[
\begin{aligned}
    x(t) &= a e^{b t} \cos(t), \quad t \geq 0, \\
    y(t) &= a e^{b t} \sin(t).
\end{aligned}
\]
(It can also be represented in polar coordinates (see \S10.3) as $r(\theta) = a e^{b \theta}$.)

\begin{defn}
    The system of equations 
    \begin{align*}
        x &= f(t), \\
        y &= g(t).
    \end{align*}
    is called a \bfemph{parametric equation/parametrization}, and the resulting curve $\sC$ is called a \bfemph{parametric curve}. We call $t$ a \bfemph{parameter}.  
\end{defn}





\begin{ex}[Unit Circle]
    \begin{align*}
        x &= \cos t, 0\leq t < 2\pi \\
        y &= \sin t.
    \end{align*}
    Note that $\cos^2 t + \sin^2 t = 1$, which implies $x^2 + y^2 = 1$. Thus, these equations parametrize a unit circle.
\end{ex}

Note that parameterizations are \textbf{not unique}.
\begin{ex}[Circle with Opposite Orientation]
    \begin{align*}
        x &= \sin 2t, , 0\leq t < \pi \\
        y &= \cos 2t.
    \end{align*}
    These parametrize the same circle as in Example 1.1, but with opposite orientation.
\end{ex}


\begin{ex}[Circle of Radius $r$ Centered at $(a, b)$]
    The equation of the circle is 
    \[(x - a)^2 + (y - b)^2 = r^2.\]
    The corresponding parametrization is given by:
    \begin{align*}
        x &= a + r \cos t, 0\leq t < 2\pi\\
        y &= b + r \sin t.
    \end{align*}
    Here $r$ represents the scaling and $a,b$ represents the translation.
\end{ex}

\begin{ex}[Parabola]
    The curve $x = 6 - 4y^2$ can be parametrized directly as:
    \begin{align*}
        x &= 6 - 4t^2, t \in \R \\
        y &= t.
    \end{align*}
\end{ex}


We can compute the $x$- and $y$-intercepts, critical points, and tangent lines (recall this is $y-y_0 = y'(x_0) (x-x_0)$) just as we do in rectangular coordinates. There will be examples in the discussion worksheet.



\subsection{Calculus with Parametric Curves} \chcomment{\S10.2}
\begin{center}
\begin{tcolorbox}
    \begin{itemize}
        \item \textbf{Calculus of Parametric Curves}:
        \begin{align*}
        \dd s &=  \sqrt{\left(\dfrac{\dd x}{\dd t}\right)^2 + \left(\dfrac{\dd y}{\dd t}\right)^2} \dd t. \\
        L &= \int_{t_1}^{t_2}  \dd s,\\
        S &= \int_{t_1}^{t_2} \dd A = \int_{t_1}^{t_2} 2\pi R \dd s.
    \end{align*}
    \end{itemize}
\end{tcolorbox}
\end{center}

Calculus techniques can be applied to analyze parametrized curves. 
    For a curve parametrized as $x = f(t)$ and $y = g(t)$, we can compute the following.
\begin{itemize}
    \item \textbf{Tangent slope:} when $\dfrac{\dd x}{\dd t} \neq 0$, the chain rule says $\dfrac{\dd y}{\dd t} = \dfrac{\dd y}{\dd x}\cdot \dfrac{\dd x}{\dd t}$. Hence,
    \[ \dfrac{\dd y}{\dd x} = \dfrac{\dfrac{\dd y}{\dd t}}{\dfrac{\dd x}{\dd t}}.\]
\begin{rmk}
    \begin{enumerate}
        \item $\dfrac{\dd x}{\dd t} \neq 0$ is required to take the quotient.
        
        \item $\dfrac{\dd x}{\dd t} = 0, \dfrac{\dd y}{\dd t} \neq 0$ corresponds to the vertical line $y = ct$.

        \item $\dfrac{\dd x}{\dd t} \neq 0, \dfrac{\dd y}{\dd t} = 0$ corresponds to the horizontal line $x = ct$.
    \end{enumerate}
\end{rmk}

We now introduce the substitution $x = x(t), \dd x = \dfrac{\dd x}{\dd t} \dd t$: 
    \item \textbf{Infinitesimal line element}
    \begin{align*}
        \dd s &= \sqrt{(1+ \left(\dfrac{\dd y}{\dd x}\right)^2} \dd x = \sqrt{(1+ \left(\dfrac{\dd y/\dd t}{\dd x/\dd t}\right)^2} \dfrac{\dd x}{\dd t} \dd t = \sqrt{\left(\dfrac{\dd x}{\dd t}\right)^2 + \left(\dfrac{\dd y}{\dd t}\right)^2} \dd t. 
    \end{align*}

    
    \item \textbf{Arc length:}
    \begin{align*}
        L &= \int_a^b \sqrt{(1+ \left(\dfrac{\dd y}{\dd x}\right)^2} \dd x = \int_{t_1}^{t_2} \sqrt{\left(\dfrac{\dd x}{\dd t}\right)^2 + \left(\dfrac{\dd y}{\dd t}\right)^2} \dd t. 
    \end{align*}
    
    
    \item \textbf{Surface area (for revolution):}
    \[ S = \int_a^b 2\pi R \dd s, \quad \text{where } \dd s = \sqrt{\left(\dfrac{\dd x}{\dd t}\right)^2 + \left(\dfrac{\dd y}{\dd t}\right)^2} \dd t. \]
\end{itemize}


\subsubsection{Example}
\begin{ex}
Consider the parametrization \begin{align*}
    x &= \cos^2 t, 0 \leq t \leq \dfrac{\pi}{4}.\\
    y &= \sin^2 t.
\end{align*}
The \textit{infinitesimal line element} is given by
\begin{align*}
    \dd s &= \sqrt{\left(\dfrac{dx}{dt}\right)^2 + \left(\dfrac{dy}{dt}\right)^2} \dd t = \sqrt{\left(-2\cos t \sin t\right)^2 + \left(2\sin t \cos t\right)^2} \dd t \\
    &= \sqrt{4\cos^2 t \sin^2 t + 4\sin^2 t \cos^2 t} \dd t  = \sqrt{8 \sin^2 t \cos^2 t } \dd t \\
   &= \sqrt{2} \sin(2t) \dd t.
\end{align*}

Hence the \textit{arc length} is given by
\begin{align*}
    L = \int \dd s = \int_0^{\dfrac{\pi}{4}} \sqrt{2} \sin(2t) \dd t = -\dfrac{\sqrt{2}}{2} \cos(2t) \Big|_0^{\dfrac{\pi}{4}} = \dfrac{\sqrt{2}}{2}.
\end{align*}
Rotated about the $x$-axis, the \textit{surface area} is given by
\begin{align*}
    A &= \int_0^{\dfrac{\pi}{4}} 2\pi \cdot \sin^2(t) \cdot \sqrt{2} \sin(2t) \dd t = 2 \sqrt{2} \pi \int_0^{\dfrac{\pi}{4}} \sin^2(t) \cdot \sin(2t) \dd t \\
    &= 2 \sqrt{2} \pi \int_0^{\dfrac{\pi}{4}} \dfrac{1-\cos(2t)}{2} \cdot \sin(2t) \dd t \\
    &= \sqrt{2} \pi \int_0^{\dfrac{\pi}{4}} \sin(2t) \dd t + \sqrt{2} \pi \int_0^{\pi/4} \dfrac{1}{2} \sin(4t) \dd t \\
    %
    &= \sqrt{2} \pi  \left[ -\dfrac{\cos(2t)}{2} \right]_0^{\pi/4} -  \sqrt{2} \pi \left[ -\dfrac{\cos(4t)}{8} \right]_0^{\dfrac{\pi}{4}} \\
    &= \sqrt{2} \pi \left(\dfrac{1}{2} - \left(\dfrac{1}{8} + \dfrac{1}{8}\right)\right)  = \sqrt{2} \pi \left(\dfrac{1}{2} - \dfrac{1}{4} \right) = \dfrac{\pi \sqrt{2}}{4}.
\end{align*}
\end{ex}

So far, our examples involve parametric curves easily expressed as the graph of a differentiable function. The next example shows that parametric curves are more general than those defined by a differentiable function, as seen in Chapter 8.
\begin{ex}
    Consider the parametrization \begin{align*}
    x &= 3\cos(\pi t), 0 \leq t \leq \dfrac{1}{2}.\\
    y &= 5t+2.
\end{align*}
Note that as $t$ increases, the $x$-coordinate oscillates, while the $y$-coordinate increases. You can use an online plotter, such as \href{https://www.geogebra.org/m/cAsHbXEU}{GeoGebra}, to view the graph in the 2D plane.

The \textit{infinitesimal line element} is given by
\begin{align*}
    \dd s &= \sqrt{\left(\dfrac{dx}{dt}\right)^2 + \left(\dfrac{dy}{dt}\right)^2} \dd t = \sqrt{\left(-3\pi \sin(\pi t)\right)^2 + 5^2} \dd t = \sqrt{9\pi^2 \sin^2 (\pi t) + 25} \dd t.
\end{align*}

Hence the \textit{arc length} is given by
\begin{align*}
    L = \int \dd s = \int_0^{1/2} \sqrt{9\pi^2 \sin^2 (\pi t) + 25} \dd t.
\end{align*}

Rotated about the $y$-axis the \textit{surface area} is given by
\begin{align*}
    A &= \int_0^{1/2} 2\pi \cdot 3\cos(\pi t) \cdot \sqrt{9\pi^2 \sin^2 (\pi t) + 25} \dd t \tag{Substitution $u = \sin(\pi t), \dd u = \pi \cos(\pi t)$}\\
    &= \int_0^{1/2} 6 \cdot \sqrt{9\pi^2 u^2 + 25} \dd u \tag{Trig integral $u = \dfrac{5}{3\pi} \tan \theta, \dd u = \dfrac{5}{3\pi} \sec^2 \theta \dd \theta$}\\
    &= \int_0^{\arctan{(3\pi/5)}}  6\cdot \sqrt{25\tan^2 \theta + 25} \dfrac{5}{3\pi} \cdot \sec^2 \theta \dd \theta = \int_0^{\arctan{(3\pi/5)}} 6\cdot 5 \sec\theta \cdot \dfrac{5}{3\pi} \sec^2 \theta \dd \theta\\
    &= \frac{25}{\pi} \int_0^{\arctan{(3\pi/5)}} \sec^3 \theta \dd \theta \\
    &= \frac{25}{\pi} \Big(\sec\theta \tan \theta + \ln|\sec \theta + \tan \theta| \Big) \bigg|_0^{\arctan{(3\pi/5)}} \approx 43.0705.
\end{align*}
To evaluate the last quantity, use $\sec \theta = \sqrt{1+\tan^2 \theta} = \dfrac{\sqrt{25+9\pi^2}}{5}$.
\end{ex}



\subsection{Polar Coordinates} \chcomment{\S10.3}
\begin{center}
\begin{tcolorbox}
    \begin{itemize}
        \item \textbf{New Concept}: Polar Coordinates 
        \[\begin{cases}
                x = r \cos \theta, \\
                y = r \sin \theta.
            \end{cases} \quad  \iff \quad \begin{cases}
                r = \sqrt{x^2 + y^2},  \\
                \theta = \tan^{-1}\left(\dfrac{y}{x}\right).
            \end{cases}\] 
            \item \textbf{Examples}:
            \begin{itemize}
                \item Circles $r=R$ or $r=a\cos\theta+ b\sin\theta$.
                \item Cardioids $r=a\pm a\cos\theta$ or $r=a\pm a\sin\theta$.
                \item Lima\c{c}ons $r=a\pm b\cos\theta$ or $r=a\pm b\sin\theta$, $a\neq b$.
            \end{itemize}
    \end{itemize}
\end{tcolorbox}
\end{center}


In this section, we will explore an alternative method for representing points on the Euclidean plane.
\subsubsection{Curves in Polar Coordinates}
In polar coordinates, a point $(r, \theta)$ is represented as:
\begin{align*}
    x &= r \cos \theta, \\
    y &= r \sin \theta.
\end{align*}
Conversely:
\begin{align*}
    r &= \sqrt{x^2 + y^2}, \text{ if } r>0 \\
    \theta &= \arctan\left(\dfrac{y}{x}\right).
\end{align*}
When $r<0$, the angle $\theta$ get added by $\pi$. See Example 3.2.

\begin{ex} Converting $(x,y) = (1, 1)$ to polar coordinates
    Given $(x, y) = (1, 1)$:
    \begin{align*}
        r &= \sqrt{1^2 + 1^2} = \sqrt{2}, \\
        \theta &= \tan^{-1}\left(\dfrac{1}{1}\right) = \dfrac{\pi}{4}.
    \end{align*}
    Thus, the polar coordinates are $(\sqrt{2}, \pi/4)$. 

    Note that the polar coordinate for this point is \textbf{not unique}. We can also pick $(\sqrt{2},2\pi+\dfrac{\pi}{4})$.
\end{ex}
\begin{ex}
    Convert $(r,\theta) = (-\sqrt{2},\dfrac{\pi}{4})$ to rectangular coordinates:
    We know 
    \begin{align*}
        x=r\cos\theta = -\sqrt{2} \cdot \frac{1}{\sqrt{2}} = -1,\\
        y=r\sin\theta = -\sqrt{2} \cdot \frac{1}{\sqrt{2}} = -1.
    \end{align*}
The point is symmetric to $(1,1)$ with respect to the origin, as shown in the previous example. You can check that $(r,\theta) = (-\sqrt{2},\pi+\dfrac{\pi}{4})$ corresponds to $(x,y) = (1,1)$.
\end{ex}

Some curves, such as circles or spirals, can be expressed as simple functions in terms of polar coordinates
\[F(r,\theta) = 0.\]
We will explore how to compute arc length and surface area using polar coordinates.


\subsubsection{Examples}
\begin{ex}[circle centered at the origin]
    In rectangular coordinates, a circle of radius $R$ centered at the origin is given by $x^2+y^2 = R^2$. In polar coordinates, this is given by $r = R, \theta \in [0, 2\pi]$.
    \begin{figure}[H]
        \centering
        \resizebox{0.3\textwidth}{!}{% CIRCLE arc segment
\begin{tikzpicture}
    \def\xmax{2.4}
    \def\R{2}
    \def\ang{55}
    \coordinate (O) at (0,0);
    \coordinate (X) at (\R,0);
    \coordinate (R) at (\ang:\R);
%2d plane and cicle    
    \draw (O) circle (\R);
    \draw[->,line width=0.9] (-\xmax,0) -- (1.08*\xmax,0) node[right] {$x$};
    \draw[->,line width=0.9] (0,-\xmax) -- (0,1.08*\xmax) node[left] {$y$};
%polar
    \draw pic[->,"$\Delta\theta$",draw=black,angle radius=20,angle eccentricity=1.5] {angle=X--O--R};
    \draw[->,enji, very thick] (O) -- (R) node[midway,right=4,above left=0] {$r$};
    \draw[enji,very thick,line cap=round] (X) arc (0:\ang:\R) ;
%rectangular
    \draw[dashed, shinbashi] (R) -- ({\R*cos(\ang)},0) node[below] {$x$};
    \draw[dashed, shinbashi] (R) -- (0,{\R*sin(\ang)}) node[left] {$y$};
\end{tikzpicture}} % Include your TikZ file
        \caption{Circle of radius $r$}
        \label{fig:circle}
    \end{figure}
\end{ex}

\begin{ex}
    Consider a circle centered at $(0, \dfrac{1}{2})$ with radius $\dfrac{1}{2}$, then $x^2+(y-\dfrac{1}{2})^2 = \dfrac{1}{4}$.
    We convert this into polar coordinates by plug in $x = r \cos\theta, y = r \sin\theta$:
    \begin{align*}
        x^2+(y-\dfrac{1}{2})^2 = \dfrac{1}{4} &\iff r^2 \cos^2\theta +(r \sin \theta-\dfrac{1}{2})^2 = \dfrac{1}{4}\\
        &\iff r^2 \cos^2\theta + r^2 \sin^2 \theta - r \sin \theta + \dfrac{1}{4} = \dfrac{1}{4}\\
        &\iff r^2 - r \sin \theta = 0.
    \end{align*}
    Since $r > 0$, this equation is equivalent to $r = \sin \theta$.

    \begin{figure}[H]
        \centering
        \resizebox{0.6\textwidth}{!}{\begin{tikzpicture}
    \def\xmax{2.4}
    \def\R{2}
    \def\ang{55}
    \coordinate (O) at (0,0);
    \coordinate (A) at (0,2);
    \coordinate (X) at (\R,0);
    \coordinate (R) at (1.7,3.1);
%2d plane and cicle    
    \draw[thick] (A) circle (\R);
    \draw[->,line width=0.9] (-\xmax,0) -- (1.08*\xmax,0) node[right] {$x$};
    \draw[->,line width=0.9] (0,-0.5) -- (0,4.5) node[left] {$y$};
% center and radius
    \draw[<->,enji, thick] (A) -- (R) node[midway,right=4,above left=0] {$\frac{1}{2}$};
    \fill[shinbashi] (xy polar cs:angle=90, radius=2) circle (2pt);
    \node[below left] at (0,2) {\(O\)};

\end{tikzpicture}

\begin{tikzpicture}[>=latex]
    % Draw the curve for r = 2*sin(theta)
    \draw[thick, domain=0:2*pi, samples=200, smooth] 
        plot (xy polar cs:angle=\x r, radius={4*sin(\x r)});
    
    % Label the equation
    %\node at (3,2.5) {$r=2\sin\theta$};
    
    % Draw axes
    \draw[->] (-2.5,0) -- (2.5,0) node[right] {\small $x$};
    \draw[->] (0,-0.5) -- (0,4.5) node[above] {\small $y$};

    % Add sample points at specified angles and connect to the origin
    \foreach \angle/\label in {
        30/$\theta=\frac{\pi}{6}$, 
        60/$\theta=\frac{\pi}{3}$, 
        90/$\theta=\frac{\pi}{2}$, 
        135/$\theta=\frac{3\pi}{4}$
    } {
        \pgfmathsetmacro{\r}{4*sin(\angle)} % Calculate r for each angle
        % Draw the point
        \fill[shinbashi] (xy polar cs:angle=\angle, radius=\r) circle (2pt);
        % Add label
        \node[anchor=west] at (xy polar cs:angle=\angle, radius=\r) {\small \label};
        % Draw solid line to the origin
        \draw[shinbashi] (0,0) -- (xy polar cs:angle=\angle, radius=\r);
    }
\end{tikzpicture}
} % Include your TikZ file
        \caption{Circle of radius $\dfrac{1}{2}$ centered at $(0,\dfrac{1}{2})$}
        \label{fig:circle shifted}
    \end{figure}

    Note that with the points winding around the \textit{full} circle once when $\theta \in [0,\pi]$.
\end{ex}

In general, in polar coordinates, the equations $r=R$ or $r=a\cos\theta+b\sin\theta$ represent circles.

\begin{ex}[$r=a+b\sin \theta$]
    The polar curve $r = a+b\sin \theta$ gives a cardioid.
    \begin{figure}[H]
        \centering
        \resizebox{\textwidth}{!}{\begin{tikzpicture}[>=latex]
    % Draw the cardioid
    \draw[thick, color=enji, domain=0:2*pi, samples=200, smooth] 
        plot (xy polar cs:angle=\x r, radius={1+0.5*sin(\x r)});
    
    % Label the equation
    %\node at (3,2.5) {$r=1+\sin\theta$};
    
    % Draw axes
    \draw[->] (-2,0) -- (2,0) node[right] {\small $x$};
    \draw[->] (0,-1) -- (0,2.5) node[above] {\small $y$};

    % Add sample points at specified angles and connect to the origin
    \foreach \angle/\label in {
        30/$\theta=\frac{\pi}{6}$, 
        60/$\theta=\frac{\pi}{3}$, 
        90/$\theta=\frac{\pi}{2}$, 
        135/$\theta=\frac{3\pi}{4}$
    } {
        \pgfmathsetmacro{\r}{1 + 0.5*sin(\angle)} % Calculate r for each angle
        % Draw the point
        \fill[shinbashi] (xy polar cs:angle=\angle, radius=\r) circle (2pt);
        % Add label
        \node[anchor=west] at (xy polar cs:angle=\angle, radius=\r) {\small \label};
        % Draw solid line to the origin
        \draw[shinbashi] (0,0) -- (xy polar cs:angle=\angle, radius=\r);
    }
\end{tikzpicture}


\begin{tikzpicture}[>=latex]
    % Draw the cardioid
    \draw[thick, color=enji, domain=0:2*pi, samples=200, smooth] 
        plot (xy polar cs:angle=\x r, radius={1+0.75*sin(\x r)});
    
    % Label the equation
    %\node at (3,2.5) {$r=1+\sin\theta$};
    
    % Draw axes
    \draw[->] (-2,0) -- (2,0) node[right] {\small $x$};
    \draw[->] (0,-1) -- (0,2.5) node[above] {\small $y$};

    % Add sample points at specified angles and connect to the origin
    \foreach \angle/\label in {
        30/$\theta=\frac{\pi}{6}$, 
        60/$\theta=\frac{\pi}{3}$, 
        90/$\theta=\frac{\pi}{2}$, 
        135/$\theta=\frac{3\pi}{4}$
    } {
        \pgfmathsetmacro{\r}{1 + 0.75*sin(\angle)} % Calculate r for each angle
        % Draw the point
        \fill[shinbashi] (xy polar cs:angle=\angle, radius=\r) circle (2pt);
        % Add label
        \node[anchor=west] at (xy polar cs:angle=\angle, radius=\r) {\small \label};
        % Draw solid line to the origin
        \draw[shinbashi] (0,0) -- (xy polar cs:angle=\angle, radius=\r);
    }
\end{tikzpicture}

\begin{tikzpicture}[>=latex]
    % Draw the cardioid
    \draw[thick, color=enji, domain=0:2*pi, samples=200, smooth] 
        plot (xy polar cs:angle=\x r, radius={1+sin(\x r)});
    
    % Label the equation
    %\node at (3,2.5) {$r=1+\sin\theta$};
    
    % Draw axes
    \draw[->] (-2,0) -- (2,0) node[right] {\small $x$};
    \draw[->] (0,-1) -- (0,2.5) node[above] {\small $y$};

    % Add sample points at specified angles and connect to the origin
    \foreach \angle/\label in {
        30/$\theta=\frac{\pi}{6}$, 
        60/$\theta=\frac{\pi}{3}$, 
        90/$\theta=\frac{\pi}{2}$, 
        135/$\theta=\frac{3\pi}{4}$
    } {
        \pgfmathsetmacro{\r}{1 + sin(\angle)} % Calculate r for each angle
        % Draw the point
        \fill[shinbashi] (xy polar cs:angle=\angle, radius=\r) circle (2pt);
        % Add label
        \node[anchor=west] at (xy polar cs:angle=\angle, radius=\r) {\small \label};
        % Draw solid line to the origin
        \draw[shinbashi] (0,0) -- (xy polar cs:angle=\angle, radius=\r);
    }
\end{tikzpicture}

\begin{tikzpicture}[>=latex]
    % Draw the cardioid
    \draw[thick, color=enji, domain=0:2*pi, samples=200, smooth] 
        plot (xy polar cs:angle=\x r, radius={1+2*sin(\x r)});
    
    % Label the equation
    %\node at (3,2.5) {$r=1+\sin\theta$};
    
    % Draw axes
    \draw[->] (-2,0) -- (2,0) node[right] {\small $x$};
    \draw[->] (0,-1) -- (0,2.5) node[above] {\small $y$};

    % Add sample points at specified angles and connect to the origin
    \foreach \angle/\label in {
        30/$\theta=\frac{\pi}{6}$, 
        60/$\theta=\frac{\pi}{3}$, 
        90/$\theta=\frac{\pi}{2}$, 
        135/$\theta=\frac{3\pi}{4}$
    } {
        \pgfmathsetmacro{\r}{1 + 2*sin(\angle)} % Calculate r for each angle
        % Draw the point
        \fill[shinbashi] (xy polar cs:angle=\angle, radius=\r) circle (2pt);
        % Add label
        \node[anchor=west] at (xy polar cs:angle=\angle, radius=\r) {\small \label};
        % Draw solid line to the origin
        \draw[shinbashi] (0,0) -- (xy polar cs:angle=\angle, radius=\r);
    }
\end{tikzpicture}} % Include your TikZ file
        \caption{Convex lima\c{c}on $r=1+\dfrac{1}{2}\sin\theta$, dimpled lima\c{c}on $r=1+\dfrac{3}{4}\sin\theta$, cardioid $r=1+\sin\theta$ and lima\c{c}on with inner loop $r=1+2\sin\theta$.}
        \label{fig:cardioid}
    \end{figure}
\end{ex}
Note that as $b\to 0$, the polar curve converges to a circle centered at the origin.

In general, in polar coordinates, the equations $r=a\pm b\cos\theta$ or $r=a\pm b\sin\theta$ represent 
\begin{itemize}
    \item Cardioids: if $a=b$.
    \item Lima\c{c}ons with an inner loop: if $a<b$.
    \item Lima\c{c}ons without an inner loop: if $a>b$.
\end{itemize}


\newpage

\subsection{Areas and Lengths in Polar Coordinates} \chcomment{\S10.4} \chcomment{Week 7}
\begin{center}
\begin{tcolorbox}
    \begin{itemize}
        \item \textbf{Calculus with Polar Coordinates}:
        \begin{align*}
        A &= \int_{\theta_1}^{\theta_2} \frac{1}{2}r^2 \dd \theta \tag{area enclosed by $r=f(\theta)$}\\
        \dd s &=  \sqrt{\left(\dfrac{\dd x}{\dd \theta}\right)^2 + \left(\dfrac{\dd y}{\dd \theta}\right)^2} \dd \theta = \sqrt{r^2 + \left(\dfrac{\dd r}{\dd \theta}\right)^2} \dd \theta. \\
        L &= \int_{\theta_1}^{\theta_2} \dd s,\\
        S &= \int_{\theta_1}^{\theta_2} \dd A = \int_{\theta_1}^{\theta_2} 2\pi R \dd s.\tag{surface area of revolution of a polar curve}
    \end{align*}
    \end{itemize}
\end{tcolorbox}
\end{center}


\subsubsection{Tangent}
Now consider a polar curve of the form $r = f(\theta)$. Then,
\[x = f(\theta) \cos \theta, \quad y = f(\theta) \sin \theta.\]
The derivative of the parametrization with respect to $\theta$ is given by
\[\dfrac{dx}{d\theta} = f'\cos \theta - f\sin \theta, \quad \dfrac{dy}{d\theta} = f'\sin \theta + f\cos \theta.\]
We can compute its tangent by the chain rule:
\[
    \dfrac{dy}{dx} = \dfrac{\dfrac{dy}{d\theta}}{\dfrac{dx}{d\theta}}=\dfrac{f'\cos \theta + f\sin \theta}{f'\sin \theta - f\cos \theta}
\]
\begin{ex}
    Let $r = 1 + \sin \theta$. Compute $\dfrac{dy}{dx}$.
    \[x = (1 + \sin \theta) \cos \theta, \quad y = (1 + \sin \theta) \sin \theta. \]
    
    Differentiating,
    \[ \dfrac{dx}{d\theta} = \cos \theta \cdot \sin \theta - (1+\sin \theta) \cos \theta, \quad \dfrac{dy}{d\theta} = \cos \theta \cdot \cos \theta - (1+\sin \theta) \sin \theta.\]
    
    Thus,
    \[\dfrac{dy}{dx} = \dfrac{\cos \theta + 2\cos \theta \sin \theta}{\cos^2 \theta - \sin^2 \theta - \sin \theta} = \dfrac{\cos \theta + \sin (2\theta)}{\cos (2\theta) - \sin \theta}.\]
    
    Note that:
    \begin{equation*}
        \lim_{\theta \to \frac{3\pi}{2}^-} \dfrac{dy}{dx} = \lim_{\theta \to \frac{3\pi}{2}^-} \dfrac{\cos \theta + \sin (2\theta)}{\cos (2\theta) - \sin \theta} = \lim_{\theta \to \frac{3\pi}{2}^-} \dfrac{-\cos \theta +2 \cos (2\theta)}{-2\sin (2\theta) - \cos \theta} = -\infty. \tag{L'H}
    \end{equation*}
    This means the tangent blows up at $\dfrac{3\pi}{2}$.
\end{ex}

\subsubsection{Area Enclosed by Polar Curves}

For $r = f(\theta)$, the area of a sector is approximately
\[\Delta A \approx \dfrac{1}{2} r^2 \Delta \theta.\]

Using a Riemann sum,
\[ A \approx \sum_{i=1}^n \frac{1}{2} \left[f(\xi_i)\right]^2 \Delta \theta \quad \implies \quad A = \int_{\theta_1}^{\theta_2} \dfrac{1}{2} \left[f(\xi)\right]^2 \dd \theta = \int_{\theta_1}^{\theta_2} \dfrac{1}{2} r^2 \dd \theta. \]

\begin{ex}
    Find the area enclosed by one loop of the four-leaved rose $r = \cos(2\theta)$.
    \begin{align*}
        A &= \dfrac{1}{2} \int_{-\pi/4}^{\pi/4} r^2 \dd \theta = \dfrac{1}{2} \int_{0}^{\pi/4} \cos^2(2\theta) \dd \theta \tag{Integrand is an even function}\\
        &= \int_{0}^{\pi/4} \dfrac{1 + \cos(4\theta)}{2} \dd \theta = \dfrac{1}{2} \left[\theta + \dfrac{1}{4}\sin(4\theta)\right]_{-\pi/4}^{\pi/4} = \frac{\pi}{4}.
    \end{align*}
    \chcomment{Typo22}

    \begin{figure}[H]
        \centering
        \resizebox{0.5\textwidth}{!}{\usetikzlibrary{patterns.meta}

\newcommand\eq{80*sqrt(2)*cos(2*\x r)}

\begin{tikzpicture}
  \path
  coordinate (E) at (0.6,4)
  coordinate (F) at (7,3)
  coordinate (P) at (3,3)
  coordinate (I) at (10,0)
  coordinate (O) at (0,0);
  
  % Shading the region from -\pi/4 to \pi/4
  \begin{scope}
    \clip[domain=0:6.28,samples=200,smooth] plot (canvas polar cs:angle=\x r,radius={\eq});
    \draw[fill=shinbashi, opacity=0.3] (5, 5) -- (5, -5) -- (0, 0) -- cycle; % gray-filled triangle

    \end{scope}
  
  \node at (O) [above=1mm] {$O$};   % nodes also accept distance for labels
  \node at (E) [right] {$r=4\sqrt{2}\cos{2\theta}$};
  
  % Plot the curve
  \draw[thick ,domain=0:6.28,samples=200,smooth] plot (canvas polar cs:angle=\x r,radius={\eq});
  
  % Draw x and y axes
  \draw[->] (-5,0) -- (5,0) node[right] {$x$};
  \draw[->] (0,-5) -- (0,5) node[above] {$y$};
\end{tikzpicture}
} % Include your TikZ file
        \caption{Four-leaf $r=\cos (2\theta)$}
        \label{fig:four-leaf}
    \end{figure}
\end{ex}


\subsubsection{Arc Length}
We compute the infinitesimal line element $\dd s$ as follows:
\begin{align*}
    \dd s = \sqrt{1 + \left(\dfrac{\dd y}{\dd x}\right)^2} \dd x &= \sqrt{\left(\dfrac{\dd x}{\dd \theta}\right)^2 + \left(\dfrac{\dd y}{\dd \theta}\right)^2} \dd \theta.
\end{align*}
Note that 
\begin{align*}
    \left(\dfrac{\dd x}{\dd \theta}\right)^2 + \left(\dfrac{\dd y}{\dd \theta}\right)^2 &= (r')^2 \cos^2\theta - 2rr'\cos\theta \sin\theta + r^2 \sin^2\theta \\
    &+ (r')^2 \sin^2\theta + 2rr'\sin\theta \cos\theta + r^2 \cos^2\theta\\
    &= (r')^2 + r^2, \qquad \text{ where } r' = f'(\theta).
\end{align*}

So 
\[\dd s = \sqrt{r^2 + \left(\dfrac{\dd r}{\dd \theta}\right)^2} \dd \theta = \sqrt{\Big(f(\theta)\Big)^2 + \Big(f'(\theta)\Big)^2} \dd \theta.\]
The arc length of a polar curve is given by:
\begin{align*}
    L &= \int \dd s= \int_{\theta_1}^{\theta_2} \sqrt{\Big(f(\theta)\Big)^2 + \Big(f'(\theta)\Big)^2} \dd \theta.
\end{align*}


\begin{ex}
    Find the arc length of $r = \theta$, $0 \leq \theta \leq 1$.
    \[
        L = \int_0^1 \sqrt{\theta^2 + 1} \dd \theta.
    \]
    
    We have seen this integral in \S7. Using the substitution $\theta = \tan x$, $d\theta = \sec^2 x \dd x$, we have
    \begin{align*}
        L &= \int_0^{\pi/4} \sec x \sec^2 x \dd x \tag{IBP with $u = \sec x$, $v = \tan x$}\\
        &= \sec x \tan x - \int_0^{\pi/4} \tan x \cdot \tan x \sec x \dd x = \sec x \tan x - \int_0^{\pi/4} \tan^2 x \sec x \dd x\\
        &= \sec x \tan x - \int_0^{\pi/4} (\sec^2 x - 1) \sec x \dd x = \sec x \tan x - L + \int_0^{\pi/4} \sec x \dd x
    \end{align*}
    This implies 
    \begin{align*}
        2L &= \sec x \tan x + \int_0^{\pi/4} \sec x \dd x\\
        &=\sec x \tan x + \ln|\sec x + \tan x| \Big|_0^{\pi/4} = \frac{1}{2}\Big(\sqrt{2}+ \ln(1+\sqrt{2})\Big). 
    \end{align*}
\end{ex}

\newpage
\subsection{Summary of Arc Length and Area Integrals}
Here is a summary of the integrals we learned in Chapters 8 and 10. In practice, you only need to remember the formulae in bold text; the others can be derived from them using the chain rule (for differentiation) and the substitution rule (for integration). 

\begin{center}
\begin{tcolorbox}
    \begin{center}
    \renewcommand{\arraystretch}{6}
    \begin{tabular}{|p{2cm}|p{4.5cm}|p{3.5cm}|p{3.5cm}|} 
        \hline
         & $ \dd s $ & surface area of revolution $ \dd A $ & surface area enclosed by curve\\ 
        \hline
        $ (x,y) $ & \begin{minipage}{4.5cm}
            $\boldsymbol{\sqrt{(\dd x)^2 + (\dd y)^2}}$ \\
            \\
            $= \sqrt{1 + \left(\dfrac{\dd y}{\dd x}\right)^2} \dd x$ \\ 
            $= \sqrt{\left(\dfrac{\dd x}{\dd y}\right)^2 + 1} \dd y$
        \end{minipage} & \begin{minipage}{3.5cm}
            $ \boldsymbol{2\pi R \dd s} $, \\
            $R$ is a function of $x$ or $y$
        \end{minipage}
        & \begin{minipage}{4cm}
            $\dint_{x_1}^{x_2} f(x) \dd x$ \\
            or $\dint_{y_1}^{y_2} g(y) \dd y$
        \end{minipage}\\
        \hline
        \begin{minipage}{2cm}
            $ \big( x(t), y(t) \big) $ \\
            $x' = \dfrac{\dd x}{\dd t}$\\
            $y' = \dfrac{\dd y}{\dd t}$
        \end{minipage}
        & $ \sqrt{(x')^2 + (y')^2} \dd t $ & \begin{minipage}{3.5cm}
            $ 2\pi R \dd s $, \\
            $R$ is a function of $t$
        \end{minipage} & \begin{minipage}{4cm}
            $\dint_{t_1}^{t_2} f(x(t)) \; x'\dd t$ \\
            or $\dint_{t_1}^{t_2} g(y(t)) \; y'\dd y$
        \end{minipage} \\
        \hline
        \begin{minipage}{2cm}
            $ (r,\theta)$, \\
            $r = r(\theta) $,\\
            $r' = \dfrac{\dd r}{\dd \theta} $
        \end{minipage} & \begin{minipage}{4.5cm}
            $ \sqrt{\left(\dfrac{\dd x}{\dd \theta}\right)^2 + \left(\dfrac{\dd y}{\dd \theta}\right)^2} \dd \theta $ \\
            \\
            $= \boldsymbol{\sqrt{r^2 + (r')^2} \dd \theta} $
        \end{minipage} & \begin{minipage}{3.5cm}
            $ 2\pi R \dd s $, \\
            $R$ is a function of $\theta$
        \end{minipage} & $ \boldsymbol{\dint_{\theta_1}^{\theta_2} \frac{1}{2} \; r^2 \dd \theta} $ \\
        \hline
    \end{tabular}
    \end{center}
\end{tcolorbox}
\end{center}
Note that the last row can also be derived from the second row, but it is convenient to remember them.
\section{Chapter 11: Infinite Sequences and Series} \chcomment{Week 8}
In this chapter, we introduce sequences and series. We will focus on how to test for convergence using tools like the Integral Test, the Comparison Tests, and the Ratio and Root Tests. We will also discuss alternating and absolutely convergent series, along with strategies for analyzing them. Finally, we will explore power series and Taylor and Maclaurin series, and how to use them to approximate functions.

\subsection{Sequences} \chcomment{\S11.1}
\begin{center}
\begin{tcolorbox}
    \begin{itemize}
        \item \textbf{New Concept}: Sequence, limit of sequence, sequence converges/diverges 
        \item \textbf{Example to memorize}:
        \begin{itemize}
        \item \begin{align*}
            \dlim_{n \to \infty} \dfrac{1}{n^p} = \begin{cases} 
            0 & \text{if } p > 0 \\
            1 & \text{if } p = 0 \\
            \infty & \text{if } p < 0
        \end{cases}
        \end{align*}
        
        \item \begin{align*}
                \dlim_{n \to \infty} r^n = \begin{cases} 
            0 & \text{if } -1 < r < 1 \\
            1 & \text{if } r = 1 \\
            \infty & \text{if } r > 1 \\
            \text{DNE} & \text{if } r < -1
            \end{cases}
        \end{align*}
        \end{itemize}
    \end{itemize}
\end{tcolorbox}
\end{center}


\begin{defn}
     A \bfemph{sequence} is an infinite list of members written with an order. We denote the sequence as $\{a_1, a_2, \ldots, a_n, \ldots\}$, $\{a_n\}$ or $\{a_n\}_{n=1}^\infty$.
\end{defn}

\begin{ex}[sequences]\leavevmode
\begin{enumerate}
    \item $\{1, 2, 3, 9, \ldots\}$.
    \item $\{7, 1, 8, 2, 8, \ldots\}$.
\end{enumerate}
\end{ex}

Some sequences can be defined by giving a formula for the $n$-th term $a_n$.

\begin{ex}[sequences given by formulae]\leavevmode
\begin{enumerate}
    \item $a_n = \dfrac{1}{n}$, $\{a_n\} = \{1, \dfrac{1}{2}, \dfrac{1}{3}, \ldots\}$.
    \item $a_n = (-1)^{n-1}$, $\{a_n\} = \{-1, 1, -1, 1, \ldots\}$.
    \item $a_n = \dfrac{1}{3^n}$, $\{a_n\} = \{\dfrac{1}{3}, \dfrac{1}{9}, \dfrac{1}{27}, \ldots\}$.
\end{enumerate}
\end{ex}


Some sequences may not have a simple/explicit defining equation.

\begin{ex}[sequences without explicit formulae] \leavevmode
\begin{enumerate}
    \item $a_n = $ the digit in the $n$-th decimal place of $\pi$.
    \item The Fibonacci sequence: $a_1 = 1, a_2 = 1, a_n = a_{n-1} + a_{n-2}$ \\
    $\{a_n\} = \{1, 1, 2, 3, 5, 8, 13, 21, \ldots\}$. 
\end{enumerate}
\end{ex}


\begin{rmk}
    A sequence can be thought of as a function $f$ defined only on the natural numbers. Therefore, we can examine properties such as the graph and convergence. For example, \[ \lim_{n \to \infty} a_n = 0. \]
\end{rmk}

\begin{defn}
    A sequence has \bfemph{limit} $L$ if for any $\epsilon > 0$, there is an $N$ such that if $n > N$, then $|a_n - L| < \epsilon$. (We write this as 
    \[\forall \epsilon > 0, \exists N \text{ s.t. if } n > N, \text{ then } |a_n - L| < \epsilon.\]
    
    We say $a_n$ \bfemph{converges} to $L$ and denote it as $\dlim_{n\to \infty} a_n = L$.
\end{defn}


\begin{rmk}[Intuition]
    $\dlim_{n \to \infty} a_n = \infty$ means that for every positive number $M$, there is an integer $N$ such that if $n > N$, then $a_n > M$.
\end{rmk}
\begin{figure}[ht]
    \centering
    \resizebox{0.6\textwidth}{!}{\begin{tikzpicture}[scale=1.2]
    % Axes
    \draw[thick, ->] (-0.5,0) -- (6.5,0) node[right] {\small $n$};
    \draw[thick, ->] (0,-0.5) -- (0,3.5) node[above] {\small $a_n$};

    % Dots representing sequence values
    \foreach \x/\y in {1/1.5, 2/2.1, 3/1.5, 4/2.3, 5/2.5, 6/3} {
        \fill (\x,\y) circle (2pt);
    }

    % Label for M
    \draw[thick, dashed, enji] (-0.3,2) -- (6.5,2);
        \node[left] at (-0.3,2) {\small $M$};

    % Label for N
    \node[below] at (3,-0.1) {\small $N$};
    \draw[thick, dashed, enji] (3,-0.3) -- (3,2);

    % Arrow emphasizing terms beyond N exceeding M
    \draw[->, thick, shinbashi] (3.2,2.2) -- (5.5,2.8);

    % Label for a_n
    \node[right] at (5.5,2.8) {\small $a_n$};
\end{tikzpicture}
} % Include your TikZ file
    \label{fig:sequence}
    \caption{Illustration of the definition of $\lim_{n \to \infty} a_n = \infty$. Beyond some $N$, all $a_n$ exceed $M$.}
\end{figure}
\begin{ex}[limit of a sequence] \leavevmode
\begin{enumerate}
    \item $\dlim_{n \to \infty} \dfrac{n}{n+1} = 0 = \dlim_{n \to \infty} \dfrac{1}{1+1/n} = 1$.

    \begin{align*}
        \dlim_{n \to \infty} \dfrac{1}{n^p} = \begin{cases} 
        0 & \text{if } p > 0 \\
        1 & \text{if } p = 1 \\
        \infty & \text{if } p < 0
        \end{cases}
    \end{align*}
    
    \item 
    \begin{align*}
            \dlim_{n \to \infty} r^n = \begin{cases} 
        0 & \text{if } -1 < r < 1 \\
        1 & \text{if } r = 1 \\
        \infty & \text{if } r > 1 \\
        \text{DNE} & \text{if } r < -1
        \end{cases}
    \end{align*}
\end{enumerate}
\end{ex}


\subsubsection{Limit Laws for Sequences}

\begin{center}
\begin{tcolorbox}
     \begin{itemize}
         \item \textbf{Tools for evaluate limits}: limit law, squeeze theorem, 
         \item \textbf{Continuous function commutes with limit}: 
         \[f \text{ continuous } \implies \dlim_{n \to \infty} f(a_n) = f\Big(\dlim_{n \to \infty} a_n \Big).\]
     \end{itemize}   
\end{tcolorbox}
\end{center}


If $\{a_n\}$ and $\{b_n\}$ are convergent sequences, then
\begin{enumerate}
    \item $\dlim_{n \to \infty} (a_n \pm b_n) = \dlim_{n \to \infty} a_n \pm \dlim_{n \to \infty} b_n$.
    
    \item $\dlim_{n \to \infty} (c a_n) = c \dlim_{n \to \infty} a_n$, $c$ constant.
    
    \item $\dlim_{n \to \infty} (a_n b_n) = \dlim_{n \to \infty} a_n \cdot \lim_{n \to \infty} b_n$.
    
    \item $\dlim_{n \to \infty} \dfrac{a_n}{b_n} = \dfrac{\dlim_{n \to \infty} a_n}{\dlim_{n \to \infty} b_n}$, provided $\dlim_{n \to \infty} b_n \neq 0$.
    
    \item $\dlim_{n \to \infty} (a_n)^p = \left(\dlim_{n \to \infty} a_n \right)^p$.
\end{enumerate}

Note that if the convergent condition fails, the equality could also fail. For example, if $a_n = (-1)^n$ and $b_n = \dfrac{1}{n}$. $\dlim_{n \to \infty} a_n = \mathrm{DNE}$, but $\dlim_{n \to \infty} a_n b_n = 0$.

\begin{thm}[Squeeze Theorem]
    If $b_n \leq a_n \leq c_n$ holds for every $n \geq N$ ($N$ is some natural number) and $\dlim_{n \to \infty} b_n = \lim_{n \to \infty} c_n = L$, then $\dlim_{n \to \infty} a_n = L$.
\end{thm}
\begin{figure}[h]
    \centering
    \resizebox{0.6\textwidth}{!}{\begin{tikzpicture}[scale=1.2]
    % Axes
    \draw[thick, ->] (-0.5,0) -- (6.5,0) node[right] {\small $n$};
    \draw[thick, ->] (0,-0.5) -- (0,3.5) node[above] {\small Sequence values};

    % Dashed horizontal line for L
    \draw[thick, dashed] (-0.3,2) -- (6.5,2);
    \node[left] at (-0.3,2) {\small $L$};

    % Points for sequences
    \foreach \x/\yb/\ya/\yc in {1/1.2/1.6/2.5, 2/1.5/1.8/2.4, 3/1.7/1.9/2.2, 4/1.85/1.95/2.1, 5/1.95/1.99/2.02, 6/1.99/2/2.01} {
        \fill[shinbashi] (\x,\yb) circle (2pt); % b_n (lower bound)
        \fill[kohaku] (\x,\ya) circle (2pt); % a_n (middle sequence)
        \fill[enji] (\x,\yc) circle (2pt); % c_n (upper bound)
    }

    % Labels for sequences
    \node[left, shinbashi] at (1,1.2) {\small $b_n$};
    \node[left, enji] at (1,2.5) {\small $c_n$};
    \node[left, kohaku] at (1,1.6) {\small $a_n$};

    % Annotations
    %\draw[->, thick] (3,2.5) -- (6,2);
    \node[right] at (3.2,2.5) {\small $b_n, c_n \to L$};

\end{tikzpicture}} % Include your TikZ file
    \label{fig:squeeze thm}
    \caption{Visualization of the Squeeze Theorem: if $b_n \leq a_n \leq c_n$ and both $b_n$ and $c_n$ converge to $L$, then $a_n$ also converges to $L$.}
\end{figure}

\begin{ex}
    If $a_n = (-1)^n \dfrac{1}{n}$, $b_n = -\dfrac{1}{n}$ and $c_n = \dfrac{1}{n}$. Then $b_n \leq a_n \leq c_n$ and $\dlim_{n \to \infty} b_n = \dlim_{n \to \infty} c_n = 0$ implies $\dlim_{n \to \infty} a_n = 0$ by the Squeeze Theorem.
\end{ex}
\begin{thm}\leavevmode
    \begin{enumerate}
        \item If $\dlim_{n \to \infty} |a_n| = 0$ then $\dlim_{n \to \infty} a_n =0$.

        \item \textit{Continuous function commutes with limit}. If $f$ is \textbf{continuous}, then  $\dlim_{n \to \infty} a_n = L$ implies $\dlim_{n \to \infty} f(a_n) = f \left(\dlim_{n \to \infty} a_n\right)= f(L)$.
    \end{enumerate}
\end{thm}

\begin{ex}
    \begin{enumerate}
        \item $\dlim_{n \to \infty} \sin\left(\dfrac{1}{n}\right) = \sin\left(\lim_{n \to \infty} \dfrac{1}{n}\right) = \sin(0) = 0$.

        \item $\dlim_{n \to \infty} \dfrac{\ln(n+2)}{\ln(1+4n)}$. Note that this is same as  
        \begin{align*}
            \dlim_{x \to \infty} \dfrac{\ln(x+2)}{\ln(1+4x)} = \dlim_{x \to \infty} \dfrac{\dfrac{1}{x+2}}{\dfrac{4}{1+4x}} = \dlim_{x \to \infty} \dfrac{1+4x}{4(x+2)} = 1.  \tag{L'H\^opital's rule}
        \end{align*}

        \item \begin{align*}
            \dlim_{x \to \infty} \left(1 + \dfrac{1}{n}\right)^n = \dlim_{x \to \infty} e^{\left(1 + \dfrac{1}{n}\right)^n} = e^{\dlim_{x \to \infty} \left(1 + \dfrac{1}{n}\right)^n} = e.  \tag{Can apply L'H\^opital's rule to compute the limit}
        \end{align*}
    \end{enumerate}
\end{ex}


\subsection{Series} \chcomment{\S11.2}
\begin{center}
\begin{tcolorbox}
     \begin{itemize}
         \item \textbf{New concept}: series, partial sum, converges/diverges
         \item \textbf{Examples to memorize}: \begin{itemize}
             \item Geometric series
                 \begin{align*}
                \sum_{n=0}^\infty r^n &= \begin{cases} 
                    \dfrac{1}{1 - r} & \text{if } |r| < 1, \\
                    \infty & \text{if } r \geq 1\\
                    \text{DNE} & \text{if } r \leq -1.
                \end{cases}
            \end{align*}
            \item Harmonic series $\dsum_{n=1}^\infty \dfrac{1}{n}$ diverges.
         \end{itemize}
         \item \textbf{Tools to study series}: Series laws, $\dsum a_n \text{ converges } \implies a_n \to 0$.
     \end{itemize}   
\end{tcolorbox}
\end{center}

\begin{defn}
    We call $\dsum_{n=1}^\infty a_n$ or $\dsum a_n$ a \bfemph{series}, and 
    \[S_N = \sum_{n=1}^N a_n = a_1 + a_2 + \ldots + a_N\]
    the \bfemph{partial sum}. 
\end{defn}

\begin{rmk}
    Note that $S_N$ is itself a sequence. So it makes sense to talk about whether $S_N$ converges or not.
\end{rmk}

\begin{defn}
    The series $\dsum a_n$ is called \bfemph{convergent} if its partial sum is convergent. Otherwise, $\dsum a_n$ is called \bfemph{divergent}.
\end{defn}

\begin{ex}[Geometric Series]
    Consider $a_n = r^n$, where $r$ is the common ratio.
    \begin{align*}
        a_0 &= 1,  &&S_0 = a_0 = 1, \\
        a_1 &= r,  &&S_1 = a_0 + a_1 = 1 + r, \\
        a_2 &= r^2,  &&S_2 = a_0 + a_1 + a_2 = 1 + r + r^2.\\
        & && \vdots\\
        & && S_N = 1 + r + r^2 + \ldots + r^N.
    \end{align*}
    
    Let $R_N = \dsum_{\red{n=0}}^N r^n = 1 + r + r^2 + \ldots + r^N$, then
    \begin{align*}
        r R_N &= r + r^2 + \ldots + r^{N+1}, \\
        R_N - r R_N &= 1 - r^{N+1}, \\
        R_N &= \dfrac{1 - r^{N+1}}{1 - r}, \quad \text{for } r \neq 1.
    \end{align*}
    Thus,
    \begin{align*}
        \sum_{n=0}^\infty r^n &= \begin{cases} 
            \dfrac{1}{1 - r} & \text{if } |r| < 1, \\
            \infty & \text{if } r \geq 1\\
            \text{DNE} & \text{if } r \leq -1.
        \end{cases}
    \end{align*}
    
    Also, note that $\dsum_{n=1}^\infty r^n = \dfrac{r}{1-r}$ because $\dsum_{n=0}^\infty r^n = 1 + \dsum_{n=1}^\infty r^n$. Thus, the starting point matters.
\end{ex}

\begin{ex}
    Compute $\dsum_{n=1}^\infty 2^{2n} \cdot 6^{1-n}$ using the formula from the previous example. \chcomment{Typo22}
    \begin{align*}
        \sum_{n=1}^\infty 2^{2} \cdot 6^{1-n} &= \sum_{n=1}^\infty 4^n \cdot 6 \cdot \left(\dfrac{1}{6}\right)^n = 6 \cdot \sum_{n=1}^\infty \left(\dfrac{4}{6}\right)^n = 6 \cdot \sum_{n=1}^\infty \left(\dfrac{2}{3}\right)^n \tag{Here $r = \dfrac{2}{3}$}\\
        &= 6 \cdot \dfrac{\dfrac{2}{3}}{1-\dfrac{2}{3}} = 6\cdot 2 = 12.
    \end{align*}
\end{ex}

\begin{ex}[Harmonic Series]
    $\dsum_{n=1}^\infty \dfrac{1}{n} = 1 + \dfrac{1}{2} + \dfrac{1}{3} + \dfrac{1}{4} + \ldots$ \\
    Partial sums:
    \begin{align*}
        S_2 &= 1 + \dfrac{1}{2}, \\
        S_4 &= 1 + \dfrac{1}{2} + \blue{\dfrac{1}{3} + \dfrac{1}{4}} \\
        &> 1 + \dfrac{1}{2} + \blue{\dfrac{1}{4} + \dfrac{1}{4}} = 1+\dfrac{2}{2}, \\
        S_8 &= 1 + \dfrac{1}{2} + \blue{\dfrac{1}{3} + \dfrac{1}{4}} + \red{\dfrac{1}{5} + \dfrac{1}{6} + \dfrac{1}{7} + \dfrac{1}{8}}\\
        &> 1 + \dfrac{1}{2} + \blue{\dfrac{1}{4} + \dfrac{1}{4}} + \red{\dfrac{1}{8} + \dfrac{1}{8} + \dfrac{1}{8} + \dfrac{1}{8}} = 1+\dfrac{3}{2}, \\
        S_{2^n} &= 1 + \dfrac{1}{2} + \dfrac{1}{3} + \ldots + \dfrac{1}{2^n} > 1+\dfrac{n}{2} \xrightarrow{n \to \infty} \infty.
    \end{align*}
    Hence, $\dsum_{n=1}^\infty \dfrac{1}{n}$ diverges.    
\end{ex}

\begin{ex}[Telescope series]
    Check that $\dsum_{n=1}^\infty \dfrac{1}{n(n+1)} = 1$.

    We note that:
    \begin{align*}
        \dfrac{1}{n(n+1)} &= \dfrac{1}{n} - \dfrac{1}{n+1}, \\
        S_N &= \left(1 - \blue{\dfrac{1}{2}}\right) + \left(\blue{\dfrac{1}{2}} - \red{\dfrac{1}{3}}\right) + \left(\red{\dfrac{1}{3}} - \gray{\dfrac{1}{4}}\right) + \cdots + \left(\gray{\dfrac{1}{n-1}} - \blue{\dfrac{1}{n}}\right) + \left(\blue{\dfrac{1}{n}} - \dfrac{1}{n+1}\right)\\
        &= 1 - \dfrac{1}{n+1} \xrightarrow{n \to \infty} 1.
    \end{align*}
    
    Hence, $\dsum_{n=1}^\infty \dfrac{1}{n(n+1)}$ converges to 1.
\end{ex}



\begin{thm}
    If $\dsum_{n=1}^\infty a_n$ and $\dsum_{n=1}^\infty b_n$ converge, and $c$ is a constant, then 
\begin{enumerate}
    \item $\dsum_{n=1}^\infty a_n \pm b_n = \dsum_{n=1}^\infty a_n \pm \dsum_{n=1}^\infty b_n$.
    
    \item $\dsum_{n=1}^\infty c a_n = c \dsum_{n=1}^\infty a_n$.
\end{enumerate}
\end{thm}

\begin{ex}
    Evaluate $\dsum_{n=1}^\infty \dfrac{3}{n(n+1)} + \dfrac{1}{2^n}$.
    
    We have 
    \begin{align*}
        \dsum_{n=1}^\infty \dfrac{1}{2^n} = \dfrac{1}{1-\dfrac{1}{2}} - 1 = 2-1 = 1 \quad \text{and} \quad \dsum_{n=1}^\infty \dfrac{1}{n(n+1)} = 1.
    \end{align*}
    So the original series converges to $3\cdot 1 + 1 = 4$.
\end{ex}

\begin{thm}
    If $\dsum_{n=1}^\infty a_n$ converges, then $\dlim_{n\to \infty} a_n = 0$.
\end{thm}
\begin{proof}
    By definition, we know if the series converges to some real number $L$, we have
    \begin{align*}
        &\dlim_{n\to \infty} S_{N-1} = \dlim_{n\to \infty} S_N = L\\
        \implies &\dlim_{N\to \infty} a_N = \dlim_{N\to \infty} (S_N - S_{N-1}) = \dlim_{N\to \infty} S_N - \dlim_{n\to \infty} S_{N-1} = L - L = 0.
    \end{align*}
\end{proof}

\begin{coro}[The Divergent Test] 
    If $\dlim_{n \to \infty} a_n \neq 0$, then $\dsum_{n=1}^\infty a_n$ diverges.
\end{coro}

\begin{rmk}
    Note that when $\dlim_{n\to\infty} a_n = 0$, there is no conclusion. For example, 
    \begin{align*}
        \dlim_{n \to \infty} \dfrac{1}{n(n+1)} &= 0, \text{ we know } \dsum_{n=1}^\infty \dfrac{1}{n(n+1)} = 1.\\
        \dlim_{n \to \infty} \dfrac{1}{n} &= 0, \text{ but } \hspace{25pt}\dsum_{n=1}^\infty \dfrac{1}{n} \text{ diverges}.
    \end{align*}
\end{rmk}

\begin{ex}[The Divergent Test] \leavevmode
    \begin{enumerate*}
        \item $\dsum_{n=1}^\infty (-1)^n$.
        \item $\dsum_{n=1}^\infty \left( 1+\dfrac{1}{n} \right)^n$.
        \item $\dsum_{n=1}^\infty \dfrac{n}{n+1}$.
    \end{enumerate*}
\end{ex}


\subsection{The Integral Test and Estimates of Sums} \chcomment{\S11.3}
\begin{center}
\begin{tcolorbox}
    \begin{itemize}
        \item \textbf{New tool for testing convergence}: 
        \begin{itemize}
            \item \textbf{The Integral Test}
            $f$ positive, continuous, decreasing for $x \geq 1$, and let $a_n = f(n)$. Then:
            \[\dsum_{n=1}^\infty a_n \text{ converges } \iff \dint_1^\infty f(x) \dd x \text{ converges.}\]
            
            \item \textbf{Error estimate}: 
            \[\dint_{N+1}^\infty f(x) \dd x \leq R_N = S - S_N \leq \dint_{N}^\infty f(x) \dd x.\]
        \end{itemize}
    \end{itemize}
\end{tcolorbox}
\end{center}

We have been computing the exact value of a series so far for some special cases. However, in general, this is quite difficult. In those cases, we are interested in finding an estimate.

\subsubsection{The Integral Test}
\begin{thm}
    Suppose \begin{enumerate*}[label = \circled{\arabic*}]
        \item $f(x) > 0$ is a  \item continuous and \item decreasing function for $x \geq 1$, and 
        \item let $a_n = f(n)$. 
    \end{enumerate*} Then: 
    \[\dsum_{n=1}^\infty a_n \text{ converges } \iff \dint_1^\infty f(x) \dd x \text{ converges.}\]
    Moreover,
    \[\dsum_{n=k+1}^\infty a_n \leq \dint_k^\infty f(x) \dd x \leq \dsum_{n=k}^\infty a_n.\]
\end{thm}

The error of this estimate is given by $R_N = \dsum_{n=1}^\infty a_n - \dsum_{n=1}^N a_n = \dsum_{n=N+1}^\infty a_n$. We have 
\[\int_{N+1}^\infty f(x) \dd x \leq R_N \leq \int_{N}^\infty f(x) \dd x.\]
\begin{figure}[h]
    \centering
    \resizebox{0.85\textwidth}{!}{
    \begin{tikzpicture}
    % Axes
    \draw[->] (0,0) -- (6.5,0) node[right] {$x$};
    \draw[->] (0,0) -- (0,4) node[above] {$y$};
    
    % Function
    \draw[domain=0.5:6.5,samples=100,smooth,thick] plot ({\x},{3/(\x+1)}) node[right] {$f(x)$};
    
    % Subintervals and rectangles for Right Riemann Sum
    \foreach \x in {1,2,3,4,5} {
        \fill[shinbashi!50, opacity=0.5] (\x,0) rectangle (\x+1,{3/(\x+1)}); % Light blue and transparent
        \draw[thick] (\x,0) rectangle (\x+1,{3/(\x+1)});   % Rectangle borders
        \draw[dashed] (\x+1,0) -- (\x+1,{3/(\x+1)}); % Vertical dashed lines
    }
    
    % Labels
    \node[below] at (1,0) {$a_1$};
    \node[below] at (2,0) {$a_2$};
    \node[below] at (3,0) {$a_3$};
    \node[below] at (4,0) {$a_4$};
    \node[below] at (5,0) {$a_5$};
    \node[below] at (6,0) {$a_6$};
    
    \node at (3,3) [anchor=north] {Left Riemann Sum};
\end{tikzpicture}
    \quad
    \begin{tikzpicture}
    % Axes
    \draw[->] (0,0) -- (6.5,0) node[right] {$x$};
    \draw[->] (0,0) -- (0,4) node[above] {$y$};
    
    % Function
    \draw[domain=0.5:6.5,samples=100,smooth,thick] plot ({\x},{3/(\x+1)}) node[right] {$f(x)$};
    
    % Subintervals and rectangles for Right Riemann Sum
    \foreach \x in {1,2,3,4,5} {
        \fill[shinbashi!50, opacity=0.5] (\x,0) rectangle (\x+1,{3/(\x+2)}); % Right endpoint height
        \draw[thick] (\x,0) rectangle (\x+1,{3/(\x+2)});  % Rectangle borders
        \draw[dashed] (\x,0) -- (\x,{3/(\x+2)}); % Vertical dashed lines from x_i
    }
    
    % Labels
    \node[below] at (1,0) {$a_1$};
    \node[below] at (2,0) {$a_2$};
    \node[below] at (3,0) {$a_3$};
    \node[below] at (4,0) {$a_4$};
    \node[below] at (5,0) {$a_5$};
    \node[below] at (6,0) {$a_6$};
    
    \node at (3,3) [anchor=north] {Right Riemann Sum};
\end{tikzpicture}
} % Include your TikZ file
    \label{fig:Riemann sums}
    \caption{Upper and lower bounds for the Integral Test}
\end{figure}
\begin{proof}
    Since $f$ is decreasing, we know from the Riemann sum (see the picture above)
    \[ \sum_{n=k+1}^\infty a_n = \sum_{n=k}^\infty a_{n+1} \leq \int_k^\infty f(x) \dd x \leq \sum_{n=k}^\infty a_n.\]
    The left inequality with $k=N$ gives $R_N \leq \dint_N^\infty f(x) \dd x$; the right inequality with $k=N+1$ gives $R_N \leq \dint_{N+1}^\infty f(x) \dd x$. 
\end{proof}



\begin{ex} $\dsum_{n=1}^\infty \dfrac{1}{n^2}$ converges.

Let $f(x) = \dfrac{1}{x^2}$. For $x \geq 1$, $f(x)$ is continuous, positive, and decreasing. Then $\displaystyle \int_1^\infty \dfrac{1}{x^2} \dd x $ converges implies $\dsum_{n=1}^\infty \dfrac{1}{n^2}$ converges.
\end{ex}

\begin{ex}
\begin{align*}
    \dsum_{n=1}^\infty  \dfrac{1}{n^p} \quad \begin{cases} 
    \text{converges}  & \text{if } p > 1 \\
    \text{diverges} & \text{if } p \leq 1.
\end{cases}
\end{align*}
Recall $p > 1$, $\int_1^\infty \dfrac{1}{x^p} \dd x$ converges. For $p \leq 1$, it diverges. Apply the Integral Test.
\end{ex}

\begin{ex}
    $\dsum_{n=1}^\infty  \dfrac{1}{n^2+1} $ converges
\end{ex}
Let $f(x) = \dfrac{1}{x^2+1}$ > 0. For $x \leq 1$, we check $f$ is continuous and decreasing: 
\[f'(x) = -(x^2+1)^{-2} \cdot 2x < 0, x \geq 1.\]

Apply the Integral Test as follows:
\[\int_0^\infty \dfrac{1}{x^2+1} \dd x = \lim_{t \to \infty} \Big(\arctan x |_1^t \Big) = \lim_{t \to \infty} \Big(\arctan t - \dfrac{\pi}{4} \Big) = \dfrac{\pi}{2}- \dfrac{\pi}{4} = \dfrac{\pi}{4} \leq \infty.\]
So the series converges.

\begin{ex}
    There is an example in the discussion worksheet regarding the error estimate. See Question 2 in Worksheet w8-2.
\end{ex}



\subsection{The Comparison Tests} \chcomment{\S11.4} \chcomment{Week 10}
\begin{center}
\begin{tcolorbox}
    \begin{itemize}
        \item \textbf{New tools for testing convergence}: 
        \begin{itemize}
            \item \textbf{The (Direct) Comparison Test (DCT) for Series}: \begin{enumerate*}[label = \circled{\arabic*}]
            \item $0\leq a_n \leq b_n$ \item for all $n \geq N$, then
            \end{enumerate*}
            \begin{align*}
                \dsum_{n=N}^\infty b_n \text{ converges } \quad &\implies  \quad \dsum_{n=N}^\infty a_n \text{ converges}, \\
                \dsum_{n=N}^\infty a_n \text{ diverges } \quad &\implies  \quad \dsum_{n=N}^\infty b_n \text{ diverges}.
            \end{align*}
            \item \textbf{The Limit Comparison Test (LCT)}: \begin{enumerate*}[label = \circled{\arabic*}]
            \item $a_n > 0$ and $b_n > 0$ 
            \item for all $n \geq N$, and 
            \item $\dlim_{n \to \infty} \dfrac{a_n}{b_n} = c$, where
            \item $0 < c < \infty$.
            \end{enumerate*}
            Then 
            \[\dsum_{n=N}^\infty a_n \text{ converges } \quad \iff \quad \dsum_{n=N}^\infty b_n \text{ converges}.\]
        \end{itemize}
    \end{itemize}
\end{tcolorbox}
\end{center}
The idea of the Comparison Test for sequences is similar to that for integrals.

\subsubsection{The (Direct) Comparison Test for Series (DCT)}
\begin{thm}
    Suppose $\dsum a_n$ and $\dsum b_n$ are series such that \begin{enumerate*}[label = \circled{\arabic*}]
        \item $0 < a_n \leq b_n$ 
        \item for all $n \geq N$. 
    \end{enumerate*}
    Then:
    \begin{itemize}
        \item If $\dsum_{n=N}^\infty b_n$ converges, then $\dsum_{n=N}^\infty a_n$ converges.
        \item If $\dsum_{n=N}^\infty a_n$ diverges, then $\dsum_{n=N}^\infty b_n$ diverges.
    \end{itemize}
\end{thm}

\begin{rmk}
    Here we can compare the Direct Comparison Test for series with the Comparison Test for integrals:
    \begin{itemize}
        \item $a_n$ and $b_n$ play the roles of $f$ and $g$.
        \item Integrals are replaced by summations.
        \item The lower bound $x \geq a$ is replaced by $n \geq N$.
    \end{itemize}
\end{rmk}


\begin{ex} Show that $\dsum_{n=1}^\infty a_n = \dsum_{n=1}^\infty  \dfrac{5}{2n^2+4n+3}$ converges.

\textit{Step 1}. Note that $2n^2+4n+3 \geq 2n^2$ for $n \geq 1$. This implies 
\[\circled{1}~0 < a_n : = \dfrac{5}{2n^2+4n+3} \leq \dfrac{5}{2n^2} =: b_n \quad \circled{2}~ \text{for } n \geq 1.\]
\textit{Step 2}. We apply the DCT to conclude that 
\[\dsum_{n=1}^\infty  \dfrac{5}{2n^2} = \dfrac{5}{2} \dsum_{n=1}^\infty  \dfrac{1}{n^2} < \infty. \quad \implies \quad \dsum_{n=1}^\infty a_n \text{ converges}.\]
\end{ex}

\begin{ex} Show that $\dsum_{n=1}^\infty b_n = \dsum_{n=1}^\infty  \dfrac{\ln n}{n}$ diverges.

\textit{Step 1}. Note that $\ln n > 1$ for $n > e$. We take $\circled{2}~N=3$, which is the next integer after $e$. This implies 
\[ \circled{1}~0 < a_n := \dfrac{1}{n} \leq b_n \dfrac{\ln n}{n},\quad \text{when} \quad \circled{2}~n \geq 3.\]

\textit{Step 2}. To show that $\dsum_{n=3}^\infty a_N = \dsum_{n=3}^\infty \dfrac{1}{n}$ diverges, note that 
\begin{align*}
    \dsum_{n=3}^\infty \dfrac{1}{n} &=  - \dfrac{1}{1} - \dfrac{1}{2} = \text{harmonic series (divergent)} - \text{ finite number}.
\end{align*}
So $\dsum_{n=3}^\infty a_n$ diverges


\textit{Step 3}. We apply the DCT to conclude that
\[\dsum_{n=3}^\infty  \dfrac{1}{n} \text{ diverges}. \quad \implies \quad \dsum_{n=3}^\infty b_n \text{ converges}.\]


\textit{Step 4}. To show that $\dsum_{n=1}^\infty b_n$ diverges, note that 
\begin{align*}
    \dsum_{n=1}^\infty b_n &= (b_1 + b_2) + \dsum_{n=3}^\infty b_n = \text{finite number } +  \text{ divergent series}.
\end{align*}
So $\dsum_{n=1}^\infty b_n$ diverges
\end{ex}



\subsubsection{The Limit Comparison Test (LCT)}
\begin{thm}
    Suppose $\dsum a_n$ and $\dsum b_n$ are series with \begin{enumerate*}[label = \circled{\arabic*}]
        \item $a_n > 0$ and $b_n > 0$, and
        \item $\dlim_{n \to \infty} \dfrac{a_n}{b_n} = c$, where 
        \item $0 < c < \infty$.
    \end{enumerate*}
    Then  
    \[\dsum_{n=N}^\infty a_n \text{ converges } \quad \iff \quad \dsum_{n=N}^\infty b_n \text{ converges}.\]
\end{thm}

\begin{rmk}
    Note that the $\dlim_{n \to \infty} \dfrac{a_n}{b_n} = c$ is saying $a_n$ and $cb_n$ have the same growth rate as $n \to \infty$.
\end{rmk}

\begin{ex} 
    Show that $\dsum_{n=1}^\infty a_n = \dsum_{n=1}^\infty  \dfrac{1}{2^n - 1}$ converges.

    Take $b_n = \dfrac{1}{2^n}$. Then
    \[\lim_{n \to \infty} \dfrac{a_n}{b_n} = \lim_{n \to \infty} \dfrac{\dfrac{1}{2^n - 1}}{\dfrac{1}{2^n}} = \lim_{n \to \infty} \dfrac{2^n}{2^n-1} = \lim_{n \to \infty} \dfrac{1}{1-\dfrac{1}{2^n}} = 1.\]
    Apply the Limit Comparison Test, we conclude that $\dsum_{n=1}^\infty b_n$ converges implies  $\dsum_{n=1}^\infty \dfrac{1}{2^n - 1}$ converges.
\end{ex}

\begin{ex} 
    Show that $\dsum_{n=1}^\infty a_n = \dsum_{n=1}^\infty \dfrac{2n^2 + 3n}{\sqrt{5+n^5}}$ diverges.

    Take $b_n = \dfrac{2n^2}{n^{5/2}} = \dfrac{2}{\sqrt{n}}$ (this is the dominant part). Then
    \[\lim_{n \to \infty} \dfrac{a_n}{b_n} = \lim_{n \to \infty} \dfrac{\dfrac{2n^2 + 3n}{\sqrt{5+n^5}}}{\dfrac{2}{\sqrt{n}}} = \lim_{n \to \infty} \dfrac{2n^{5/2}+3n^{1/2}}{2\sqrt{5+n^5}} = \lim_{n \to \infty} \dfrac{2+\dfrac{3}{n}}{2\sqrt{\dfrac{5}{n^5}+1}} = \dfrac{2}{2} = 1.\]
    Apply the Limit Comparison Test, we conclude that $\dsum_{n=1}^\infty b_n$ diverges implies $\dsum_{n=1}^\infty a_n$ diverges.
\end{ex}

\subsection{Alternating Series} \chcomment{\S11.5}
\begin{center}
\begin{tcolorbox}
    \begin{itemize}
        \item \textbf{New Concept}: Alternating series
        \begin{itemize}
        \item \textbf{The Alternating Series Test}: 
        
        $a_n$ \begin{enumerate*}[label = \circled{\arabic*}]
        \item positive, \item decreasing, \item $\dlim_{n \to \infty} a_n = 0$.
        \end{enumerate*}
        $\implies \quad \dsum_{n=0}^\infty (-1)^n a_n$ converges.
        \item \textbf{Error estimate}: $|R_N| = |S - S_N| \leq a_{n+1}$.
        \end{itemize}
        
        \item \textbf{Example to memorize}: The alternating harmonic series $\dsum \dfrac{(-1)^{n+1}}{n}$ converges.
    \end{itemize}
\end{tcolorbox}
\end{center}
So far, we have studied series with positive terms. In this section, we will study series whose terms are alternating series, such as
\begin{itemize}
    \item $\dsum \dfrac{(-1)^{n+1}}{n}$ (alternating harmonic series).
    \item $\dsum (-1)^{n} a_n$, where $a_n > 0$ and terms alternate in sign.
\end{itemize}

\subsubsection{The Alternating Series Test}
The following theorem tells us how to determine if an alternating series converges or diverges.

\begin{thm}
    Given an alternating series $\dsum_{n=0}^\infty (-1)^n a_n$, if 
    
    \begin{center}
    \begin{enumerate*}[label = \circled{\arabic*}]
        \item $a_n > 0$, 
        \item $a_{n+1} \leq a_n$ for all $n$, and 
        \item $\lim_{n \to \infty} a_n = 0$,
    \end{enumerate*}
    \end{center}
    
    then $\dsum_{n=0}^\infty (-1)^n a_n$ converges.
\end{thm}

\begin{proof}
    Without loss of generality, we prove this for the series $S = \sum_{n=1}^\infty (-1)^{n+1} a_n$. The other case follows by shifting the sequence or multiplying the series by $-1$.

    Let $S_N = \sum_{n=1}^N (-1)^{n+1} a_n$
    By \circled{2} we have $a_n - a_{n+1} \geq 0$.
    \begin{align*}
        S_2 &= a_1 - a_2 \geq 0\\
        S_4 &= S_2 + a_3 - a_4 \geq S_2 \quad \text{as} \quad a_3 - a_4 \geq 0\\
        &\vdots\\
        S_{2n} &= S_{2n-2} + a_{2n-1} - a_{2n} \geq S_{2n-2} \quad \text{as} \quad a_{2n-1} - a_{2n} \geq 0
    \end{align*}
    So, $\{S_{2n}\}$ is an increasing sequence.
    
    Note that 
    \begin{align*}
        S_{2n} &= a_1 - a_2 + a_3 - a_4 + a_5 - \cdots + a_{2n-1} - a_{2n}\\
        &= a_1 - (a_2 - a_3) - (a_4 - a_5) + \cdots - (a_{2n-2} - a_{2n-1}) - a_{2n} \\
        &\leq a_1 
    \end{align*}

Since each of the quantities in parentheses and $a_{2n}$ are positive. So $\{S_{2n}\}$ is an increasing sequence bounded from above. So the limit exists.

\[\lim_{n \to \infty} S_{2n} = S.\]

We can apply the same procedure to $S_{2n+1}$, and see the limit also exists. Moreover,
\[\lim_{n \to \infty} S_{2n+1} = \lim_{n \to \infty} (S_{2n} + a_{2n+1}) = \lim_{n \to \infty} S_{2n} + \lim_{n \to \infty} a_{2n+1} = S + 0 = S.\]

So, alternating series is convergent. 

\end{proof}
\begin{rmk}
    From the above proof, we see that if $\dlim_{n \to \infty} a_n$ diverges, the series also diverges. So the \textbf{diverges test still holds}.   
\end{rmk}

\begin{ex}[Alternating Harmonic Series]
    Show that $\dsum_{n=0}^\infty \dfrac{(-1)^{n+1}}{n}$ converges.
    
    We first check that the Alternating Series Test applies: 
    \begin{enumerate}[label = \circled{\arabic*}]
        \item $a_n = \dfrac{1}{n} > 0$,
        \item $a_{n+1} = \dfrac{1}{n+1} < \dfrac{1}{n} = a_n$ for all $n$, and
        \item $\dlim_{n \to \infty} a_n = 0$.
    \end{enumerate}
    By the Alternating Series Test tells us that the series converges. 
\end{ex}


\begin{ex}
    Show that $\dsum_{n=0}^\infty \dfrac{(-1)^n n^2}{n^3+1}$ converges.
    
    We first check that the Alternating Series Test applies: 
    \begin{enumerate}[label = \circled{\arabic*}]
        \item $a_n = \dfrac{n^2}{n^3+1} > 0$.
        
        \item $a_{n+1} < a_n$ for $n \geq 2$ because the function $f(x) = \dfrac{x^2}{x^3+1}$ is decreasing (not obvious, we compute the derivative):
        \[f'(x) = \dfrac{x(2-x^3)}{(x^3+1)^2} < 0, \text{ where } x > \sqrt[3]{2}.\]
        
        \item $\dlim_{n \to \infty} a_n = \dlim_{n \to \infty} \dfrac{n^2}{n^3+1} = \dlim_{n \to \infty} \dfrac{\dfrac{1}{n}}{1+\dfrac{1}{n^3}} = 0$.
    \end{enumerate}
    Apply the Alternating Series Test to $\dsum_{n=2}^\infty (-1)^n a_n$ (because we need $n\geq 2$), we conclude that $\dsum_{n=2}^\infty (-1)^n a_n$ converges. So $\dsum_{n=0}^\infty (-1)^n a_n = a_0 - a_1 + \dsum_{n=2}^\infty (-1)^n a_n$ also converges.
\end{ex}

\subsubsection{Estimating alternating series}
\begin{thm}[Alternating series estimation]
    Given $\dsum_{n=0}^\infty (-1)^n a_n$, $a_n > 0$ satisfying 
    \begin{center}
    \begin{enumerate*}[label = \circled{\arabic*}]
        \item $a_n > 0$.
        \item $a_{n+1} \leq \dfrac{1}{n} = a_n$ for all $n$.
        \item $\dlim_{n \to \infty} a_n = 0$.
    \end{enumerate*}
    \end{center}
    Then $|R_N| = |S - S_N| \leq a_{n+1}$.
\end{thm}

\begin{proof}
    From the proof of the Alternating Series Test, we know $S$ is in between $S_N$ and $S_{N+1}$ for all $N$. So 
    \[|R_N| = |S - S_N| \leq |S_{N+1} - S_N| = a_{N+1}.\]
\end{proof}


\newpage
\subsection{Summary} 
So far, we have learned different ways to determine the convergence of a series with positive terms or alternating terms.

\subsubsection{Summary of the Convergence Tests in Previous Sections}

\begin{center}
\begin{tcolorbox}
\begin{enumerate}
    \item \textbf{Definition}: computing $\dlim_{N \to \infty} S_N$
    
    \item \textbf{The Divergence Test}: $\dlim_{n \to \infty} a_n \neq 0 \quad \implies \quad \sum_n a_n$ diverges
\end{enumerate}

\textbf{The following only works for series with positive terms.}
\begin{enumerate}[resume]
    \item \textbf{The Integral Test}: 
    
    $f(x)$ \begin{enumerate*}[label = \circled{\arabic*}]
        \item positive, \item continuous, \item decreasing \item for $x \geq N$, and 
        \item $a_n = f(n)$. 
    \end{enumerate*} Then:
    \[\sum_{n=N}^\infty a_n \text{ converges } \iff \int_N^\infty f(x) \dd x \text{ converges.}\]
    
    \item \textbf{The Direct Comparison Test}: \begin{enumerate*}[label = \circled{\arabic*}]
        \item $0 < a_n \leq b_n$ 
        \item for all $n \geq N$. 
    \end{enumerate*}
    Then:
    \begin{itemize}
        \item If $\dsum_{n=N}^\infty b_n$ converges, then $\dsum_{n=N}^\infty a_n$ converges.
        \item If $\dsum_{n=N}^\infty a_n$ diverges, then $\dsum_{n=N}^\infty b_n$ diverges.
    \end{itemize}
    
    \item \textbf{The Limit Comparison Test}: 
    
    \begin{enumerate*}[label = \circled{\arabic*}]
        \item $a_n > 0$ and $b_n > 0$, and
        \item $\dlim_{n \to \infty} \dfrac{a_n}{b_n} = c$, where 
        \item $0 < c < \infty$.
    \end{enumerate*}
    Then  
    \[\dsum_{n=N}^\infty a_n \text{ converges } \quad \iff \quad \dsum_{n=N}^\infty b_n \text{ converges}.\]
\end{enumerate}

\textbf{The following only works for series with alternating terms.}
\begin{enumerate}[resume]
    \item \textbf{The Alternating Series Test}: $a_n$ \begin{enumerate*}[label = \circled{\arabic*}]
        \item positive, \item decreasing, \item $\dlim_{n \to \infty} a_n = 0$.
        \end{enumerate*}
        Then $\dsum_{n=0}^\infty (-1)^n a_n$ converges.
\end{enumerate}
\end{tcolorbox}
\end{center}

\subsubsection{Steps to Apply Convergence Tests}

\begin{enumerate}
    \item Determine the type of the series. (e.g. harmonic, geometric, positive, alternating)
    \item Decide which test to use. 
    \\
    For example, if the series has negative terms, you cannot apply the integral test.  
    \item Check all the assumptions hold. Be careful with the number $N$. 
    \item Apply the test to $\dsum_{n=N}^{\infty} b_n$.
    \\
    If you are given $\dsum_{n=1}^{\infty} b_n$, $N \neq 1$, then use
    \[
    \dsum_{n=1}^{\infty} b_n = \dsum_{n=1}^{N-1} b_n + \dsum_{n=N}^{\infty} b_n = \text{ finite number} + \text{ convergence or divergence series}
    \]
    \item Conclusion.
\end{enumerate}

\subsubsection{Error estimate}
\begin{center}
\begin{tcolorbox}
    $R_N = S - S_N$
    \begin{enumerate}
        \item \textbf{The integral test}:
            \[\int_{N+1}^\infty f(x) \dd x \leq R_N \leq \int_{N}^\infty f(x) \dd x.\]
        \item \textbf{The Alternating Series Test}: $|R_N| \leq a_{n+1}$.
    \end{enumerate}
\end{tcolorbox}
\end{center}

There are two types of estimation questions we could ask.

\begin{enumerate}
    \item Given $N$, what is the error bound? For this question, apply the Error estimate directly.
    \item Given an upper bound $\epsilon$, what is the smallest integer $N$ that makes $|R_N| < \epsilon$? For this question, use $|R_N| < \epsilon$ to give a lower bound for $N$.
\end{enumerate}
In the next several sections, we will study convergence tests for more general series.

\subsection{Absolute Convergence and the Ratio and Root Tests} \chcomment{\S11.6} \chcomment{Week 11}
\begin{center}
\begin{tcolorbox}
    \begin{itemize}
        \item \textbf{New Concept}: absolute convergence and conditional convergence.
        
        \item \textbf{New tool for testing convergence}: \textbf{The Ratio/Root Test}: \[L_{ratio} = \lim_{n \to \infty} \left| \dfrac{a_{n+1}}{a_n} \right|, \qquad L_{root} = \lim_{n \to \infty} \sqrt[n]{|a_n|}.\]
        Then:
        \begin{itemize}
            \item If $L < 1$, the series converges absolutely.
            \item If $L > 1$, the series diverges.
            \item If $L = 1$, the test is inconclusive.
        \end{itemize}
    \end{itemize}
\end{tcolorbox}
\end{center}
\begin{defn}
    A series $\dsum a_n$ is called \bfemph{absolutely convergent} if the series of absolute values $\dsum |a_n|$ converges. 
\end{defn}

\begin{defn}
    A series $\dsum a_n$ is called \bfemph{conditionally convergent} if it converges but is not absolutely convergent.
\end{defn}

Note that absolute convergence is stronger than convergence
\[\text{If } \sum |a_n| \text{ converges, then } \sum a_n \text{ converges.}\]

\begin{proof}
    Observe that:
    \[-a_n \leq |a_n| \leq a_n \quad \implies \quad  0 \leq a_n + |a_n| \leq 2 |a_n|.\]
    We call $A_n = a_n + |a_n|$, $B_n = 2 |a_n|$. By the Comparison Test, $\dsum B_n$ converges implies $\dsum |A_n|$ converges. Then 
    \[\sum a_n = \sum A_n - \sum |a_n| < \infty. \]  
\end{proof}


\subsubsection{Examples}
\begin{ex}
    The series $\dsum \dfrac{(-1)^{n+1}}{n}$ is \textbf{conditionally convergent} because:
    \begin{itemize}
        \item $\dsum \dfrac{1}{n}$ diverges (harmonic series).
        \item $\dsum \dfrac{(-1)^{n+1}}{n}$ converges by the Alternating Series Test.
    \end{itemize}
\end{ex}

\begin{ex}
    The series $\dsum \dfrac{(-1)^{n+1}}{n^2}$ is \textbf{absolutely convergent} because:
    \begin{itemize}
        \item $\dsum \dfrac{1}{n^2}$ converges by the $p$-series test with $p = 2 > 1$.
        \item $\dsum \dfrac{(-1)^{n+1}}{n^2}$ converges by the Alternating Series Test.
    \end{itemize}
\end{ex}

\subsubsection{The Ratio and Root Tests}
\begin{thm}[The Ratio Test]
    Given a series $\dsum a_n$, let:
    \[L = \lim_{n \to \infty} \left| \dfrac{a_{n+1}}{a_n} \right|.\]
    Then:
    \begin{itemize}
        \item If $L < 1$, the series converges absolutely.
        \item If $L > 1$, the series diverges.
        \item If $L = 1$, the test is inconclusive.
    \end{itemize}
\end{thm}


\begin{thm}[The Root Test]
    Given a series $\dsum a_n$, let:
    \[L = \lim_{n \to \infty} \sqrt[n]{|a_n|}.\]
    Then:
    \begin{itemize}
        \item If $L < 1$, the series converges absolutely.
        \item If $L > 1$, the series diverges.
        \item If $L = 1$, the test is inconclusive.
    \end{itemize}
\end{thm}

\begin{rmk}
    \begin{enumerate}
        \item Note that we have \textbf{absolute convergence} when $L<1$. 

        \item We won't have time to prove this in class. If you're interested in seeing the proof, check Paul's online notes:
        \begin{itemize}
            \item \href{https://tutorial.math.lamar.edu/Classes/CalcII/RatioTest.aspx#Series_Ratio_Proof}{Ratio Test Proof}
            \item \href{https://tutorial.math.lamar.edu/classes/calcii/roottest.aspx#Series_Root_Proof}{Root Test Proof}
        \end{itemize}
        
        \item The case when $L = 1$ is more complicated, as there are examples where the series may converge absolutely, converge conditionally, or diverge. Consider the following examples for the Ratio Test:
        \begin{itemize}
            \item \textbf{Conditional Convergence}: For the series $\dsum_{n=1}^\infty \dfrac{(-1)^{n+1}}{n}$, we compute the limit:
            \[\lim_{n\to \infty} \left| \dfrac{\dfrac{(-1)^{n+2}}{n+1}}{\dfrac{(-1)^{n+1}}{n}} \right| = \lim_{n \to \infty} \dfrac{n}{n+1} = 1.\]
            
            \item \textbf{Absolute Convergence}: For the series $\dsum_{n=1}^\infty \dfrac{(-1)^{n+1}}{n^2}$, we compute the limit:
            \[\lim_{n\to \infty} \left| \dfrac{\dfrac{(-1)^{n+2}}{(n+1)^2}}{\dfrac{(-1)^{n+1}}{n^2}} \right| = \lim_{n \to \infty} \dfrac{n^2}{n^2 + 2n + 1} = 1.\]
    
            \item \textbf{Divergence}: For the series where $a_n = 1$ for all $n$, we observe divergence.
    
        \end{itemize}
        Try to come up with your own examples for the Root Test.
        

        \item Prototype for both tests are the geometric series:
        \begin{itemize}
            \item $L_{ratio} = \lim_{n\to \infty} \left| \dfrac{r^{n+1}}{r^n}\right| = \lim_{n\to \infty} |r| = |r|$.
            \item  $L_{root} = \lim_{n\to \infty} \sqrt[n]{|r|^n} = \lim_{n\to \infty} |r| = |r|$.
        \end{itemize}
        Recall that $|r| < 1$ corresponds to convergent series; and  that $|r| > 1$ corresponds to divergent series.
    \end{enumerate}
\end{rmk}


\subsubsection{Examples}
\begin{ex}
    $\dsum_{n=2}^\infty \dfrac{n^2}{(2n-1)!}$ 
    \[ L = \lim_{n\to \infty} \left| \dfrac{\dfrac{(n+1)^2}{(2(n+1)-1)!}}{\dfrac{n^2}{(2n-1)!}}\right| = \lim_{n\to \infty} \dfrac{(n+1)^2}{(2n+1)(2n) n^2} = 0 < 1.\]
    Hence the series converges absolutely by the Ratio Test. 
\end{ex}

\begin{ex}
    $\dsum_{n=2}^\infty \dfrac{(-1)^n}{n^2+1}$ 
    \[ L = \lim_{n\to \infty} \left| \dfrac{\dfrac{(-1)^{n+1}}{(n+1)^2+1}}{\dfrac{(-1)^n}{n^2+1}}\right| = \lim_{n\to \infty} \dfrac{n^2+1}{2n^2+2n+2} = 1.\]
    The Ratio Test makes no conclusion. 
\end{ex}

Instead, one can use the Alternating Series Test to conclude that this series converges and the Comparison Test (with $A_n = \dfrac{1}{n^2+1} \leq B_n = \dfrac{1}{n^2}$) for absolute convergence.

\begin{ex}
    $\dsum_{n=0}^\infty \left( \dfrac{3n+1}{4-2n} \right)^{2n}$ 
    \[ L = \lim_{n\to \infty} \left|\sqrt[n]{\left( \dfrac{3n+1}{4-2n} \right)^{2n}}\right| = \lim_{n\to \infty} \left( \dfrac{3n+1}{4-2n} \right)^{2} = \lim_{n\to \infty} \dfrac{9n^2+6n+1}{4n^2-16n+16} = \dfrac{9}{4} > 1.\]
    Hence the series diverges absolutely by the Root Test. \chcomment{Typo22}
\end{ex}


\begin{ex}
    $\dsum_{n=4}^\infty \left(1+\dfrac{1}{n} \right)^{-n^2}$ 
    \[ L = \lim_{n\to \infty} \left| \dfrac{\dfrac{(n+1)^2}{(2(n+1)-1)!}}{\dfrac{n^2}{(2n-1)!}}\right| = \lim_{n\to \infty} \dfrac{(n+1)^2}{(2n+1)(2n) n^2} = 0 < 1.\]
    Hence the series converges absolutely by the Root Test. 
\end{ex}
For strategy of choosing convergence tests, see "Supplementary Resources" on course webpage.


\subsection{Power Series} \chcomment{\S11.8} \chcomment{Week 12}
\begin{center}
\begin{tcolorbox}
    \begin{itemize}
        \item \textbf{New Concept}: 
        \begin{itemize}
            \item Power Series Centered at $a$: $\dsum_{n=0}^\infty c_n (x-a)^n$
            \item Radius of Convergence 
            \item Interval of Convergence
        \end{itemize}
    \end{itemize}
\end{tcolorbox}
\end{center}
\begin{defn}
    A \bfemph{power series centered at $a$} is a series of the form
    \[\sum_{n=0}^\infty c_n (x-a)^n = c_0 + c_1 (x-a) + c_2 (x-a)^2 + \dots\]
    Here, $x$ is a variable, and $c_n$ are coefficients.
\end{defn}

\begin{ex}
    Take $a=0$, then 
    \[\sum_{n=0}^\infty c_n x^n = c_0 + c_1 x + c_2 x^2 + \cdots + c_n x^n + \cdots.\]
    This is a polynomial of infinite degree.
    Moreover if $c_n = 1$ for all $n$, then 
    \[f(x) = 1 + x + x^2 + \cdots = \sum_{n=0}^\infty x^n.\]
    This is a geometric series, we know it converges when $|x|<1$.
\end{ex}

The above example shows that a power series may converge for some values of $x$ and diverge for others. We use convergence tests to determine this.


\begin{ex}
    When does $\dsum_{n=0}^\infty \dfrac{(x-3)^n}{n}$ converges?

    Using the Ratio Test:
    \[ L = \lim_{n\to \infty} \left| \dfrac{\dfrac{(x-3)^{n+1}}{n+1}}{\dfrac{(x-3)^n}{n}}\right| = \lim_{n\to \infty} \dfrac{|x-3|}{1+\dfrac{1}{n}} = |x-3|.\]
    Hence the series converges absolutely when $|x-3|<1$ (i.e. $2<x<4$) and $|x-3|>1$ (i.e. $x<2$ or $x>4$) diverges by the Ratio Test. 
    
    Now we analysis the boundary cases:
    
    \begin{itemize}
        \item When $x=2$, $\sum a_n = \dfrac{(-1)^n}{n}$ converges.

        \item When $x=4$, $\sum a_n = \dfrac{1}{n}$ diverges.
    \end{itemize}

    Conclusion: the series converges when $x \in [2,4)$.
\end{ex}

\begin{thm}
    For a power series $\sum c_n (x-a)^n$, there are three possibilities:
    \begin{enumerate}
        \item The series converges only at $x=a$.
        \item The series converges for all $x$.
        \item There exists $R > 0$ such that the series converges for $|x-a| < R$ and diverges for $|x-a| > R$.
    \end{enumerate}
\end{thm}

\begin{defn}
    The number $R$ is called the \bfemph{radius of convergence}. The \bfemph{interval of convergence} is the interval that consists of all values of $x$ for which the power series converges.
\end{defn}

\begin{ex}
    For the series $\dsum \dfrac{(x-3)^n}{n}$, the radius of convergence is $R = 2$, and the interval of convergence is $[2,4)]$.
\end{ex}

\begin{ex}  \chcomment{Week 13}
    Compute the radius of converges and integral of converges for $\dsum_{n=0}^\infty \dfrac{n(x+2)^n}{3^n}$.

    Using the Ratio Test:
    \[ L = \lim_{n\to \infty} \left| \dfrac{\dfrac{(n+1)(x+2)^{n+1}}{3^{n+1}}}{\dfrac{n(x+2)^n}{3^n}}\right| = \lim_{n\to \infty} \dfrac{|x+3|}{3\left(1+\dfrac{1}{n}\right)} = \dfrac{|x+2|}{3}.\]
    The series converges when $\dfrac{|x+2|}{3}< 1$, so the radius of converges is $R=3$.
    
    Now we analysis the boundary cases:
    
    \begin{itemize}
        \item When $x=-5$, $\dsum_{n=0}^\infty  a_n = \dsum_{n=0}^\infty (-1)^n n$ diverges by the divergence test. (The alternating series test will not work here as the third condition fails.)

        \item When $x=1$, $\dsum_{n=0}^\infty  a_n = \dsum_{n=0}^\infty n$ diverges by the divergence test. 
    \end{itemize}

    So the interval of convergence is $x \in (-5,1)$.
\end{ex}

\subsubsection{Steps to Find Radius of Convergence and Interval of Convergence}
\begin{enumerate}
    \item \textbf{Identify the power series}: $\dsum_{n=0}^\infty c_n (x - a)^n$.

    \item \textbf{Apply the Ratio/Root Test}:
    \[
    L = \lim_{n \to \infty} \left| \dfrac{c_{n+1}(x-a)^{n+1}}{c_n(x-a)^n} \right| = |x-a| \cdot \lim_{n \to \infty} \left| \dfrac{c_{n+1}}{c_n} \right| \quad \text{or}\quad \lim_{n \to \infty} |c_n(x-a)^n|^{1/n} = |x-a| \cdot \lim_{n \to \infty} |c_n|^{1/n}.
    \]
    Solve for $L < 1$ and find the radius of convergence $R$.
Notice that:
\begin{itemize}
    \item If $L = \infty$, then $R = 0$ and the interval of convergence $I$ is empty.
    \item If $L = 0$, then $R = \infty$, and the interval of convergence $I$ is $(-\infty, \infty)$.
    \item If $L \in (0, \infty)$, then the radius of convergence is given by: $R = \dfrac{1}{L}$.
\end{itemize}
\end{enumerate}

If $R \in (0, \infty)$,  we continue:
\begin{enumerate}[resume]
    \item The series converges absolutely for $|x - a| < R$, i.e., on the open interval $(a - R, a + R)$.

    \item \textbf{Test the endpoints $x = a \pm R$:} Substitute each endpoint into the original series. Apply appropriate convergence tests (e.g., Alternating Series Test, $p$-series Test, or Comparison Test).
    
    \item \textbf{Find the interval of convergence}. It may be one of the following four possibilities: 
    \[(a - R, a + R), \quad  [a - R, a + R), \quad (a - R, a + R], \quad  [a - R, a + R].\]
\end{enumerate}

\subsection{Representation of Functions by Power Series} \chcomment{\S11.9}

\begin{center}
\begin{tcolorbox}
    \begin{itemize}
        \item \textbf{Representing functions as power series $\dsum_{n=0}^\infty c_n (x-a)^n$.}
        \item \textbf{Tools to use}: \begin{itemize}
            \item Substitution: $\dfrac{1}{1-u(x)} = \sum_{n=0}^\infty u(x)^n$ for $|u(x)| < 1$.
            \item Term-by-Term Differentiation: $f'(x) = \dsum_{\red{n=1}}^\infty n c_n (x-a)^{n-1}$ for $|x-a| < R$.
            \item Term-by-Term Integration
            $\dint f(x) \dd x = C + \dsum_{\red{n=0}}^\infty n c_n \dfrac{(x-a)^{n+1}}{n+1}$ for $|x-a| < R$. 
        \end{itemize}
    \end{itemize}
\end{tcolorbox}
\end{center}

In this section, we will learn how to represent some functions as power series. An application of this technique is the approximation of certain integrals that do not have elementary antiderivatives.

We start by discussing how to find the power series representation through substitution, integration, and differentiation.

\subsubsection{Deriving Power Series via Substitution}
Recall we have seen that
\[\dfrac{1}{1-u} = \sum_{n=0}^\infty u^n, \quad \text{for } |u| < 1.\]

\begin{ex}
    Find the power series for $\dfrac{1}{1+x^2}$.
    \[
    \dfrac{1}{1+x^2} = \dfrac{1}{1-(-x^2)} = \sum_{n=0}^\infty (-x^2)^2 = \sum_{n=0}^\infty (-1)^n x^{2n}, \quad \text{for } |x| < 1.
    \]
    Take $u = (-x^2)$, then $|u| = |-x^2| = x^2 < 1$. So we have $|x| < 1$.
\end{ex}

\begin{ex}
    Find the power series for $\dfrac{1}{2+x}$.
    \begin{align*}
        \dfrac{1}{2+x} &= \dfrac{1}{2} \dfrac{1}{1+\dfrac{x}{2}} = \dfrac{1}{2} \dfrac{1}{1 - \left(-\dfrac{x}{2}\right)}\tag{If $|x|<2$, then $|u| = \left|\dfrac{x}{2}\right|<1$, we may use the Equation of $\dfrac{1}{1-u}$.}\\
        & = \dfrac{1}{2} \sum_{n=0}^\infty \left(-\dfrac{x}{2}\right)^n = \sum_{n=0}^\infty \dfrac{(-1)^n}{2^{n+1}} (x)^n.
    \end{align*}
\end{ex}

\subsubsection{Term-by-Term Differentiation and Integration}

\begin{thm}
    If $\dsum_{n=0}^\infty c_n (x-a)^n$ has radius of convergence $R>0$, then $f(x) = \dsum_{n=0}^\infty c_n (x-a)^n$ is differentiable within $(a-R, a+R)$.
    \begin{align*}
        f'(x) &= \dsum_{\red{n=1}}^\infty n c_n (x-a)^{n-1},\\
        \int f(x) \dd x &= C + \dsum_{\red{n=0}}^\infty n c_n \dfrac{(x-a)^{n+1}}{n+1}.
    \end{align*}
\end{thm}
\begin{proof}
    One can prove this by computing the differentiation:
    \[
    \dfrac{d}{\dd x} \left( \sum_{n=0}^\infty c_n (x-a)^n \right) = \sum_{n=1}^\infty n c_n (x-a)^{n-1}, \quad \text{for } |x-a| < R.
    \]
    and the integration:
    \[
    \int \sum_{n=0}^\infty c_n (x-a)^n \dd x = C + \sum_{n=0}^\infty \dfrac{c_n}{n+1} (x-a)^{n+1}, \quad \text{for } |x-a| < R.
    \]
\end{proof}

\subsubsection{Examples}
\begin{ex}
    \begin{align*}
        \dfrac{1}{(1-x)^2} = \dfrac{\dd x}{\dd x} \left(\dfrac{1}{1-x} \right) = \dfrac{\dd x}{\dd x} \sum_{n=0}^\infty x^n = \sum_{\red{n=1}}^\infty n x^{n-1} \text{ when } |x| < 1.
    \end{align*}
\end{ex}


\begin{ex}
    Recall by the Fundamental Theorem of Calculus, 
    \[ \ln(1+x) - \ln (1+0) = \int_0^x \dfrac{1}{1+t} \dd t.\] 
    This implies (note that $\ln (1+0) = 0$) for $|x| < 1$,
    \begin{align*}
        \ln(1+x) &= \int_0^x \dfrac{1}{1-(-t)} \dd t = \int_0^x \sum_{n=0}^\infty (-t)^n \dd t \\
        &= \sum_{n=0}^\infty  \int_0^x (-1)^n t^n \dd t = \sum_{n=0}^\infty (-1)^n \left[ \dfrac{t^{n+1}}{n+1}\right]_{t=0}^x = \sum_{n=0}^\infty (-1)^n  \dfrac{x^{n+1}}{n+1}.
    \end{align*}
    Thus:
    \[\ln(1+x)= \sum_{n=0}^\infty (-1)^n \dfrac{x^{n+1}}{n+1}, \quad \text{for } |x| < 1.\]
\end{ex}

\begin{ex}
Another solution for solving $\ln(1+x)$.
\begin{align*}
    \ln(1+x) &= \int \sum_{n=0}^{\infty} (-1)^n x^n \dd x \tag{Take $u = -t$, need $|u| = |-t|<1$, i.e. $|t|<1$}\\
    &=\int \sum_{n=0}^{\infty} (-1)^n x^n \dd x =\sum_{n=0}^{\infty} (-1)^n \int x^n\dd x =\sum_{n=0}^{\infty} (-1)^n \dfrac{x^{n+1}}{n+1} + C, \quad \text{when } |x| < 1.
\end{align*}
To determine $C$, take $x = 0$, we have
\[\ln(1+0) = 0 = C.\]
\end{ex}

\begin{ex}[$\arctan(x)$]
the Fundamental Theorem of Calculus, 
    \[ \arctan(x) - \arctan(0) = \int_0^x \dfrac{1}{1+t^2} \dd t.\] 
    This implies (note that $\arctan(0) = 0$) for $|x| < 1$,
    \begin{align*}
        \arctan(x) &= \int_0^x \dfrac{1}{1+t^2} \dd t = \int_0^x \sum_{n=0}^\infty (-t^2)^n \dd t \\
        &= \sum_{n=0}^\infty  \int_0^x (-1)^n t^{2n} \dd t = \sum_{n=0}^\infty (-1)^n \left[ \dfrac{t^{2n+1}}{2n+1}\right]_{t=0}^x \\
        &= \sum_{n=0}^{\infty} (-1)^n \dfrac{x^{2n+1}}{2n+1}, \quad \text{when } |x| < 1.
    \end{align*}
    Thus:
    \[\arctan(x) = \sum_{n=0}^{\infty} (-1)^n \dfrac{x^{2n+1}}{2n+1}, \quad \text{for } |x| < 1.\]
\end{ex}

\begin{ex}
Another solution for solving $\arctan(x)$.
\begin{align*}
    \arctan(x) &= \int \dfrac{1}{1+x^2} 
    = \int \sum_{n=0}^{\infty} (-x^2)^n \dd x = \sum_{n=0}^{\infty} (-1)^n \int x^{2n} \dd x\\
    &= \int \sum_{n=0}^{\infty} (-1)^n \dfrac{x^{2n+1}}{2n+1}\dd x + C, \quad \text{when } |x| < 1.
\end{align*}
To determine $C$, take $x = 0$. We have
\[\arctan(0) = 0 = C.\]
\end{ex}

\subsection{Taylor and Maclaurin Series} \chcomment{\S11.10} \chcomment{Week 14}
\begin{center}
\begin{tcolorbox}
    \begin{itemize}
        \item \textbf{Taylor series}: 
        \[f(x) = \sum_{n=0}^{\infty} \dfrac{f^{(n)}(a)}{n!}(x-a)^n \quad |x-a| < R. \]
        \item \textbf{Maclaurin series}: Taylor series centered at 0.
    \end{itemize}
\end{tcolorbox}
\end{center}
\begin{thm}
    Suppose the function $f(x)$ has a power series representation at $a$ given by:
    \[f(x) = \sum_{n=0}^{\infty} \dfrac{f^{(n)}(a)}{n!}(x-a)^n, \quad |x-a| < R.\]
    Then $c_n = \dfrac{f^{(n)}(a)}{n!}$.
\end{thm}

\begin{proof}
    We compute derivatives:
    \begin{align*}
        f'(x) &= \sum_{n=1}^{\infty} c_n n (x-a)^{n-1}, \\
        f''(x) &= \sum_{n=2}^{\infty} c_n n (n-1) (x-a)^{n-2}.
    \end{align*}
    Taking $x = a$ yields
    \begin{align*}
        f'(a) &= c_1, \tag{$C_1$ is the only non-vanishing term}\\
        f''(a) &= 2!c_2,\\
        &\vdots\\
        f^{(n)}(a) &= n!c_n.
    \end{align*}
\end{proof}
\begin{defn}
    We define the \bfemph{Taylor series} of $f$ centered at $a$ as:
    \[T_f(x) = \sum_{n=0}^{\infty} \dfrac{f^{(n)}(a)}{n!}(x-a)^n, \quad |x-a| < R.\]

    When $a=0$, this is called the \bfemph{Maclaurin series}:
    \[T_f(x) = \sum_{n=0}^{\infty} \dfrac{f^{(n)}(0)}{n!}x^n.\]
\end{defn}

\subsubsection{Examples of Maclaurin Series}
\begin{ex}[$f(x) = e^x$ at $a = 0$]
    The derivatives of $f(x)$ are given by 
    \[f^{(n)}(x) = e^x \quad \text{for all } n.\]
    So $e^x = \dsum_{n=0}^{\infty} \dfrac{x^n}{n!}$. We compute the radius of convergence:
    \[L = \lim_{n\to \infty} \left| \dfrac{\dfrac{x^{n+1}}{(n+1)!}}{\dfrac{x^n}{n!}}\right| = 0 < 1.\]
    The radius of convergence is $\infty$.
\end{ex}

\begin{ex}[$f(x) = \sin(x)$ at $a = 0$]
    The derivatives of $f(x)$ are given by 
    \[f'(x) = \cos(x), \quad f''(x) = -\sin(x), \quad f'''(x) = -\cos(x), \quad f^{(4)}(x) = \sin(x).\]
    Higher order derivatives repeat. So 
    \[\sin(x) = \sum_{n=0}^{\infty} (-1)^n \dfrac{x^{2n+1}}{(2n+1)!}.\]
    (Note that $\sin$ is an odd function).
    The radius of convergence $R = \infty$, as
    \[L = \lim_{n\to \infty} \left| \dfrac{\dfrac{x^{2n+3}}{(2n+3)!}}{\dfrac{x^{2n+1}}{(2n+1)!}}\right| =\lim_{n\to \infty} \left| \dfrac{x^2}{(2n+3)(2n+2)}\right|= 0 < 1.\]
\end{ex}

\begin{ex}[$f(x) = \cos(x)$ at $a = 0$]    
    Check that \[f'(x) = -\sin(x), \quad f''(x) = -\cos(x), \quad f'''(x) = \sin(x), \quad f^{(4)}(x) = \cos(x).\]
    So \[\cos(x) = \sum_{n=0}^{\infty} (-1)^n \dfrac{x^{2n}}{(2n)!}, \quad \text{even function}. \]
    The radius of convergence is again $R = \infty$.
\end{ex}

\begin{ex}[$f(x) = x^4 e^{-3x^2}$ about $x = 0$]

    We know the Taylor series expansion of $e^u$ about $u = 0$ is given by $e^u = \sum_{n=0}^{\infty} \dfrac{u^n}{n!}$.
    
    Substituting $u = -3x^2$, we obtain:
    \[
    e^{-3x^2} = \sum_{n=0}^{\infty} \dfrac{(-3x^2)^n}{n!} = \sum_{n=0}^{\infty} \dfrac{(-3)^n x^{2n}}{n!}.
    \]
    
    Multiplying by $x^4$, we get:
    \[
    x^4 e^{-3x^2} = x^4 \sum_{n=0}^{\infty} \dfrac{(-3)^n x^{2n}}{n!} = \sum_{n=0}^{\infty} \dfrac{(-3)^n x^{2n + 4}}{n!}.
    \]
    
    Therefore, 
    \[f(x) = \sum_{n=0}^{\infty} \dfrac{(-3)^n}{n!} x^{2n + 4}.\]
\end{ex}
\subsubsection{Examples of Taylor Series}

\begin{ex}[Taylor Series for $f(x) = \dfrac{1}{x^2}$ about $x = -1$.]

The derivatives of $f(x)$ evaluated at $x = -1$ are given by
\begin{align*}
    f(x) &= \dfrac{1}{x^2}, & f^{(0)}(-1) &= \dfrac{1}{(-1)^2} = 1, \\
    f^{(1)}(x) &= -\dfrac{2}{x^3}, & f^{(1)}(-1) &= -\dfrac{2}{(-1)^3} = 2, \\
    f^{(2)}(x) &= \dfrac{2 \cdot 3}{x^4}, & f^{(2)}(-1) &= \dfrac{6}{1} = 6, \\
    f^{(3)}(x) &= -\dfrac{2 \cdot 3 \cdot 4}{x^5}, & f^{(3)}(-1) &= \dfrac{24}{-1} = -24, \\
    &\vdots \\
    f^{(n)}(x) &= (-1)^n (n+1)! \, x^{-(n+2)}, & f^{(n)}(-1) &= (-1)^n (n+1)! \cdot (-1)^{-(n+2)} = (n+1)!
\end{align*}

So the Taylor series for $f(x)$ about $x = -1$ is given by
\[f(x) = \sum_{n=0}^{\infty} \dfrac{f^{(n)}(-1)}{n!}(x + 1)^n = \sum_{n=0}^{\infty} \dfrac{(n+1)!}{n!}(x + 1)^n = \sum_{n=0}^{\infty} (n+1)(x + 1)^n.\]
\end{ex}
\begin{ex}[Taylor Series for $f(x) = 7x^2 - 6x + 1$ about $x = 2$.]

The derivatives of $f(x)$ evaluated at $x = 2$ are given by
\begin{align*}
    f(x) &= 7x^2 - 6x + 1, & f(2) &= 28 - 12 + 1 = 17, \\
    f'(x) &= 14x - 6, & f'(2) &= 28 - 6 = 22, \\
    f''(x) &= 14, & f''(2) &= 14, \\
    f^{(n)}(x) &= 0 \quad \text{for } n \geq 3, & f^{(n)}(2) &= 0.
\end{align*}

The Taylor series is then
\[ f(x) = \sum_{n=0}^{\infty} \dfrac{f^{(n)}(2)}{n!} (x - 2)^n 
= 17 + 22(x - 2) + \dfrac{14}{2}(x - 2)^2 
= 17 + 22(x - 2) + 7(x - 2)^2.\]
When $f(x)$ is a polynomial of degree $s$, its Taylor series about any point $x=a$ terminates after the $d$th derivative term. 
\end{ex}

\subsubsection{List of Common Maclaurin Series}
\begin{center}
\begin{tcolorbox}
    \begin{align*}
        \dfrac{1}{1-x} &= \sum_{n=0}^{\infty} x^n, \quad |x| < 1. \\
        e^x &= \sum_{n=0}^{\infty} \dfrac{x^n}{n!}, \quad |x| < \infty. \\
        \sin(x) &= \sum_{n=0}^{\infty} (-1)^n \dfrac{x^{2n+1}}{(2n+1)!}, \quad |x| < \infty. \\
        \cos(x) &= \sum_{n=0}^{\infty} (-1)^n \dfrac{x^{2n}}{(2n)!}, \quad |x| < \infty. \\
        \ln(1+x) &= \sum_{n=1}^{\infty} (-1)^{n+1} \dfrac{x^n}{n}, \quad |x| < 1. \\
        \arctan(x) &= \sum_{n=0}^{\infty} (-1)^n \dfrac{x^{2n+1}}{2n+1}, \quad |x| < 1.
    \end{align*}
\end{tcolorbox}
\end{center}

\subsection{Applications of Taylor Polynomials} \chcomment{\S11.11}

\begin{center}
\begin{tcolorbox}
    Let $T_N(x) = \dsum_{n=0}^N \dfrac{f^{(n)}(a)}{n!} (x-a)^n$
    \begin{itemize}
        \item \textbf{Estimating Integrals:}  
        \[\dint_a^b f(x) \dd x \approx \dint_a^b T_N(x) \dd x.\]
        
        \item \textbf{Estimating Functions:} 
        \[f(x) \approx T_N(x).\]
        
        \item \textbf{Error bound:} 
        \[|R_N(x)| = |f(x) - T_N(x)| \leq \dfrac{M}{(N+1)!}|x - a|^{N+1}\]
        where $M \geq |f^{(N+1)}(\xi)|$ on the interval.
    \end{itemize}
\end{tcolorbox}
\end{center}


\subsubsection{Taylor Polynomials}
The Taylor polynomial provides a powerful method for approximating smooth functions near a given point. Given a function $f(x)$ that is sufficiently differentiable at a point $a$, its degree $N$ Taylor polynomial centered at $a$ is defined as  
\[
T_N(x) = \sum_{n=0}^N \frac{f^{(n)}(a)}{n!} (x-a)^n = f(a) + f'(a)(x - a) + \frac{f''(a)}{2!}(x - a)^2 + \cdots + \frac{f^{(N)}(a)}{N!}(x - a)^N.
\]
This polynomial approximates the behavior of $f(x)$ near $a$ to order $N$, and forms the foundation for various analytical techniques. In particular, Taylor approximations can be employed to estimate definite integrals, evaluate limits, and simplify complex expressions in both theoretical and applied contexts.

\begin{ex}[Taylor Polynomials of $f(x) = \ln(1 - x)$ about $x = -2$]

We compute the Taylor series of $f(x) = \ln(1 - x)$ centered at $x = -2$.  
We begin by computing derivatives of $f(x)$ evaluated at $x = -2$:

\begin{alignat*}{2}
f(x) &= \ln(1 - x),       &\quad f(-2) &= \ln(3), \\
f'(x) &= -\dfrac{1}{1 - x}, &\quad f'(-2) &= -\dfrac{1}{3}, \\
f''(x) &= -\dfrac{1}{(1 - x)^2}, &\quad f''(-2) &= -\dfrac{1}{9}, \\
f^{(3)}(x) &= -\dfrac{2}{(1 - x)^3}, &\quad f^{(3)}(-2) &= -\dfrac{2}{27}, \\
f^{(4)}(x) &= -\dfrac{6}{(1 - x)^4}, &\quad f^{(4)}(-2) &= -\dfrac{6}{81} = -\dfrac{2}{27}.
\end{alignat*}

Hence, the Taylor polynomials are:

\[
\begin{aligned}
T_2(x) &= \ln(3) - \frac{1}{3}(x + 2) - \frac{1}{18}(x + 2)^2, \\
T_3(x) &= \ln(3) - \frac{1}{3}(x + 2) - \frac{1}{18}(x + 2)^2 - \frac{1}{81}(x + 2)^3, \\
T_4(x) &= \ln(3) - \frac{1}{3}(x + 2) - \frac{1}{18}(x + 2)^2 - \frac{1}{81}(x + 2)^3 - \frac{1}{162}(x + 2)^4.
\end{aligned}
\]
\end{ex}
\begin{figure}[H]
        \centering
        \resizebox{\textwidth}{!}{\begin{tikzpicture}
\begin{axis}[
    width=12cm, height=8cm,
    domain=-4:0,
    samples=200,
    legend style={at={(1.05,1)}, anchor=north west},
    xlabel={$x$}, ylabel={$y$},
    title={},
    grid=both,
    axis lines=middle,
]
\addplot [black, thick] {ln(1 - x)};
\addlegendentry{\( f(x) = \ln(1 - x) \)}

\addplot [akabeni, thick] {ln(3) - (1/3)*(x + 2) - (1/18)*(x + 2)^2};
\addlegendentry{\( T_2(x) \)}

\addplot [kohaku, thick] {ln(3) - (1/3)*(x + 2) - (1/18)*(x + 2)^2 - (1/81)*(x + 2)^3};
\addlegendentry{\( T_3(x) \)}

\addplot [shinbashi, thick] {ln(3) - (1/3)*(x + 2) - (1/18)*(x + 2)^2 - (1/81)*(x + 2)^3 - (1/162)*(x + 2)^4};
\addlegendentry{\( T_4(x) \)}
\end{axis}

\end{tikzpicture}} % Include your TikZ file
        \caption{Taylor Approximations of $\ln(1 - x)$ at $x = -2$.}
        \label{fig:Taylor polynomial}
    \end{figure}
 

\subsubsection{Error Bound}
When approximating function using the Taylor polynomial $T_N(x)$, the error in this approximation is given by:
\[R_N(x) = f(x) - T_N(x).\]
\begin{thm}
    Given that $f$ satisfies the hypotheses of Taylor's theorem, and there exists a real number $M$ such that
    \[|f^{(N+1)}(x)| \leq M \quad \text{for all} \quad x \in I = [a,b],\]
    the remainder \( R_N(x) \) satisfies the \textbf{Taylor's inequality} 
    \[|R_N(x)| \leq \frac{M |x - a|^{N+1}}{(N+1)!}\]
    on the same interval $I$.
\end{thm}  

\begin{ex}
Approximate $f(x) = \sqrt[3]{x}$ using the second-degree Taylor polynomial $T_2(x)$ centered at $x=8$, and find an error bound for $x \in [7,9]$.

We compute:
\begin{alignat*}{2}
f(x) &= \sqrt[3]{x}, &\quad f(8) &= 2, \\
f'(x) &= \dfrac{1}{3}x^{-2/3}, &\quad f'(8) &= \dfrac{1}{12}, \\
f''(x) &= -\dfrac{2}{9}x^{-5/3}, &\quad f''(8) &= -\dfrac{1}{144}.
\end{alignat*}

The second-degree Taylor polynomial is:
\[
T_2(x) = f(8) + f'(8)(x-8) + \dfrac{f''(8)}{2}(x-8)^2 = 2 + \dfrac{1}{12}(x - 8) - \dfrac{1}{288}(x - 8)^2.
\]

To estimate the error, we use Taylor's inequality:
\[
|R_2(x)| \leq \dfrac{M |x - 8|^3}{3!},
\]
where \( M \) is an upper bound for \( |f^{(3)}(x)| = \left| \dfrac{10}{27}x^{-8/3} \right| \) on \( [7,9] \).

Since \( x^{-8/3} \) decreases as \( x \) increases, the maximum occurs at \( x = 7 \). Therefore,
\[
M = \dfrac{10}{27} \cdot 7^{-8/3} \approx 0.0104.
\]

Thus, the error is bounded by:
\[
|R_2(x)| \leq \dfrac{0.0104}{6} \cdot |x - 8|^3 \leq \dfrac{0.0104}{6} \cdot 1^3 \approx 0.00173.
\]
\end{ex}


   
\subsubsection{Estimating Integrals}
\begin{ex}
Estimate \(\displaystyle \int_0^1 e^{-x^2} \dd x\) using the 5th-degree Taylor polynomial \(T_5(x)\) centered at 0. Find an error bound of the estimation using the fact that
\[|f^{(6)}(x)| \leq 28 \quad \text{for } x \in [0,1].\]


\textit{Step 1}. Using substitution $u = -x^2$, the Maclaurin series of \(e^{-x^2}\) is 
\[
e^{-x^2} = \sum_{k=0}^{\infty} \frac{(-1)^k x^{2k}}{k!} = 1 - x^2 + \frac{x^4}{2!} - \frac{x^6}{3!} + \cdots, \quad |x| < \infty.
\]
Therefore,
\[
T_5(x) = 1 + 0 \cdot x - x^2 + 0 \cdot x^3 + \frac{x^4}{2} + 0 \cdot x^5 =  1 - x^2 + \frac{x^4}{2}.
\]
\begin{rmk}\leavevmode
    \begin{enumerate}
        \item The Taylor polynomial centered at 0 is by definition
        \[T_5(x) = \sum_{n=0}^{\infty} \frac{f^{(n)} x^n}{n!}.\] 
        In the Taylor expansion we computed for $e^{-x^2}$, $n = 2k$. 
        \item Be careful, the solution could ask you to use the first five nonvanishing terms, which corresponds to $T_8(x)$.
    \end{enumerate}
\end{rmk}

\textit{Step 2}. We approximate:
\begin{align*}
    \int_0^1 e^{-x^2} \dd x &\approx \int_0^1 T_5(x) \dd x = \int_0^1 \left(1 - x^2 + \frac{x^4}{2} \right) \dd x\\
    &= \left[x - \frac{x^3}{3} + \frac{x^5}{10} \right]_0^1 = 1 - \frac{1}{3} + \frac{1}{10} = \frac{23}{30}.
\end{align*}

\textit{Step 3. Error bound via Taylor’s theorem.}

Let \(f(x) = e^{-x^2}\), and note that the remainder after degree \(5\) on the interval \([0,1]\) bounded by:
\[|R_5(x)| \leq \frac{M x^6}{6!}, \text{ with } M = \max_{x\in[0,1]} |f^{(6)}(x)|.\]
So the error in the integral is bounded by:
\[\left| \int_0^1 f(x) - T_5(x) \dd x \right| \leq  \int_0^1 |f(x) - T_5(x)| \dd x = \int_0^1 \left| R_5(x) \right| \dd x \leq \frac{M}{6!} \int_0^1  x^6\dd x.\]

\begin{rmk}
    Note that $(N+1)! = 6!$ in the Taylor inequality.
\end{rmk}
Since $|f^{(6)}(x)| \leq 28 \quad \text{for } x \in [0,1]$,
\[
\left| \int_0^1 f(x) - T_5(x) \dd x \right| \leq \frac{28}{6!} \int_0^1 x^6 \dd x = \frac{28}{6!} \cdot \frac{1}{7} = \frac{1}{180}.
\]

\gray{Aside: The maximum value of the function \( f^{(6)}(x) = (-240 + 720x^2 - 480x^4 + 64x^6) e^{-x^2} \) on the interval \([0, 1]\) occurs at approximately \( x = 0.905 \), with a maximum value of approximately 27.72.}
\end{ex}

\subsubsection{Evaluating Limits}  \chcomment{Week 15}
We can use the Taylor expansion to evaluate a limit. This approach is similar in spirit to applying L'Hôpital's Rule, where we differentiate the numerator and denominator when encountering an indeterminate form, such as $\frac{0}{0}$. By expanding the functions in the limit as Taylor series around the point of interest, we can often simplify the expression and evaluate the limit directly.
\begin{ex}    
\[\dlim_{x \to 0} \dfrac{\ln(1+x)- x}{\cos x - (1+x^3)^{-1/2}}\]
    For $(1+x^3)^{-1/2}$, we have
    \begin{align*}
    (1+x^3)^{-1/2} &= \dsum_{n=0}^\infty {-\frac{1}{2} \choose n} x^{3n} = 1 - \dfrac{1}{2} x^3 + \dfrac{-\dfrac{1}{2}(-\dfrac{1}{2}-1)}{2} x^6 + \cdots 
    \end{align*}
    We have computed the Maclaurin series of $\ln(1+x)$ and $\cos x$. So the numerator and the denominator are given by
    \begin{align*}
    (1+x^3)^{-1/2} &= \dsum_{n=0}^\infty {-\dfrac{1}{2} \choose n} x^{3n} = 1 - \dfrac{1}{2} x^3 + \dfrac{-\dfrac{1}{2}(-\dfrac{1}{2}-1)}{2} x^6 + \cdots \\
    \ln(1+x) - x &= \left( x - \dfrac{x^2}{2} + \dfrac{x^3}{3} - \dfrac{x^4}{4} + \cdots \right) - x = - \dfrac{x^2}{2} + \dfrac{x^3}{2} + \cdots \\
    \cos x - (1+x^3)^{-1/2} &= \left( 1 - \dfrac{x^2}{2!} + \dfrac{x^4}{4!} - \cdots \right) - \left( 1 - \dfrac{1}{2} x^3 + \cdots \right) = - \dfrac{x^2}{2} + \dfrac{x^3}{2} + \cdots.        
    \end{align*}
    Substitute the above we have
    \[\dlim_{x \to 0} \dfrac{\ln(1+x) - x}{\cos x - (1+x^3)^{-1/2}} = \dlim_{x \to 0} \dfrac{ - \dfrac{x^2}{2} + \dfrac{x^3}{2} + \cdots}{ - \dfrac{x^2}{2} + \dfrac{x^3}{2} + \cdots} = \dlim_{x \to 0} \dfrac{ - \dfrac{1}{2} + \dfrac{x}{2} + \cdots}{ - \dfrac{1}{2} + \dfrac{x}{2} + \cdots} = \dfrac{ - \dfrac{1}{2}}{ - \dfrac{1}{2}} = 1.\]
\end{ex}
\begin{ex}
    $\dlim_{x \to 0} \dfrac{\arctan(x) - \ln (1+x)}{1 - \cos x}$
    The Maclaurin series of the numerator is 
    \begin{align*}
        \arctan(x) - \ln (1+x) &= \left( x - \dfrac{x^3}{3} + \cdots \right) - \left( x - \dfrac{x^2}{2} + \dfrac{x^3}{3} - \cdots \right) \\
        &= \dfrac{x^2}{2} - \dfrac{2x^3}{3} + \cdots.
    \end{align*}
    Therefore
    \begin{align*}    
        \dfrac{\arctan(x) - \ln (1+x)}{1 - \cos x} &= \dfrac{ \dfrac{x^2}{2} - \dfrac{2x^3}{3} + \cdots }{\dfrac{x^2}{2} - \dfrac{x^4}{4!} + \cdots} =  \dfrac{ \dfrac{1}{2} - \dfrac{2x}{3} + \cdots }{\dfrac{1}{2} - \dfrac{x^2}{4!} + \cdots} \\
        \dlim_{x \to 0} \dfrac{\arctan(x) - \ln (1+x)}{1 - \cos x} &= \dlim_{x \to 0} \dfrac{ \dfrac{1}{2} - \dfrac{2x}{3} + \cdots }{\dfrac{1}{2} - \dfrac{x^2}{4!} + \cdots} = \dfrac{\dfrac{1}{2}}{\dfrac{1}{2}} = 1.
    \end{align*}
\end{ex}

\end{document}
